\documentclass[11pt,a4paper]{article}

% ===== 中文支持(关键)=====
\usepackage{ctex} % ⭐ 一行解决中文问题(Overleaf 已内建)

% ===== 其他套件 =====
\usepackage{geometry}
\usepackage{hyperref}
\usepackage{xcolor}
\usepackage{enumitem}
\usepackage{listings}
\usepackage{fancyhdr}
\usepackage{titlesec}

\geometry{margin=1in}


\hypersetup{
  colorlinks=true,
  linkcolor=blue,
  urlcolor=blue,
  citecolor=blue
}

\pagestyle{fancy}
\fancyhf{}
\lhead{LearnWeb3 First dApp}
\rhead{Code Walkthrough}
\cfoot{\thepage}

% ---- Styling ----
\titleformat{\section}{\Large\bfseries}{\thesection}{1em}{}
\titleformat{\subsection}{\large\bfseries}{\thesubsection}{1em}{}

% ---- Code listing setup ----
\lstdefinelanguage{HTML}{
  sensitive=true,
  morecomment=[s]{<!--}{-->},
  morestring=[b]",
  morekeywords={
    <!doctype,html,head,meta,title,style,body,div,h1,p,label,br,input,button,script
  }
}
\lstset{
  basicstyle=\ttfamily\small,
  breaklines=true,
  frame=single,
  rulecolor=\color{black!20},
  backgroundcolor=\color{black!2},
  numbers=left,
  numberstyle=\tiny\color{black!50},
  xleftmargin=2em,
  framexleftmargin=1.5em,
  tabsize=2,
  showstringspaces=false
}

% ---- Document ----
\begin{document}

\begin{center}
  {\LARGE \textbf{MoodDiary dApp:\;index.html 全部解析}}\\[6pt]
  {\small(本讲义为 Overleaf / \LaTeX\ 版本,可直接编译成 PDF)}
\end{center}
\vspace{0.5em}

\section{目标与整体概念}

本 dApp 的核心目标是:用前端网页通过 \textbf{MetaMask}(用户身份+签名)与 \textbf{Sepolia 测试网}上的 \textbf{MoodDiary 智能合约}交互,实现:

\begin{itemize}[leftmargin=1.5em]
  \item \textbf{Get Mood}:读取链上状态(\texttt{getMood()},\textbf{read / view},不弹钱包、不花 gas)
  \item \textbf{Set Mood}:写入链上状态(\texttt{setMood(string)},\textbf{write / transaction},弹钱包、需要 gas)
\end{itemize}

\section{1) HTML 骨架:网页外壳}

\begin{lstlisting}[language=HTML]
<!doctype html>
<html lang="en">
\end{lstlisting}

\begin{itemize}[leftmargin=1.5em]
  \item \texttt{<!doctype html>}:声明 HTML5,避免浏览器用旧渲染模式。
  \item \texttt{lang="en"}:页面语言标记(SEO / 辅助工具 / 翻译等会用到)。
\end{itemize}

\section{2) \texttt{<head>}:页面设定与样式}

\subsection{Meta 标签(显示与兼容性)}

\begin{lstlisting}[language=HTML]
<meta charset="UTF-8" />
<meta http-equiv="X-UA-Compatible" content="IE=edge" />
<meta name="viewport" content="width=device-width, initial-scale=1.0" />
<title>LearnWeb3 First dApp</title>
\end{lstlisting}

\begin{itemize}[leftmargin=1.5em]
  \item \texttt{charset="UTF-8"}:避免中文/特殊字符乱码。
  \item \texttt{X-UA-Compatible}:旧式兼容性声明(现代浏览器影响很小)。
  \item \texttt{viewport}:手机端显示比例正确(响应式基础)。
  \item \texttt{title}:浏览器标签页标题。
\end{itemize}

\subsection{CSS:纯前端排版(与区块链逻辑无关)}

\begin{lstlisting}[language=HTML]
<style>
  body { text-align: center; font-family: Arial, Helvetica, sans-serif; }
  div { width: 20%; margin: 0 auto; display: flex; flex-direction: column; }
  button { width: 100%; margin: 10px 0px 5px 0px; }
  input { padding: 8px; }
  #status { margin-top: 12px; font-size: 14px; opacity: 0.9; }
</style>
\end{lstlisting}

\begin{itemize}[leftmargin=1.5em]
  \item \texttt{div} 使用 \texttt{flex} + \texttt{column},让输入框/按钮纵向排列。
  \item \texttt{width:20\%} + \texttt{margin:0 auto}:让内容窄窄居中。
  \item \texttt{\#status}:专门显示“连接中 / 发交易 / 等确认 / 完成”等状态文字。
\end{itemize}

\section{3) \texttt{<body>}:UI 元素(用户会操作的部分)}

\subsection{输入框:用户输入 mood}

\begin{lstlisting}[language=HTML]
<input type="text" id="mood" placeholder="e.g. happy" />
\end{lstlisting}

\begin{itemize}[leftmargin=1.5em]
  \item \texttt{id="mood"}:后续 JS 用 \texttt{document.getElementById("mood").value} 读用户输入。
\end{itemize}

\subsection{按钮:调用 get / set}

\begin{lstlisting}[language=HTML]
<button onclick="getMood()">Get Mood</button>
<button onclick="setMood()">Set Mood</button>
\end{lstlisting}

\begin{itemize}[leftmargin=1.5em]
  \item 点击按钮会调用全局函数 \texttt{getMood()} / \texttt{setMood()}。
  \item \textbf{重点:}由于后面使用 \texttt{<script type="module">},模块作用域下的函数默认不是全局,因此需要显式挂到 \texttt{window.getMood} / \texttt{window.setMood},让 \texttt{onclick} 能找到。
\end{itemize}

\subsection{输出区:显示 mood 与状态}

\begin{lstlisting}[language=HTML]
<p id="showMood"></p>
<p id="status"></p>
\end{lstlisting}

\begin{itemize}[leftmargin=1.5em]
  \item \texttt{showMood}:显示链上读到的 mood(例如 \texttt{Your Mood: happy})。
  \item \texttt{status}:显示当前流程状态(连接钱包、发送交易、等待确认等)。
\end{itemize}

\section{4) \texttt{<script type="module">}:dApp 核心逻辑}

\subsection{4.1 为什么要用 module?}

\begin{lstlisting}[language=HTML]
<script type="module">
\end{lstlisting}

\begin{itemize}[leftmargin=1.5em]
  \item 允许使用 ES Module 的 \texttt{import ... from ...} 直接引入 viem。
  \item 允许顶层使用 \texttt{await}(例如请求钱包地址)。
  \item \textbf{副作用:}模块内变量/函数默认不挂在全局,若要让 HTML 的 \texttt{onclick="..."} 调到,需要 \texttt{window.getMood = ...}。
\end{itemize}

\section{5) 引入 viem:区块链接口工具}

\begin{lstlisting}[language=JavaScript]
import {
  createWalletClient,
  createPublicClient,
  custom,
  getContract,
} from "https://esm.sh/viem";
import { sepolia } from "https://esm.sh/viem/chains";
\end{lstlisting}

\begin{itemize}[leftmargin=1.5em]
  \item \texttt{createWalletClient}:用于 \textbf{写入}(需要钱包签名的交易)。
  \item \texttt{createPublicClient}:用于 \textbf{读取}与等待交易确认(不需要签名)。
  \item \texttt{custom(window.ethereum)}:使用 MetaMask 注入的 provider 作为 transport(桥接到链)。
  \item \texttt{getContract}:用 ABI+地址创建合约实例,支持 \texttt{.read} / \texttt{.write} 调用。
  \item \texttt{sepolia}:链配置(chainId、网络参数等)。
\end{itemize}

\section{6) 合约地址与 ABI:告诉前端“找谁、怎么叫它”}

\subsection{合约地址}

\begin{lstlisting}[language=JavaScript]
const MoodContractAddress = "0x8089Ad39704F3Bc5f9b85C4BE4427631A137e58e";
\end{lstlisting}

\begin{itemize}[leftmargin=1.5em]
  \item 这是你在 Remix 部署到 Sepolia 后得到的 \textbf{合约地址}。
  \item \textbf{合约地址}是链上“程序”的位置(\emph{不是}你的钱包地址)。
\end{itemize}

\subsection{ABI(接口说明书)}

\begin{lstlisting}[language=JavaScript]
const MoodContractABI = [
  {
    inputs: [{ internalType: "string", name: "_mood", type: "string" }],
    name: "setMood",
    outputs: [],
    stateMutability: "nonpayable",
    type: "function",
  },
  {
    inputs: [],
    name: "getMood",
    outputs: [{ internalType: "string", name: "", type: "string" }],
    stateMutability: "view",
    type: "function",
  },
];
\end{lstlisting}

\begin{itemize}[leftmargin=1.5em]
  \item ABI = \textbf{前端与合约沟通的契约/说明书},描述函数签名与参数类型。
  \item ABI 不匹配会导致:读函数回 \texttt{0x}(无数据)或调用失败。
\end{itemize}

\section{7) 取得 DOM 元素:更新画面文字}

\begin{lstlisting}[language=JavaScript]
const statusEl = document.getElementById("status");
const showMoodEl = document.getElementById("showMood");

function setStatus(msg) {
  statusEl.textContent = msg || "";
}
\end{lstlisting}

\begin{itemize}[leftmargin=1.5em]
  \item \texttt{statusEl} / \texttt{showMoodEl}:用于把信息写回页面。
  \item \texttt{setStatus}:让更新状态文字变得更简洁。
\end{itemize}

\section{8) 检查 MetaMask 是否存在}

\begin{lstlisting}[language=JavaScript]
if (!window.ethereum) {
  setStatus("MetaMask not detected. Please install MetaMask.");
  throw new Error("MetaMask not detected");
}
\end{lstlisting}

\begin{itemize}[leftmargin=1.5em]
  \item \texttt{window.ethereum} 是 MetaMask 注入到网页的对象。
  \item 没有它就无法连接钱包、无法签名交易,因此直接 \texttt{throw} 停止执行更安全。
\end{itemize}

\section{9) 确保网络是 Sepolia(关键防踩坑)}

\begin{lstlisting}[language=JavaScript]
const chainId = await window.ethereum.request({ method: "eth_chainId" });
if (chainId !== "0xaa36a7") {
  setStatus("Please switch MetaMask network to Sepolia, then refresh.");
  try {
    await window.ethereum.request({
      method: "wallet_switchEthereumChain",
      params: [{ chainId: "0xaa36a7" }],
    });
    setStatus("Switched to Sepolia. Refreshing...");
    location.reload();
  } catch (e) {}
}
\end{lstlisting}

\begin{itemize}[leftmargin=1.5em]
  \item \texttt{eth\_chainId}:询问当前连接的链。
  \item Sepolia chainId(十六进制)为 \texttt{0xaa36a7}。
  \item 若用户不在 Sepolia:
    \begin{itemize}
      \item 提示用户切网
      \item 尝试自动触发 MetaMask 切网
      \item 成功后刷新页面,确保后续 client 走对网络
    \end{itemize}
  \item \textbf{为什么重要:}合约地址只在对应网络存在;如果连错链,读合约常见表现是返回 \texttt{0x} 或调用失败。
\end{itemize}

\section{10) 创建两个 client:read / write 分工}

\begin{lstlisting}[language=JavaScript]
const publicClient = createPublicClient({
  chain: sepolia,
  transport: custom(window.ethereum),
});

const walletClient = createWalletClient({
  chain: sepolia,
  transport: custom(window.ethereum),
});
\end{lstlisting}

\begin{itemize}[leftmargin=1.5em]
  \item \texttt{publicClient}:负责 \textbf{读取}(\texttt{eth\_call})与等待交易确认。
  \item \texttt{walletClient}:负责 \textbf{写入}(发交易、弹 MetaMask 签名)。
  \item 分工清晰、稳定,也更符合 viem 的常见用法。
\end{itemize}

\section{11) 请求用户钱包地址(连接网站)}

\begin{lstlisting}[language=JavaScript]
setStatus("Connecting wallet...");
const [address] = await walletClient.requestAddresses();
setStatus(`Connected: ${address.slice(0, 6)}...${address.slice(-4)}`);
\end{lstlisting}

\begin{itemize}[leftmargin=1.5em]
  \item \texttt{requestAddresses()} 会触发 MetaMask 弹窗,用户点 Connect 后返回地址列表。
  \item \texttt{address} 将用于写入交易的 \texttt{account} 参数。
  \item 使用 \texttt{slice} 只显示地址头尾,属于常见 UX。
\end{itemize}

\section{12) 创建合约实例:把 ABI + Address 变成可调用对象}

\begin{lstlisting}[language=JavaScript]
const MoodContract = getContract({
  address: MoodContractAddress,
  abi: MoodContractABI,
  client: { public: publicClient, wallet: walletClient },
});
\end{lstlisting}

\begin{itemize}[leftmargin=1.5em]
  \item 这一步把“合约地址 + ABI + client”组合起来,得到可直接调用的对象。
  \item 后续使用:
    \begin{itemize}
      \item \texttt{MoodContract.read.getMood()}
      \item \texttt{MoodContract.write.setMood(...)}
    \end{itemize}
\end{itemize}

\section{13) \texttt{window.getMood}:读链(不弹 MetaMask)}

\begin{lstlisting}[language=JavaScript]
window.getMood = async function () {
  try {
    setStatus("Reading mood from contract...");
    const mood = await MoodContract.read.getMood();
    showMoodEl.innerText = `Your Mood: ${mood}`;
    setStatus("Read successful.");
  } catch (err) {
    console.error(err);
    setStatus("Read failed. Check console for details.");
  }
};
\end{lstlisting}

\begin{itemize}[leftmargin=1.5em]
  \item \textbf{read} 调用本质是 \texttt{eth\_call}:不改变链上状态、不消耗 gas、不需要签名,所以不会弹钱包。
  \item 成功后把 mood 显示在 \texttt{\#showMood},状态显示在 \texttt{\#status}。
\end{itemize}

\section{14) \texttt{window.setMood}:写链(弹 MetaMask + 等确认)}

\begin{lstlisting}[language=JavaScript]
window.setMood = async function () {
  try {
    const mood = document.getElementById("mood").value.trim();
    if (!mood) {
      setStatus("Please enter a mood first.");
      return;
    }

    setStatus("Sending transaction (MetaMask will pop up)...");
    const hash = await MoodContract.write.setMood([mood], { account: address });

    setStatus(`Tx sent: ${hash.slice(0, 10)}... Waiting for confirmation...`);

    await publicClient.waitForTransactionReceipt({ hash });

    setStatus("Transaction confirmed! Reading updated mood...");
    const updated = await MoodContract.read.getMood();
    showMoodEl.innerText = `Your Mood: ${updated}`;
    setStatus("Done ✅");
  } catch (err) {
    console.error(err);
    setStatus("Write failed or rejected. Check console for details.");
  }
};
\end{lstlisting}

\subsection*{流程拆解(为什么这样写)}
\begin{enumerate}[leftmargin=1.8em]
  \item 读取输入框 mood,并进行非空验证(避免无意义交易、节省 gas)。
  \item 调用 \texttt{write.setMood}:
    \begin{itemize}
      \item 会弹 MetaMask
      \item 用户确认后广播交易
      \item 返回 \textbf{交易 hash}(此时还未确认)
    \end{itemize}
  \item 用 \texttt{waitForTransactionReceipt} 等待交易被打包确认(避免读到旧状态)。
  \item 交易确认后再次调用 \texttt{getMood()},并更新 UI 显示最新状态。
\end{enumerate}

\section{15) 可选:页面加载自动读取}

\begin{lstlisting}[language=JavaScript]
// window.getMood();
\end{lstlisting}

\begin{itemize}[leftmargin=1.5em]
  \item 取消注释后,页面一打开就会自动读取链上 mood 并显示。
\end{itemize}

\section{总结:这份 dApp 的本质}

一句话概括:

\begin{quote}
\textbf{用 MetaMask 提供身份与签名,用 viem 把网页变成合约的遥控器:}\\
\texttt{Get Mood} $\rightarrow$ 免费读取链上状态;\quad
\texttt{Set Mood} $\rightarrow$ 发送交易写入链上状态(需要 gas)。
\end{quote}

\section*{附:建议的下一步(可选提升)}
\begin{itemize}[leftmargin=1.5em]
  \item 把 ABI 独立成 \texttt{abi.json}(更像真实项目结构)
  \item 把 JS 独立成 \texttt{app.js}(可维护性更好)
  \item 增加 loading/disable 按钮(等待确认时禁用重复点击)
  \item 合约加入 \texttt{event MoodChanged(address user, string mood)} 并在前端监听事件,实现自动更新
\end{itemize}

\end{document}
