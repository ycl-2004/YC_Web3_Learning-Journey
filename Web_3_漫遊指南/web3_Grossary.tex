% !TEX program = xelatex
\documentclass[11pt,a4paper]{article}

% =========================
% style.tex  (ALL STYLES)
% =========================

% ---------- Page & Fonts ----------
\usepackage[a4paper,margin=1in]{geometry}
\usepackage{fontspec}
\usepackage{xeCJK}

% CJK font fallbacks (Overleaf 常見可用 Noto)
\IfFontExistsTF{Noto Serif CJK TC}{
  \setCJKmainfont{Noto Serif CJK TC}
}{
  \IfFontExistsTF{Noto Sans CJK TC}{
    \setCJKmainfont{Noto Sans CJK TC}
  }{
    \setCJKmainfont{SimSun}
  }
}
\setmainfont{TeX Gyre Termes}

% ---------- Packages ----------
\usepackage{hyperref}
\usepackage{microtype}
\usepackage{booktabs}
\usepackage{tabularx}
\usepackage{enumitem}
\usepackage{tikz}
\usetikzlibrary{positioning}
\usepackage{xcolor}
\usepackage{tcolorbox}
\tcbuselibrary{skins,breakable}
\usepackage{titlesec}
\usepackage{fancyhdr}

% ---------- Color Palette ----------
\definecolor{ink}{HTML}{1F2937}
\definecolor{muted}{HTML}{6B7280}
\definecolor{brand}{HTML}{2563EB}
\definecolor{brand2}{HTML}{0EA5E9}
\definecolor{soft}{HTML}{EFF6FF}
\definecolor{warn}{HTML}{F59E0B}
\definecolor{good}{HTML}{10B981}
\definecolor{rose}{HTML}{F43F5E}
\definecolor{paper}{HTML}{FFFFFF}

% ---------- Hyperref ----------
\hypersetup{
  colorlinks=true,
  linkcolor=brand,
  urlcolor=brand2,
  citecolor=brand
}

% ---------- Global typography ----------
\setlength{\parindent}{0pt}
\setlength{\parskip}{6pt}

% Clean lists inside boxes
\setlist[itemize]{itemsep=4pt, topsep=4pt, leftmargin=1.4em}
\setlist[enumerate]{itemsep=4pt, topsep=4pt, leftmargin=1.6em}

% ---------- Header ----------
\pagestyle{fancy}
\fancyhf{}
\lhead{\textcolor{muted}{Web3 / DApps 筆記}}
\rhead{\textcolor{muted}{\thepage}}
\renewcommand{\headrulewidth}{0pt}

% ---------- Section style ----------
\titleformat{\section}{\Large\bfseries\color{ink}}{\thesection}{0.8em}{}
\titleformat{\subsection}{\large\bfseries\color{ink}}{\thesubsection}{0.8em}{}
\titleformat{\subsubsection}{\normalsize\bfseries\color{ink}}{\thesubsubsection}{0.8em}{}
\titlespacing*{\section}{0pt}{1.0em}{0.4em}
\titlespacing*{\subsection}{0pt}{0.8em}{0.3em}
\titlespacing*{\subsubsection}{0pt}{0.5em}{0.2em}

% ---------- tcolorbox base ----------
\tcbset{
  enhanced,
  breakable,
  colback=paper,
  colframe=brand!18,
  boxrule=0.5pt,
  arc=10pt,
  left=10pt,right=10pt,top=10pt,bottom=10pt,
  before skip=10pt, after skip=10pt
}

% ---------- Box components ----------
\newtcolorbox{KeyBox}[1]{
  colback=paper,
  colframe=brand!18,
  borderline west={4pt}{0pt}{brand},
  title=\textbf{#1},
  coltitle=ink,
  fonttitle=\bfseries,
  colbacktitle=paper,
  boxed title style={boxrule=0pt, colframe=paper}
}

\newtcolorbox{DefBox}[1]{
  colback=paper,
  colframe=brand2!18,
  borderline west={4pt}{0pt}{brand2},
  title=\textbf{#1},
  coltitle=ink,
  fonttitle=\bfseries,
  colbacktitle=paper,
  boxed title style={boxrule=0pt, colframe=paper}
}

\newtcolorbox{HowBox}[1]{
  colback=paper,
  colframe=good!18,
  borderline west={4pt}{0pt}{good},
  title=\textbf{#1},
  coltitle=ink,
  fonttitle=\bfseries,
  colbacktitle=paper,
  boxed title style={boxrule=0pt, colframe=paper}
}

\newtcolorbox{WarnBox}[1]{
  colback=paper,
  colframe=warn!22,
  borderline west={4pt}{0pt}{warn},
  title=\textbf{#1},
  coltitle=ink,
  fonttitle=\bfseries,
  colbacktitle=paper,
  boxed title style={boxrule=0pt, colframe=paper}
}

% ---------- Tag pill ----------
\newcommand{\tagpill}[1]{%
  \tikz[baseline=(X.base)]\node[
    fill=soft, draw=brand!25, rounded corners=6pt,
    inner xsep=7pt, inner ysep=2.5pt
  ](X){\small\textcolor{brand}{\textbf{#1}}};%
}

\renewcommand{\arraystretch}{1.2}

\begin{document}

% ---------- Title ----------
\begin{center}
  {\LARGE \textbf{Web 3.0 社区常用缩写与术语速查表}}\\[6pt]
  {\large \textcolor{muted}{考試 / 社群 / Discord / Twitter 一次看懂}}\\[10pt]
  \tagpill{Web3}\quad
  \tagpill{NFT}\quad
  \tagpill{DeFi}\quad
  \tagpill{DAO}\quad
  \tagpill{Slang}
\end{center}

\begin{KeyBox}{使用說明(怎麼背)}
\begin{itemize}
  \item 本表只收錄 \textbf{高頻、實際會看到/用到} 的縮写與黑話
  \item 解釋以「\textbf{一句話就懂}」為原則
  \item 考試寫概念題、看白皮書、混社群都夠用
\end{itemize}
\end{KeyBox}

\tableofcontents

% =========================================================
\section{基础操作 / 技术相关}

\begin{KeyBox}{基础术语}
\begin{itemize}
  \item \textbf{Mint}:鑄造 NFT,將其寫入區塊鏈的過程
  \item \textbf{Airdrop}:項目方免費向用戶錢包發放通證(行銷/激勵)
  \item \textbf{Snapshot}:錢包快照,用來判定是否有資格參與活動
  \item \textbf{Gas}:區塊鏈交易手續費(以太坊上通常以 ETH 支付)
  \item \textbf{RPC}:連接區塊鏈節點的接口(錢包與鏈溝通用)
\end{itemize}
\end{KeyBox}

\begin{WarnBox}{新手常踩坑}
Mint ≠ 買到著作權;Airdrop ≠ 一定有價值;Gas 高時亂 Mint 很容易虧。
\end{WarnBox}

% =========================================================
\section{钱包 / 身份相关}

\begin{KeyBox}{钱包与身份}
\begin{itemize}
  \item \textbf{MM}:MetaMask,小狐狸錢包(以太坊生態主流錢包)
  \item \textbf{EOA}:Externally Owned Account,一般使用者錢包
  \item \textbf{ENS}:以太坊域名系統(如 \texttt{vitalik.eth})
  \item \textbf{PFP}:Profile Picture,頭像類 NFT
  \item \textbf{OG}:Original Gangster,早期參與者/元老
\end{itemize}
\end{KeyBox}

% =========================================================
\section{NFT / 社群文化}

\begin{KeyBox}{NFT 圈黑话}
\begin{itemize}
  \item \textbf{WL(White List)}:白名單,提前/折扣 Mint 資格
  \item \textbf{Roadmap}:項目未來計畫藍圖
  \item \textbf{Rug / Rug Pull}:項目跑路、不再營運
  \item \textbf{10k Project}:總量約 10,000 的 NFT 系列(如 CryptoPunks)
  \item \textbf{AMA}:Ask Me Anything,項目方社群問答
\end{itemize}
\end{KeyBox}

\begin{HowBox}{社群互動用語}
\begin{itemize}
  \item \textbf{GM / GN}:Good Morning / Good Night,社群打招呼
  \item \textbf{Ser}:Sir 的反諷寫法,帶點玩笑/不太認真
  \item \textbf{Fren}:Friend 的變形寫法,社群友好語氣
  \item \textbf{Meme}:梗文化,用於描述社群共識與傳播
\end{itemize}
\end{HowBox}

% =========================================================
\section{市场 / 交易相关}

\begin{KeyBox}{价格与市场}
\begin{itemize}
  \item \textbf{Floor Price}:地板價,當前最低掛單價
  \item \textbf{Floor Sweeping}:掃地板,買光低價 NFT
  \item \textbf{ATH}:All Time High,歷史最高價
  \item \textbf{Moon}:價格暴漲(Going to the moon)
  \item \textbf{Whale}:巨鯨,大戶
\end{itemize}
\end{KeyBox}

\begin{WarnBox}{情绪相关詞}
\begin{itemize}
  \item \textbf{FOMO}:Fear of Missing Out,怕錯過而追高
  \item \textbf{FUD}:Fear, Uncertainty, Doubt,恐慌與不確定
  \item \textbf{Diamond Hands}:鑽石手,怎麼跌都不賣
  \item \textbf{Paper Hands}:紙手,容易恐慌賣出
\end{itemize}
\end{WarnBox}

% =========================================================
\section{投资 / DeFi / DAO}

\begin{KeyBox}{投资与治理}
\begin{itemize}
  \item \textbf{DYOR}:Do Your Own Research,自己做研究
  \item \textbf{HODL}:Hold 的誤拼,長期持有
  \item \textbf{BUIDL}:Build 的誤拼,強調建設而非炒作
  \item \textbf{DAO}:去中心化自治组织
  \item \textbf{Treasury}:金庫,DAO 的資金池
\end{itemize}
\end{KeyBox}

\begin{KeyBox}{DeFi 高频指标}
\begin{itemize}
  \item \textbf{TVL}:Total Value Locked,總鎖倉價值
  \item \textbf{APY}:年化收益率
  \item \textbf{LP}:Liquidity Provider,流動性提供者
  \item \textbf{Slippage}:滑點,成交價偏差
\end{itemize}
\end{KeyBox}

% =========================================================
\section{态度 / 社群信念}

\begin{KeyBox}{信念型用语}
\begin{itemize}
  \item \textbf{WAGMI}:We All Gonna Make It,我們都會成功
  \item \textbf{NGMI}:Not Gonna Make It,不會成功
  \item \textbf{GOAT}:Greatest Of All Time,最頂級
  \item \textbf{McDonald's}:麥當勞,指投資失敗後去打工(自嘲)
\end{itemize}
\end{KeyBox}

% =========================================================
\section{最后提醒}

\begin{WarnBox}{一句话总结}
Web 3.0 的缩写與黑话,本质是\textbf{社群文化 + 投机情绪 + 技术背景}的混合体,\\
\textbf{看得懂不代表该跟风,理解比参与更重要。}
\end{WarnBox}

\end{document}
