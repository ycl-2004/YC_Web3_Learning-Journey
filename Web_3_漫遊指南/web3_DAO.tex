% !TEX program = xelatex
\documentclass[11pt,a4paper]{article}

% =========================
% style.tex  (ALL STYLES)
% =========================

% ---------- Page & Fonts ----------
\usepackage[a4paper,margin=1in]{geometry}
\usepackage{fontspec}
\usepackage{xeCJK}

% CJK font fallbacks (Overleaf 常見可用 Noto)
\IfFontExistsTF{Noto Serif CJK TC}{
  \setCJKmainfont{Noto Serif CJK TC}
}{
  \IfFontExistsTF{Noto Sans CJK TC}{
    \setCJKmainfont{Noto Sans CJK TC}
  }{
    \setCJKmainfont{SimSun}
  }
}
\setmainfont{TeX Gyre Termes}

% ---------- Packages ----------
\usepackage{hyperref}
\usepackage{microtype}
\usepackage{booktabs}
\usepackage{tabularx}
\usepackage{enumitem}
\usepackage{tikz}
\usetikzlibrary{positioning}
\usepackage{xcolor}
\usepackage{tcolorbox}
\tcbuselibrary{skins,breakable}
\usepackage{titlesec}
\usepackage{fancyhdr}

% ---------- Color Palette ----------
\definecolor{ink}{HTML}{1F2937}
\definecolor{muted}{HTML}{6B7280}
\definecolor{brand}{HTML}{2563EB}
\definecolor{brand2}{HTML}{0EA5E9}
\definecolor{soft}{HTML}{EFF6FF}
\definecolor{warn}{HTML}{F59E0B}
\definecolor{good}{HTML}{10B981}
\definecolor{rose}{HTML}{F43F5E}
\definecolor{paper}{HTML}{FFFFFF}

% ---------- Hyperref ----------
\hypersetup{
  colorlinks=true,
  linkcolor=brand,
  urlcolor=brand2,
  citecolor=brand
}

% ---------- Global typography ----------
\setlength{\parindent}{0pt}
\setlength{\parskip}{6pt}

% Clean lists inside boxes
\setlist[itemize]{itemsep=4pt, topsep=4pt, leftmargin=1.4em}
\setlist[enumerate]{itemsep=4pt, topsep=4pt, leftmargin=1.6em}

% ---------- Header ----------
\pagestyle{fancy}
\fancyhf{}
\lhead{\textcolor{muted}{Web3 / DApps 筆記}}
\rhead{\textcolor{muted}{\thepage}}
\renewcommand{\headrulewidth}{0pt}

% ---------- Section style ----------
\titleformat{\section}{\Large\bfseries\color{ink}}{\thesection}{0.8em}{}
\titleformat{\subsection}{\large\bfseries\color{ink}}{\thesubsection}{0.8em}{}
\titleformat{\subsubsection}{\normalsize\bfseries\color{ink}}{\thesubsubsection}{0.8em}{}
\titlespacing*{\section}{0pt}{1.0em}{0.4em}
\titlespacing*{\subsection}{0pt}{0.8em}{0.3em}
\titlespacing*{\subsubsection}{0pt}{0.5em}{0.2em}

% ---------- tcolorbox base ----------
\tcbset{
  enhanced,
  breakable,
  colback=paper,
  colframe=brand!18,
  boxrule=0.5pt,
  arc=10pt,
  left=10pt,right=10pt,top=10pt,bottom=10pt,
  before skip=10pt, after skip=10pt
}

% ---------- Box components ----------
\newtcolorbox{KeyBox}[1]{
  colback=paper,
  colframe=brand!18,
  borderline west={4pt}{0pt}{brand},
  title=\textbf{#1},
  coltitle=ink,
  fonttitle=\bfseries,
  colbacktitle=paper,
  boxed title style={boxrule=0pt, colframe=paper}
}

\newtcolorbox{DefBox}[1]{
  colback=paper,
  colframe=brand2!18,
  borderline west={4pt}{0pt}{brand2},
  title=\textbf{#1},
  coltitle=ink,
  fonttitle=\bfseries,
  colbacktitle=paper,
  boxed title style={boxrule=0pt, colframe=paper}
}

\newtcolorbox{HowBox}[1]{
  colback=paper,
  colframe=good!18,
  borderline west={4pt}{0pt}{good},
  title=\textbf{#1},
  coltitle=ink,
  fonttitle=\bfseries,
  colbacktitle=paper,
  boxed title style={boxrule=0pt, colframe=paper}
}

\newtcolorbox{WarnBox}[1]{
  colback=paper,
  colframe=warn!22,
  borderline west={4pt}{0pt}{warn},
  title=\textbf{#1},
  coltitle=ink,
  fonttitle=\bfseries,
  colbacktitle=paper,
  boxed title style={boxrule=0pt, colframe=paper}
}

% ---------- Tag pill ----------
\newcommand{\tagpill}[1]{%
  \tikz[baseline=(X.base)]\node[
    fill=soft, draw=brand!25, rounded corners=6pt,
    inner xsep=7pt, inner ysep=2.5pt
  ](X){\small\textcolor{brand}{\textbf{#1}}};%
}

\renewcommand{\arraystretch}{1.18}

% =========================
% DAO Notes (NEW FILE)
% =========================
\begin{document}

% ---------- Title ----------
\begin{center}
  {\LARGE \textbf{DAO(去中心化自治組織)重點筆記}}\\[6pt]
  {\large \textcolor{muted}{只保留:是什麼|用途|用在哪|怎麼 work|例子}}\\[10pt]
  \tagpill{DAO}\quad \tagpill{Governance}\quad \tagpill{Smart Contract}\quad \tagpill{Voting}\quad \tagpill{Token}\quad
  \tagpill{Org Design}
\end{center}

\begin{KeyBox}{快速總結(考前 30 秒)}
\begin{itemize}
  \item \textbf{一句話}:DAO 是 Web3 的「社會關係/組織作業系統」;用\textbf{智能合約 + 社群協作}共同管理資產、達成共同目標。
  \item \textbf{本質}:把公司裡的「規則、資金、決策」盡可能\textbf{寫進合約},再用\textbf{投票/委託/聲譽}等機制協調人。
  \item \textbf{為何需要}:跨國跨時區、匿名也能協作;規則更透明、資金流向可追溯;參與者也可能是所有者。
  \item \textbf{現實版本}:很多事情仍無法 100\% 合約自動化,會演化成 \textbf{DO(去中心化組織)}:關鍵決策仍需人工協作 + 投票。
\end{itemize}
\end{KeyBox}

\tableofcontents

% =========================================================
\section{DAO 是什麼?(定位與比喻)}

\begin{DefBox}{一句話定義}
\textbf{DAO(Decentralized Autonomous Organization)}是一種由\textbf{成員集體所有與共同治理}的組織方式:成員共同管理加密資產,為了\textbf{共同目標}運作;其規則與資金管理多由\textbf{智能合約}界定與執行。
\end{DefBox}

\begin{KeyBox}{用在哪(你要能回答「在哪裡出現」)}
\begin{itemize}
  \item \textbf{Web3 社群/專案治理}:資金庫(Treasury)怎麼花、資助哪些專案、誰可以做決策。
  \item \textbf{公共資助/公益捐助}:用投票決定資金分配(例:捐助型 DAO)。
  \item \textbf{產品/協議發展}:協議升級、參數調整、開發者補助、社群提案流程。
\end{itemize}
\end{KeyBox}

\begin{HowBox}{直覺比喻:DApps vs DAO}
如果說 \textbf{DApps} 是 Web3 世界「看得見的應用/磚瓦」,那 \textbf{DAO} 更像是 Web3 的「\textbf{社會作業系統}」:定義人怎麼協作、怎麼決策、錢怎麼分配。
\end{HowBox}

\begin{DefBox}{DAO vs DO(考點:為什麼會變成 DO)}
\begin{itemize}
  \item \textbf{DAO(理想型)}:規則與執行盡量由合約自動完成。
  \item \textbf{DO(現實型)}:有些目標/工作無法完全自動化,需要\textbf{人工協作與介入}(但仍保持去中心化決策)。
\end{itemize}
\end{DefBox}

% =========================================================
\section{DAO 的「至少要有」的組成(框架)}

\begin{KeyBox}{DAO 最少包含哪些部分(記成 5 個模組)}
\begin{enumerate}
  \item \textbf{目標與章程(Mission/Charter)}:讓所有成員對「要達成什麼」有共同理解。
  \item \textbf{參與方式與激勵(Participation \& Incentives)}:用勞動/資金/資源換取報酬與權益(最難設計)。
  \item \textbf{爭議處理(Dispute/Consensus)}:靠溝通 + 投票機制把分歧收斂成共識。
  \item \textbf{協作流程(Workflow/Tools)}:日常討論、文件、任務協作工具(常見:Discord)。
  \item \textbf{發展路徑(Growth Path)}:組織如何從小到大、從創始團隊引導到社群自治。
\end{enumerate}
\end{KeyBox}

\begin{HowBox}{兩種發展路徑(很常考:Exit to DAO vs DAO First)}
\begin{itemize}
  \item \textbf{Exit to DAO(歸於 DAO)}:早期先用較中心化方式啟動,之後逐步把權力下放到社群(現實中更常見)。
  \item \textbf{DAO First(始於 DAO)}:從一開始就確定通證分配、資本構成與治理規則,以去中心化方式直接運作。
\end{itemize}
\end{HowBox}

% =========================================================
\section{公司 vs DAO:差在哪?(超高頻對照)}

\subsection{一句話抓差異}
\begin{DefBox}{三個根本差異(Rule / Value / Identity)}
\begin{itemize}
  \item \textbf{規則(Rule)}:公司靠法律合同與內部制度;DAO 主要靠\textbf{鏈上合約/程式規則}約束。
  \item \textbf{利益分配(Value)}:公司分配權偏中心化;DAO 參與者往往同時是\textbf{通證持有人},理論上更利害一致。
  \item \textbf{身份邊界(Identity)}:公司多為固定雇傭與層級;DAO 允許更自由的\textbf{進出與跨 DAO 協作}。
\end{itemize}
\end{DefBox}

\subsection{DAO vs 傳統企業(表格整理)}

\begin{KeyBox}{對照表(背這張就夠)}
\begin{tabularx}{\textwidth}{@{}>{\bfseries}lX X@{}}
\toprule
 & \textbf{DAO} & \textbf{傳統企業} \\
\midrule
權力結構 &
通常更偏「\textbf{平等/分散}」,依治理設計可委託代表 &
通常\textbf{等級分明}(股東/董事會/管理層/員工) \\
\addlinespace[3pt]
規則更改 &
多數需要\textbf{提案 + 成員投票}通過 &
可由\textbf{少數人決策}或依公司章程流程(看組織結構) \\
\addlinespace[3pt]
投票與執行 &
可用合約\textbf{自動計票 + 上鏈執行},不必信任中介 &
就算投票也多為\textbf{內部計票 + 人工執行} \\
\addlinespace[3pt]
服務/資金分配 &
可用去中心化方式\textbf{自動提供服務/分配資金}(例:資助/基金) &
常需\textbf{人工處理或集中控制},更容易被操縱 \\
\addlinespace[3pt]
透明度 &
活動與資金流向通常\textbf{更公開可驗證} &
活動多為\textbf{內部私密},不一定對外公開 \\
\bottomrule
\end{tabularx}
\end{KeyBox}

\begin{WarnBox}{提醒:DAO 不是「一定更好」}
\begin{itemize}
  \item 分散治理會引入\textbf{協調成本}(大家都投票=很慢)。
  \item 透明可驗證也代表\textbf{可追蹤性}更強;隱私通常需要額外機制。
\end{itemize}
\end{WarnBox}

% =========================================================
\section{DAO 怎麼 work?(組織架構 + 專案制)}

\subsection{組織架構(核心團隊 + 公會 + 專案組)}

\begin{HowBox}{典型架構(你要能講出「為什麼不是天天投票」)}
\begin{itemize}
  \item 多數 DAO 會有\textbf{核心創始團隊/決策委員會}處理日常事務,避免無休止投票。
  \item 重大方向/資金分配仍傾向由\textbf{成員投票}決定(核心團隊不是完全壟斷)。
  \item 依需求設立\textbf{公會(Guild)}:像人才池,聚集特定技能的人(開發/內容/社群/財務等)。
  \item DAO 可同時跑多個\textbf{專案(Project)}:專案成員可從不同公會橫向組隊,形成\textbf{縱橫結構}。
\end{itemize}
\end{HowBox}

\begin{KeyBox}{Holacracy(對應概念)}
DAO 常被描述為 Web3 原生的 \textbf{Holacracy(整體式組織)}:每個人既是獨立個體,也是一個更大整體的一部分;不是嚴格上下級層級。
\end{KeyBox}

\subsection{專案制運作(從提案到分配)}

\begin{HowBox}{專案流程(Proposal \texorpdfstring{$\rightarrow$}{->} Vote \texorpdfstring{$\rightarrow$}{->} Funding \texorpdfstring{$\rightarrow$}{->} Tasks \texorpdfstring{$\rightarrow$}{->} Rewards)}
\begin{enumerate}
  \item \textbf{發起專案}:任何成員提出能促進目標的計畫,在社群宣導爭取支持。
  \item \textbf{投票決策}:社群投票決定是否提供資金(Treasury)。
  \item \textbf{招募組隊}:若通過,從各公會招募成員組成專案組。
  \item \textbf{任務拆解}:專案拆成任務(短期賞金/長期協作任務)。
  \item \textbf{回報分配}:依貢獻度/積分/規則分配賞金、分紅或通證激勵。
\end{enumerate}
\end{HowBox}

\begin{DefBox}{貢獻值 / 聲譽 / 勳章(Web3 履歷感)}
\begin{itemize}
  \item 完成任務可獲\textbf{貢獻值}(不同 DAO 叫聲譽/經驗值等):多為\textbf{不可流通},但會影響未來收益與權限(白名單、PASS、POAP 等)。
  \item 勳章/徽章:通常\textbf{無財務價值、不可轉讓},但有社交與能力證明價值,可成為你的 Web3 履歷。
\end{itemize}
\end{DefBox}

\begin{KeyBox}{兩類任務(要會分)}
\begin{itemize}
  \item \textbf{賞金任務(Bounty)}:週期短、交付明確;完成拿賞金 + 一些貢獻值。
  \item \textbf{專案任務(Project Work)}:協作度高、門檻較高;回報周期長,常以\textbf{分紅/長期收益}呈現。
\end{itemize}
\end{KeyBox}

% =========================================================
\section{治理機制:DAO 常見的 7 種投票(只背核心直覺)}

\subsection{1) 流動民主(Liquid Democracy / Vote Delegation)}
\begin{DefBox}{是什麼}
成員可以把投票權\textbf{委託給專家},也能\textbf{隨時撤回}自己投;委託可多級轉委託。
\end{DefBox}

\begin{KeyBox}{用途 / 用在哪}
\begin{itemize}
  \item 解決「所有人都投每一案」的注意力問題:讓專家代投,但保留退出權。
\end{itemize}
\end{KeyBox}

\subsection{2) 二次方投票(Quadratic Voting)}
\begin{DefBox}{是什麼}
允許同一選項投多票,但\textbf{邊際成本遞增}:投 $k$ 票成本 $\propto k^2$(直覺:越偏好越貴)。
\end{DefBox}

\begin{HowBox}{例子(背直覺)}
想強烈支持某提案可以多投,但會變得非常貴;常用於\textbf{公共資源分配}(典型:捐助/資助型 DAO)。
\end{HowBox}

\begin{WarnBox}{限制}
需要嚴格的\textbf{身份防女巫(Sybil)}機制,否則可用多身份繞過成本。
\end{WarnBox}

\subsection{3) 全息共識(Holographic Consensus)}
\begin{DefBox}{是什麼(解決注意力稀缺)}
用「押注/預測」幫忙篩選提案:押注者用注意力通證對提案成功概率下注,讓小群體能\textbf{代表性地聚焦重要提案}。
\end{DefBox}

\begin{HowBox}{流程(4 步)}
\begin{enumerate}
  \item \textbf{發起提案}
  \item \textbf{提案增強}:持有人對看好提案押注,沒押注到門檻的提案被忽略
  \item \textbf{投票決策}:有投票權者表決;押對得獎勵,押錯受損失
  \item \textbf{上鏈執行}:通過提案生效並執行
\end{enumerate}
\end{HowBox}

\begin{WarnBox}{常見質疑(背兩句就好)}
\begin{itemize}
  \item 會不會只篩出「有話題/傳播」的提案,而非最重要的提案?
  \item 押注者關心「會不會過」而不是「該不該過」,是否可能扭曲結果?
\end{itemize}
\end{WarnBox}

\subsection{4) 信念投票(Conviction Voting)}
\begin{DefBox}{是什麼}
投票效用不只看票數,還看\textbf{時間累積}:支持越久影響越大;撤回支持影響會逐步衰減。
\end{DefBox}

\begin{KeyBox}{用在哪}
\begin{itemize}
  \item 特別適合\textbf{預算決策}:不要求大家在同一時點做最終決定,降低投票門檻。
\end{itemize}
\end{KeyBox}

\subsection{5) 怒退機制(Rage Quitting)}
\begin{DefBox}{是什麼(保護少數者)}
如果對治理結果不滿,成員可\textbf{退出並取回對應份額資金};常在投票通過後設\textbf{緩衝期(Grace Period)}供退出。
\end{DefBox}

\begin{HowBox}{為什麼重要}
防止大戶用投票侵吞小戶利益:你不滿意,就能帶走你那份。
\end{HowBox}

\subsection{6) 知識提取投票(KEV: Knowledge-extractable Voting)}
\begin{DefBox}{是什麼}
引入不可交易的\textbf{知識通證},讓懂該議題的專家在該類提案中\textbf{權重更高};投對(與最終結果一致)獎勵知識通證,投錯扣減。
\end{DefBox}

\begin{WarnBox}{常見問題}
\begin{itemize}
  \item 「專家是否正確」仍依賴最終投票結果來回饋,可能出現與全息共識類似的質疑。
\end{itemize}
\end{WarnBox}

\subsection{7) 聲譽投票(Reputation-based Voting)}
\begin{DefBox}{是什麼(治理權與財務價值解耦)}
把治理權從可交易通證分離出來,改用不可轉讓的\textbf{聲譽/積分}投票;聲譽可因貢獻獲得,也可被扣減或隨時間衰減。
\end{DefBox}

\begin{KeyBox}{解決什麼問題}
\begin{itemize}
  \item 降低「借幣/租票」發動治理攻擊的可行性(通證金融化帶來的治理風險)。
  \item 讓 DAO 能依自身理解定義「民主」與權重(自訂聲譽規則)。
\end{itemize}
\end{KeyBox}

% =========================
% Paste below your previous content
% Topic: DID -> SBT + DAO cases
% =========================

\section{从 DID 到 SBT:Web 3.0 的身份证(Identity)}

\begin{DefBox}{一句话定义:DID 是什么?}
\textbf{去中心化识别码(Decentralized Identifier, DID)}可以理解为 Web3 的「身份证」:它是\textbf{全局唯一、持久、且不依赖中心化登记机构}的标识符,通常通过加密算法生成或登记。
\end{DefBox}

\begin{KeyBox}{为什么 Web3 需要「身份证」?(核心直觉)}
\begin{itemize}
  \item 现实世界里,你的一张身份证背后连接着\textbf{大量社会关系与记录}(教育、工作、医保、出行等),它让你能被识别、被准入、被服务。
  \item Web3 里也是一样:\textbf{你的身份本质是你在链上留下的行为与社会关系的总和},DID 就是把这些痕迹组织成「可验证身份」的入口。
\end{itemize}
\end{KeyBox}

\begin{HowBox}{DID 怎么 work(链上可验证)}
\begin{itemize}
  \item DID \textbf{基于区块链}:链上数据\textbf{公开透明、可查验、难以篡改}。
  \item 当你在 Web3 亮出 DID:别人\textbf{不必依赖权威机构},就能验证它的真伪与准入资格(靠链上可验证性)。
\end{itemize}
\end{HowBox}

\subsection{POAP:用 NFT 做「出席证明」}

\begin{DefBox}{是什麼}
\textbf{出席证明协议(Proof of Attendance Protocol, POAP)}是在以太坊上专门用于发放「出席徽章」NFT 的协议。组织者可向 DAO 成员或活动参与者发放徽章。
\end{DefBox}

\begin{KeyBox}{用途/用在哪}
\begin{itemize}
  \item \textbf{用途}:证明「你参加过某活动/属于某社群/完成某经历」。
  \item \textbf{用在哪}:DAO 活动、线下会议、黑客松、社群任务完成证明;也可能成为其他场合的\textbf{通行证}或「炫耀资本」。
  \item 这些徽章 NFT 存在你的\textbf{钱包}中,\textbf{公开可查},像钱包里的其他链上资产与痕迹一样可被检索。
\end{itemize}
\end{KeyBox}

\begin{WarnBox}{痛点:NFT 可转手 \texorpdfstring{$\Rightarrow$}{=>} 身份会被买卖}
\begin{itemize}
  \item 目前很多 Web3 身份用「钱包里的 NFT」来证明经历,但 NFT 可以\textbf{转卖}。
  \item 结果:可能出现\textbf{多重身份}与「黄牛黑市」——不利于形成健康可持续的 Web3 身份体系。
\end{itemize}
\end{WarnBox}

\subsection{SBT:不可转让的身份凭证}

\begin{DefBox}{一句话定义}
\textbf{灵魂绑定通证(Soul Bound Token, SBT)}可以理解为「\textbf{不可转让的 NFT}」:一经授予,\textbf{无法转手},用来更可靠地绑定钱包与真实经历。
\end{DefBox}

\begin{KeyBox}{解决什么问题?}
\begin{itemize}
  \item 把「经历/资格」从可交易资产中剥离出来:\textbf{经历只能由真正完成的人拥有}。
  \item 让我们能更稳定地识别:\textbf{某个钱包到底做过什么、参与过什么、贡献过什么}。
\end{itemize}
\end{KeyBox}

\begin{HowBox}{例子:RabbitHole(任务化学习 + SBT 成就徽章)}
\begin{itemize}
  \item RabbitHole 把 DApps 拆成\textbf{游戏化任务},用户通过使用 DApps 获得 \textbf{XP/经验}、升级并拿奖励。
  \item 它发行「冒险勋章」用来展示用户的成就与知识,并将其设计为\textbf{不可转让的 SBT}。
  \item 结果:任何带有该徽章的钱包,\textbf{更可信地代表}钱包所有者确实完成了对应的任务与成就。
\end{itemize}
\end{HowBox}

\begin{WarnBox}{一句话展望}
目前 SBT 案例还不算多,但一旦大规模普及,将显著改变 Web3 的\textbf{身份、准入、声誉与协作模式}。
\end{WarnBox}

% =========================================================
\section{DAO 在各领域的典型案例(投资 / 慈善 / 艺术 / 协议治理)}

\begin{KeyBox}{本节你要记住的框架}
DAO 不只是一种「组织概念」,它会以不同目标呈现为不同类型:\textbf{投资型、慈善资助型、收藏/艺术型、协议治理型}等。
\end{KeyBox}

% ---------------- Investment DAO ----------------
\subsection{投资:Cult.DAO(去中心化风投)}

\begin{DefBox}{是什麼}
\textbf{Cult.DAO} 是一个\textbf{去中心化风投(Venture DAO)}:由成员共同决定投资哪些 Web3 项目,资金由协议机制持续注入其金库。
\end{DefBox}

\begin{HowBox}{怎麼 work(角色 + 权力结构)}
\begin{itemize}
  \item 两个群体:\textbf{守卫(Guardians)}与\textbf{选民(Voters)}。
  \item 成员\textbf{质押}原生通证 \textbf{CULT} 可获得治理通证 \textbf{dCULT}(对应治理权)。
  \item \textbf{dCULT 持有量前 50 名}成为守卫:拥有\textbf{发起提案}的权利;其他人为普通选民参与投票。
  \item 守卫提案内容需与\textbf{投资其他 Web3 项目}有关;最终仍由\textbf{成员投票}决定是否通过。
\end{itemize}
\end{HowBox}

\begin{KeyBox}{资金来源(为什么能运作)}
\begin{itemize}
  \item 投资资金来自 Cult.DAO 的\textbf{小金库(Treasury)}:从每笔 \textbf{CULT 交易}中抽取 \textbf{0.4\%} 佣金作为持续资金来源。
\end{itemize}
\end{KeyBox}

\begin{HowBox}{用传统企业类比(秒懂版)}
\begin{itemize}
  \item Cult.DAO $\approx$ Web3 的\textbf{去中心化风投基金}
  \item \textbf{CULT 质押者} $\approx$ 基金投资人(LP)
  \item \textbf{守卫} $\approx$ 项目经理/投委会(筛项目、提案)
  \item \textbf{所有成员投票} $\approx$ 基金治理与决策机制(更公开/更程序化)
\end{itemize}
\end{HowBox}

% ---------------- Charity DAO ----------------
\subsection{慈善:Gitcoin(捐助型资助平台 / DAO)}

\begin{DefBox}{是什麼}
\textbf{Gitcoin} 是构建在以太坊上的去中心化协作与资助平台:既支持开发者通过赏金协作,也通过多轮资助把资金部署到 Web3 基础设施项目中。
\end{DefBox}

\begin{KeyBox}{用途/用在哪}
\begin{itemize}
  \item \textbf{用在哪}:开源工具、协议、技术网络等 Web3 新基础设施项目。
  \item \textbf{用途}:让社区以更开放的方式表达「应该资助什么」,并帮助项目获得\textbf{曝光与注意力}。
\end{itemize}
\end{KeyBox}

\begin{HowBox}{关键价值(文本主旨)}
\begin{itemize}
  \item Gitcoin 已进行十多轮资助,覆盖数千项目。
  \item 对项目而言,Gitcoin 带来的\textbf{关注度}往往比资助金额本身更有价值:它像行业风向标,指引从业者关注方向。
\end{itemize}
\end{HowBox}

% ---------------- Art / Collector DAO ----------------
\subsection{数字艺术:PleasrDAO(收藏 + 社群共治)}

\begin{DefBox}{是什麼}
\textbf{PleasrDAO} 是由收藏家与数字艺术家组成的 DAO:共同收购并资助具有文化意义的作品(多为 NFT),并与社区共享与再创造。
\end{DefBox}

\begin{HowBox}{起源故事(你要会讲的一段)}
\begin{itemize}
  \item 2021/3/26:加密艺术家 pplpleasr 表示将把 Uniswap V3 官宣视频片段做成 NFT 出售,收益用于公益。
  \item 社区成员号召成立轻量 DAO 竞拍,\textbf{一天筹集超过 60 万美元}。
  \item 最终以 \textbf{310 ETH} 竞得;竞拍后决定持续运作,并以 \textbf{PleasrDAO} 命名纪念首次慈善竞拍。
  \item 随后发行治理通证 \textbf{PEEPS},按贡献比例分配;持有人共同享有 DAO 旗下 NFT 所有权。
\end{itemize}
\end{HowBox}

\begin{KeyBox}{作品例子(记 2--3 个即可)}
\begin{itemize}
  \item \textbf{Stay Free}:Edward Snowden 的 NFT,支持其非营利基金会 Freedom of the Press。
  \item \textbf{Dreaming at Dusk}:围绕 Tor Project 早期洋葱服务的历史与艺术呈现,支持在线隐私与匿名相关非营利努力。
  \item \textbf{Doge}:以狗狗币 meme 为原型的标志性 NFT,记录互联网历史片段。
\end{itemize}
\end{KeyBox}

% ---------------- Protocol governance DAO ----------------
% =========================
% Paste below your previous content
% Updates: Uniswap (more detailed) + add FWB DAO + CityDAO + How to build DAO (Aragon)
% =========================

% ---------------- Protocol governance DAO (UPDATED) ----------------
\subsection{协议治理:Uniswap(DEX 协议 + DAO 治理)}

\begin{DefBox}{是什麼}
\textbf{Uniswap} 是一个以\textbf{协议}形式存在的去中心化交易所(DEX),其治理以 \textbf{DAO} 形式运作:社区通过治理通证与投票流程,决定协议升级与新功能引入。
\end{DefBox}

\begin{KeyBox}{用途/用在哪}
\begin{itemize}
  \item \textbf{用在哪}:协议参数调整、功能升级、资源分配、生态发展等治理事项。
  \item \textbf{用途}:把「原本由开发团队全权决定」的开发决策,转移到\textbf{社区治理}(投票/委托投票)上。
\end{itemize}
\end{KeyBox}

\begin{HowBox}{治理转移:UNI 的意义(时间点 + 核心变化)}
\begin{itemize}
  \item 2020 年 9 月:推出治理通证 \textbf{UNI},开启新的治理结构。
  \item \textbf{UNI 引入前}:开发团队全权负责确定 Uniswap 的开发决策。
  \item \textbf{UNI 引入后}:治理权正式转移到社区;任何 UNI 持有者可\textbf{投票}或\textbf{委托他人投票}影响协议发展。
\end{itemize}
\end{HowBox}

\begin{KeyBox}{UNI 总量与分配(背比例就够)}
团队分发了 \textbf{10 亿枚 UNI} 给开发团队、社区、投资者与顾问(部分有归属期):
\begin{itemize}
  \item \textbf{60.00\%}:Uniswap 社区成员
  \item \textbf{21.266\%}:团队成员与未来员工(\textbf{4 年归属})
  \item \textbf{18.044\%}:投资者(\textbf{4 年归属})
  \item \textbf{0.69\%}:顾问(\textbf{4 年归属})
\end{itemize}
\end{KeyBox}

\begin{HowBox}{提案怎么通过?(三阶段投票流程)}
任何 UNI 持有者都可以提出「更改或引入新功能」的提案,但在实施前要过三关:
\begin{enumerate}
  \item \textbf{热度投票(Temperature Check)}\\
  提议者先在社区宣讲想法;提案需获得 \textbf{25,000 UNI} 赞成票,才进入下一阶段。
  \item \textbf{共识投票(Consensus Check)}\\
  提议者进行更正式的陈述/答辩,突出提案优势;需获得不少于 \textbf{50,000 UNI} 赞成票。
  \item \textbf{治理提案(Governance Proposal)}\\
  提交\textbf{经审计的代码}进行最终审议;提案需获得高达 \textbf{40,000,000 UNI} 赞成票才可实施。
\end{enumerate}
\end{HowBox}

\begin{WarnBox}{门槛机制(谁有资格提交提案)}
\begin{itemize}
  \item 只有当提议者\textbf{持有或被委托}合计超过 \textbf{2,500,000 UNI} 时,才能提交提案供社区考虑。
  \item 该门槛最初是 \textbf{10,000,000 UNI},后来通过治理修改为 \textbf{2,500,000 UNI},以\textbf{降低提案门槛}。
\end{itemize}
\end{WarnBox}

\begin{KeyBox}{一句话总结(考点)}
\textbf{协议是产品本体,DAO 是治理系统。}Uniswap 用 UNI + 分阶段投票,把协议升级权从团队转移给社区,并用门槛设计平衡「开放参与」与「治理效率」。
\end{KeyBox}

% =========================================================
\subsection{社交:FWB DAO(门槛式社区俱乐部)}

\begin{DefBox}{是什麼}
\textbf{FWB(Friends with Benefits)DAO} 在这里不是常见语境的含义,而是一个\textbf{包罗万象的讨论社区}:更像「有门槛的俱乐部论坛」。
\end{DefBox}

\begin{KeyBox}{用途/用在哪}
\begin{itemize}
  \item \textbf{目标}:让 Web3 成为一种\textbf{文化现象}。
  \item \textbf{载体}:主要在 \textbf{Discord} 社区内开展讨论与活动;成员通过参与活动获得参与证明(类似 POAP/徽章逻辑)。
\end{itemize}
\end{KeyBox}

\begin{HowBox}{加入门槛(如何成为成员)}
\begin{itemize}
  \item 需要持有 \textbf{75 枚 FWB 通证}
  \item 填写申请表,并由社区\textbf{审议通过}
  \item 规模与价格(文本信息):约 \textbf{3000+} 成员;FWB 曾最高约 \textbf{190 美元/枚}
\end{itemize}
\end{HowBox}

% =========================================================
\subsection{与现实结合:CityDAO(用 DAO 运作一座城市的实验)}

\begin{DefBox}{是什麼}
\textbf{CityDAO} 的目标是建立一个「以 DAO 方式运作的城市」:把治理、参与与资源配置用 DAO 机制组织起来。
\end{DefBox}

\begin{KeyBox}{关键事实(文本信息)}
\begin{itemize}
  \item CityDAO 已在美国\textbf{怀俄明州}购买 \textbf{40 英亩}(约 16 万平方米)土地。
  \item 并将 DAO 社区已确定的旗帜插在该土地上,作为现实世界的象征性落地。
\end{itemize}
\end{KeyBox}

\begin{WarnBox}{提醒}
这类项目通常会遇到「链上治理」与「现实法律/产权/行政」之间的对接问题;现实落地比链上实验更复杂。
\end{WarnBox}

% =========================================================
\section{教你建立一个 DAO:Aragon(手把手流程)}

\begin{DefBox}{为什么建议从「建 DAO」开始体验 Web3?}
参与或自己建立一个 DAO 是亲身体验 Web3 的\textbf{低门槛方式}:建立一个 DAO 在工具支持下可以非常快(文本类比:像建一个群一样简单)。
\end{DefBox}

\begin{HowBox}{Aragon 建 DAO 流程(7 步)}
\begin{enumerate}
  \item 打开 \href{https://aragon.org/}{https://aragon.org/}
  \item 点击 \textbf{Create your DAO}
  \item 点击 \textbf{Create an organization},进入模板选择(见下一框)
  \item 给你的 DAO 起名字
  \item 设置投票规则(Support\%、Minimum Approval\%、Vote Duration)
  \item 设置通证信息:通证名称、总量、初始分发钱包
  \item 检查参数后点击 \textbf{Launch your organization} 完成创建
\end{enumerate}
\end{HowBox}

\begin{KeyBox}{Aragon 三种成熟模板(考点:差在「可否转让」与「投票方式」)}
\begin{itemize}
  \item \textbf{Company}:通证\textbf{可转让},代表对组织的所有权(更像股权)。
  \item \textbf{Membership}:通证\textbf{不可转让},投票采用\textbf{一人一票}。
  \item \textbf{Reputation}:通证\textbf{不可转让},投票采用\textbf{声誉加权}(Reputation-based Voting)。
\end{itemize}
\end{KeyBox}

\begin{HowBox}{投票参数怎么理解(把三格翻译成人话)}
\begin{itemize}
  \item \textbf{SUPPORT\%}:在「参与投票的人」里,需要多少比例赞成票才算通过。
  \item \textbf{MINIMUM APPROVAL\%}:在「所有通证持有人」里,需要多少比例赞成票才算通过(防止少数人低参与度强推)。
  \item \textbf{VOTE DURATION}:投票持续时间。
\end{itemize}
\end{HowBox}

\begin{KeyBox}{建完只是开始:DAO 运营要补什么?}
\begin{itemize}
  \item Aragon 还提供插件辅助运营:\textbf{争议处理、代理投票、金库管理}等。
  \item 真正难的是「\textbf{填充成员与业务}」:通证的发放与分配像股权一样决定激励与积极性。
  \item DAO 的「办公场所」往往是线上协作空间:成员跨地域、无需面试、自愿加入、完全远程协作。
\end{itemize}
\end{KeyBox}

\begin{WarnBox}{最重要的坑(先记一句)}
\textbf{通证分配与经济模型}会直接决定 DAO 的长期激励与治理稳定性;理解它往往需要一定经济学基础。
\end{WarnBox}


% =========================================================
% =========================
% Updated Review Section
% Covers: DAO + Governance + DID/SBT + Cases + Uniswap + Aragon
% =========================

\section{快速複習題(自測用)}

\begin{KeyBox}{Q1:為什麼說 DAO 是 Web3 的「社會作業系統」?}
A:因為 DAO 解決的不是單一功能,而是「\textbf{人如何協作、如何決策、資金如何分配}」的社會關係問題;就像作業系統協調硬體與軟體,DAO 協調人與資源。
\end{KeyBox}

\begin{KeyBox}{Q2:DAO 與傳統公司的三個根本差異是什麼?}
A:\textbf{(1)規則}:公司靠法律與合同,DAO 靠鏈上合約与代码;  
\textbf{(2)利益分配}:公司偏中心化,DAO 参与者常同时是通证持有人;  
\textbf{(3)身份边界}:公司层级固定,DAO 允许自由进出与跨组织协作。
\end{KeyBox}

\begin{KeyBox}{Q3:為什麼現實中的 DAO 常會演化成 DO?}
A:因為許多目標與工作\textbf{無法完全用智能合約自動化},仍需要人工協作與判断;但決策權仍盡量保持去中心化。
\end{KeyBox}

\begin{KeyBox}{Q4:DAO 的典型組織結構為什麼不是「所有事情都投票」?}
A:為避免協調成本過高,多數 DAO 會授權\textbf{核心團隊/委員會}處理日常事務;重大方向與資金仍由\textbf{社群投票}決定。
\end{KeyBox}

\begin{KeyBox}{Q5:什麼是 DID?為什麼可以把它理解為 Web3 的身份證?}
A:DID 是\textbf{全局唯一、去中心化、可鏈上驗證}的識別碼;它背後連結的是你在 Web3 的\textbf{行為與社會關係總和},而不是一張單純的卡片。
\end{KeyBox}

\begin{KeyBox}{Q6:POAP 在 Web3 身份系統中扮演什麼角色?}
A:POAP 是用 NFT 形式發放的\textbf{出席/参与证明};它記錄你參與過的活動與社群,可作為未來准入或声誉展示的依据。
\end{KeyBox}

\begin{KeyBox}{Q7:為什麼需要 SBT?它解決了什麼問題?}
A:因為 NFT \textbf{可以轉手},會導致身份被買賣;SBT 是\textbf{不可轉讓的 NFT},能更可靠地綁定「錢包 ↔ 真實經歷」。
\end{KeyBox}

\begin{KeyBox}{Q8:RabbitHole 的 SBT 設計想解決什麼?}
A:確保擁有某成就徽章的錢包\textbf{真的完成過對應任務},避免身份與成就被轉賣或偽造。
\end{KeyBox}

\begin{KeyBox}{Q9:Uniswap 為什麼是「協議 + DAO」的經典案例?}
A:因為 Uniswap 的\textbf{交易功能由協議提供},而\textbf{協議升級與治理由 DAO 決定};UNI 的引入把決策權從團隊轉移給社群。
\end{KeyBox}

\begin{KeyBox}{Q10:Uniswap 提案為什麼要分三個投票階段?}
A:為了在\textbf{開放參與}與\textbf{治理效率}之間取得平衡:  
先用低門檻篩想法(熱度投票),再建立共識,最後才進入高門檻的正式治理投票。
\end{KeyBox}

\begin{KeyBox}{Q11:FWB DAO 與一般 DAO 最大的不同是什麼?}
A:FWB 更像一個\textbf{有門檻的文化社群/俱樂部},重點在身份、文化與社交,而非資產治理或協議升級。
\end{KeyBox}

\begin{KeyBox}{Q12:CityDAO 的實驗性在哪裡?}
A:它嘗試把 DAO 的治理模式\textbf{落地到現實世界的土地與城市概念}中,面臨鏈上治理與現實法律/行政的銜接挑戰。
\end{KeyBox}

\begin{KeyBox}{Q13:投資型 DAO(如 Cult.DAO)與慈善 DAO(如 Gitcoin)的核心差別?}
A:\textbf{是否期待財務回報}。  
投資型 DAO 追求投資收益;慈善 DAO 聚焦公共資助與產業引導,不以回報為目的。
\end{KeyBox}

\begin{KeyBox}{Q14:為什麼說「建立 DAO 很簡單,但經營很難」?}
A:工具(如 Aragon)讓建 DAO 變得容易;但\textbf{通证分配、激励机制、治理设计}決定了 DAO 能否长期运作。
\end{KeyBox}

\begin{KeyBox}{Q15:Aragon 的三種 DAO 模板差異重點是什麼?}
A:\textbf{通证是否可轉讓}與\textbf{投票權如何計算}:  
Company(可轉讓、像股權)/Membership(不可轉讓、一人一票)/Reputation(不可轉讓、聲譽加權)。
\end{KeyBox}

\begin{WarnBox}{最後一句總結(考前記這句)}
Web3 的核心不是「把公司搬到鏈上」,而是\textbf{用協議與身份系統,重組人、資源與激勵的關係};  
DAO 解決組織,DID/SBT 解決身份,兩者共同塑造 Web3 的社會結構。
\end{WarnBox}


\end{document}
