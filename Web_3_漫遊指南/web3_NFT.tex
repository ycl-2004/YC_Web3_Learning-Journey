% !TEX program = xelatex
\documentclass[11pt,a4paper]{article}

% =========================
% style.tex  (ALL STYLES)
% =========================

% ---------- Page & Fonts ----------
\usepackage[a4paper,margin=1in]{geometry}
\usepackage{fontspec}
\usepackage{xeCJK}

% CJK font fallbacks (Overleaf 常見可用 Noto)
\IfFontExistsTF{Noto Serif CJK TC}{
  \setCJKmainfont{Noto Serif CJK TC}
}{
  \IfFontExistsTF{Noto Sans CJK TC}{
    \setCJKmainfont{Noto Sans CJK TC}
  }{
    \setCJKmainfont{SimSun}
  }
}
\setmainfont{TeX Gyre Termes}

% ---------- Packages ----------
\usepackage{hyperref}
\usepackage{microtype}
\usepackage{booktabs}
\usepackage{tabularx}
\usepackage{enumitem}
\usepackage{tikz}
\usetikzlibrary{positioning}
\usepackage{xcolor}
\usepackage{tcolorbox}
\tcbuselibrary{skins,breakable}
\usepackage{titlesec}
\usepackage{fancyhdr}

% ---------- Color Palette ----------
\definecolor{ink}{HTML}{1F2937}
\definecolor{muted}{HTML}{6B7280}
\definecolor{brand}{HTML}{2563EB}
\definecolor{brand2}{HTML}{0EA5E9}
\definecolor{soft}{HTML}{EFF6FF}
\definecolor{warn}{HTML}{F59E0B}
\definecolor{good}{HTML}{10B981}
\definecolor{rose}{HTML}{F43F5E}
\definecolor{paper}{HTML}{FFFFFF}

% ---------- Hyperref ----------
\hypersetup{
  colorlinks=true,
  linkcolor=brand,
  urlcolor=brand2,
  citecolor=brand
}

% ---------- Global typography ----------
\setlength{\parindent}{0pt}
\setlength{\parskip}{6pt}

% Clean lists inside boxes
\setlist[itemize]{itemsep=4pt, topsep=4pt, leftmargin=1.4em}
\setlist[enumerate]{itemsep=4pt, topsep=4pt, leftmargin=1.6em}

% ---------- Header ----------
\pagestyle{fancy}
\fancyhf{}
\lhead{\textcolor{muted}{Web3 / DApps 筆記}}
\rhead{\textcolor{muted}{\thepage}}
\renewcommand{\headrulewidth}{0pt}

% ---------- Section style ----------
\titleformat{\section}{\Large\bfseries\color{ink}}{\thesection}{0.8em}{}
\titleformat{\subsection}{\large\bfseries\color{ink}}{\thesubsection}{0.8em}{}
\titleformat{\subsubsection}{\normalsize\bfseries\color{ink}}{\thesubsubsection}{0.8em}{}
\titlespacing*{\section}{0pt}{1.0em}{0.4em}
\titlespacing*{\subsection}{0pt}{0.8em}{0.3em}
\titlespacing*{\subsubsection}{0pt}{0.5em}{0.2em}

% ---------- tcolorbox base ----------
\tcbset{
  enhanced,
  breakable,
  colback=paper,
  colframe=brand!18,
  boxrule=0.5pt,
  arc=10pt,
  left=10pt,right=10pt,top=10pt,bottom=10pt,
  before skip=10pt, after skip=10pt
}

% ---------- Box components ----------
\newtcolorbox{KeyBox}[1]{
  colback=paper,
  colframe=brand!18,
  borderline west={4pt}{0pt}{brand},
  title=\textbf{#1},
  coltitle=ink,
  fonttitle=\bfseries,
  colbacktitle=paper,
  boxed title style={boxrule=0pt, colframe=paper}
}

\newtcolorbox{DefBox}[1]{
  colback=paper,
  colframe=brand2!18,
  borderline west={4pt}{0pt}{brand2},
  title=\textbf{#1},
  coltitle=ink,
  fonttitle=\bfseries,
  colbacktitle=paper,
  boxed title style={boxrule=0pt, colframe=paper}
}

\newtcolorbox{HowBox}[1]{
  colback=paper,
  colframe=good!18,
  borderline west={4pt}{0pt}{good},
  title=\textbf{#1},
  coltitle=ink,
  fonttitle=\bfseries,
  colbacktitle=paper,
  boxed title style={boxrule=0pt, colframe=paper}
}

\newtcolorbox{WarnBox}[1]{
  colback=paper,
  colframe=warn!22,
  borderline west={4pt}{0pt}{warn},
  title=\textbf{#1},
  coltitle=ink,
  fonttitle=\bfseries,
  colbacktitle=paper,
  boxed title style={boxrule=0pt, colframe=paper}
}

% ---------- Tag pill ----------
\newcommand{\tagpill}[1]{%
  \tikz[baseline=(X.base)]\node[
    fill=soft, draw=brand!25, rounded corners=6pt,
    inner xsep=7pt, inner ysep=2.5pt
  ](X){\small\textcolor{brand}{\textbf{#1}}};%
}

\renewcommand{\arraystretch}{1.18}

\begin{document}

% ---------- Title ----------
\begin{center}
  {\LARGE \textbf{NFT(非同質化通證)重點筆記}}\\[6pt]
  {\large \textcolor{muted}{只保留:是什麼|用途|用在哪|怎麼 work|例子}}\\[10pt]
  \tagpill{NFT}\quad \tagpill{Token}\quad \tagpill{Smart Contract}\quad \tagpill{ERC-721}\quad
  \tagpill{ERC-1155}\quad \tagpill{Digital Twin}
\end{center}

\begin{KeyBox}{快速總結(考前 30 秒)}
\begin{itemize}
  \item \textbf{一句話}:NFT 是一種\textbf{不可互換、通常不可分割}的通證,用\textbf{智能合約寫入唯一識別},把「真實世界的唯一性」帶進數位世界。
  \item \textbf{本質}:NFT 更像\textbf{區塊鏈上的證書/憑證}(provenance + ownership record),可用來對應\textbf{數位內容}或\textbf{現實資產/權益}(數位孿生)。
  \item \textbf{用在哪}:數位藝術/收藏、遊戲道具、門票/會員(訪問權)、實物綁定(可赎回)、域名/身份(ENS)等。
  \item \textbf{關鍵提醒}:買 NFT \textbf{不一定}等於買到「底層素材著作權」;以及 NFT 的\textbf{存儲方式}(鏈上/IPFS/中心化)會影響不可篡改性與安全性。
\end{itemize}
\end{KeyBox}

\tableofcontents

% =========================================================
\section{NFT 是什麼?(定位與定義)}

\begin{DefBox}{一句話定義}
\textbf{NFT(Non-Fungible Token)}是一種在區塊鏈上發行的\textbf{非同質化通證}:每個通證在智能合約中帶有\textbf{唯一識別資訊}(如 hash / tokenId / metadata 指針),因此\textbf{彼此不可互換},且多數情況\textbf{不可分割}。
\end{DefBox}

\begin{KeyBox}{FT vs NFT(抓住「可互換」這個核心差別)}
\begin{itemize}
  \item \textbf{同質化通證(FT)}:可互換、可拆分。例:\textbf{BTC/ETH},1 BTC 等於另一個 1 BTC(像兩張面額相同的鈔票)。
  \item \textbf{非同質化通證(NFT)}:不可互換、通常不可拆分。每個 NFT 都有\textbf{獨一無二}的識別資訊,因此「沒有兩個 NFT 完全相同」。
\end{itemize}
\end{KeyBox}

\begin{HowBox}{為什麼 NFT 被說成「數位孿生」的基礎?}
數位世界可以無限複製同一份檔案(像素/屬性完全一致),但現實世界的物品具有\textbf{唯一性}。\\
NFT 透過區塊鏈把「唯一識別 + 可驗證記錄」帶入數位世界,使某個數位物件/權益可以被當作\textbf{獨特的、可追溯的資產/憑證}來持有與交易。
\end{HowBox}

% =========================================================
\section{用途(能解決什麼問題?)}

\begin{KeyBox}{用途(回答:NFT 可以做什麼)}
\begin{itemize}
  \item \textbf{出处证明(Provenance)}:用唯一識別與鏈上記錄證明「這個東西從哪來、誰持有過」。
  \item \textbf{所有權/權益記錄}:像現實世界的證書一樣,NFT 可作為「所有權記錄」或「權益憑證」。
  \item \textbf{把資產與權利拆分為可交易的數位單位}:例如\textbf{門票/會員資格}、\textbf{遊戲道具}、\textbf{實物兑付權}等。
\end{itemize}
\end{KeyBox}

\begin{WarnBox}{NFT 是资产吗?(考點式回答)}
\begin{itemize}
  \item \textbf{可以是「资产」}:當 NFT 代表\textbf{可交易的權益/所有權/使用權}(例如門票、會員、兑付權、遊戲道具)時,市場會把它當作資產定價。
  \item \textbf{但本質更像「鏈上憑證」}:NFT 自身常代表的是\textbf{某種權利或對應關係},而非一定等於「你已擁有底層內容/實物的全部法律權利」。
\end{itemize}
\end{WarnBox}

% =========================================================
\section{用在哪(典型場景 / 類型)}

\begin{KeyBox}{NFT 常見 5 類應用(背分類 + 例子就夠)}
\begin{itemize}
  \item \textbf{數位藝術/收藏 NFT}:圖片、文字、攝影、音樂、影片等(常被稱為「數位藏品」)。
  \item \textbf{遊戲道具 NFT}:道具/角色/土地等;強調「玩家可真正持有」與跨場景的可組合性想像。
  \item \textbf{訪問權 NFT}:門票、會員資格、粉絲社群准入(可降低黃牛/可查出票與交易記錄)。
  \item \textbf{數位孿生 NFT}:與實物 1:1 綁定(可赎回);赎回後 NFT 可能被\textbf{銷毀}以維持對應關係。
  \item \textbf{身份/域名 NFT}:如 \textbf{ENS} 域名,把難讀的地址映射為可讀名字(例:vitalik.eth)。
\end{itemize}
\end{KeyBox}

% =========================================================
\section{怎麼 work(機制 / 合約標準 / 存儲)}

\subsection{唯一性從哪裡來?}

\begin{HowBox}{NFT 的「唯一性」怎麼被保證?}
\begin{itemize}
  \item NFT 在\textbf{智能合約}中帶有\textbf{識別資訊}(tokenId / metadata / hash 等)。
  \item 這些資訊讓每個 NFT 可被鏈上驗證,且\textbf{不能被另一枚通證直接替代}。
  \item 交易與持有變更會留下\textbf{可追溯的鏈上記錄},因此能做出处证明。
\end{itemize}
\end{HowBox}

\subsection{合約標準(ERC-721 / ERC-1155 / ERC-3664)}

\begin{DefBox}{NFT 是什麼層級的東西?}
NFT 可理解為\textbf{一套智能合約規格/標準}下鑄造出的通證。常見標準包括 \textbf{ERC-721} 與 \textbf{ERC-1155},並有更偏遊戲/元宇宙特性的延伸(如文中提到的 \textbf{ERC-3664})。
\end{DefBox}

\begin{KeyBox}{ERC-721 vs ERC-1155(只記「定位差異」)}
\begin{itemize}
  \item \textbf{ERC-721}:更偏「\textbf{收藏/投資}」場景的經典標準;用於數位藏品、證明型資產等。
  \item \textbf{ERC-1155}:更偏「\textbf{功能型/大量資產}」場景;一個合約可管理\textbf{多種類/多份}通證,常見於 Web3 遊戲資產。
\end{itemize}
\end{KeyBox}

\begin{HowBox}{ERC-3664(文中定位:更像「下一代遊戲 NFT」)}
文中強調其更貼合元宇宙/遊戲:\textbf{屬性可升級、可修改、可添加/移除、可組合、可拆分},讓遊戲道具能像「可成長的物件」那樣被合約定義。
\end{HowBox}

\subsection{NFT 存在哪裡?(鏈上 / IPFS / 中心化)}

\begin{KeyBox}{NFT 資料存儲的 3 種方式(安全性由高到低)}
\begin{itemize}
  \item \textbf{鏈上存儲}:安全級別最高、最符合不可篡改(例:文中提到 CryptoPunks、Uniswap V3 的 LP NFT)。
  \item \textbf{IPFS 分佈式存儲}:安全性相對高、去中心化程度較好(例:文中提到 Doodles、无聊猿等使用 IPFS)。
  \item \textbf{中心化伺服器存儲}:安全級別最低、受單點影響最大(不符合 Web3 去中心化精神)。
\end{itemize}
\end{KeyBox}

\begin{WarnBox}{買到 NFT = 買到著作權嗎?(超高頻陷阱)}
\begin{itemize}
  \item \textbf{不一定。}NFT 交易的是「通證所有權/權益」,但底層素材的\textbf{著作權}要看項目的授權條款。
  \item 文中給出三種常見授權:\textbf{完全開放}/\textbf{完全保留}/\textbf{有限開放}(商用有上限或需談版稅)。
\end{itemize}
\end{WarnBox}

% =========================================================
\section{例子(用案例把分類與機制對上)}

\subsection{例子 1:UniSocks(數位孿生 / 可赎回)}

\begin{DefBox}{UniSocks(數位孿生 NFT)}
UniSocks 是一種與實物商品\textbf{1:1 綁定}的 NFT:持有人可選擇\textbf{赎回}對應實物;赎回後其 NFT 可能被\textbf{銷毀}以維持唯一對應。
\end{DefBox}

\begin{HowBox}{它怎麼 work(對回「數位孿生」的邏輯)}
\begin{itemize}
  \item NFT 作為「\textbf{兑付權/所有權記錄}」存在於鏈上。
  \item 持有人可在「持有(當資產)」與「赎回(換成實物)」之間選擇。
  \item 若赎回,需有機制避免一物多賣(文中描述:赎回後 NFT 被銷毀)。
\end{itemize}
\end{HowBox}

\subsection{例子 2:ENS(身份/域名 NFT)}

\begin{DefBox}{ENS(Ethereum Name Service)}
ENS 是以太坊的域名系統:把難讀的錢包地址映射成可讀域名(例如 \texttt{vitalik.eth}),域名以 NFT 形式儲存在錢包中,可被交易轉讓。
\end{DefBox}

\begin{KeyBox}{用在哪(身份/可讀性)}
\begin{itemize}
  \item \textbf{身份呈現}:用域名替代地址,提高可識別性。
  \item \textbf{資產/權利化}:域名可作為可交易的權益(像網域一樣)。
\end{itemize}
\end{KeyBox}

\subsection{例子 3:BAYC(数位艺术 + 社群/商用權)}

\begin{DefBox}{BAYC(数字艺术 NFT + 社群身份)}
BAYC 是頭像類(PFP)數位藝術 NFT 系列;除了收藏與交易,還被強調為一種\textbf{身份認同與社群歸屬}的載體。
\end{DefBox}

\begin{HowBox}{它怎麼 work(為什麼不只是圖片)}
\begin{itemize}
  \item NFT 作為「\textbf{身份標誌}」:持有者常把它作為社交頭像,形成社群可見的身份信號。
  \item \textbf{商用授權}(文中取向):交易後著作權/商用權授權給持有者,使其可做衍生與二創(提升使用價值)。
  \item \textbf{社群運營}:持有者社群與活動(聚會、私域)把 NFT 從「收藏」推向「使用」。
\end{itemize}
\end{HowBox}

% =========================================================
\section{NFT 交易平台(用在哪:買賣 / 發行 / 流動性)}

\begin{KeyBox}{你要記住的框架(平台差在哪)}
NFT 交易平台的差異通常看 4 件事:
\begin{itemize}
  \item \textbf{審核門檻}:誰都能上架 vs 平台嚴格審核(影響詐騙風險/品質)
  \item \textbf{付款方式}:純加密錢包 vs 可刷卡/美元(影響新手門檻)
  \item \textbf{鏈生態}:Ethereum / Solana / Polygon 等(影響社群與資產集中度)
  \item \textbf{平台治理}:公司化平台 vs DAO/社群化平台(是否更符合 Web3 精神)
\end{itemize}
\end{KeyBox}

\subsection{OpenSea(最大型綜合 NFT 市集)}
\begin{DefBox}{是什麼}
\textbf{OpenSea} 是全球規模最大的 NFT 交易平台之一,支援多條主流公鏈的 NFT 進行一級/二級市場交易。
\end{DefBox}
\begin{KeyBox}{用途 / 用在哪}
\begin{itemize}
  \item \textbf{用途}:買賣各類 NFT、瀏覽系列、掛單/出價、做市場流動性。
  \item \textbf{用在哪}:多鏈資產交易的「大市場」,常作為價格與地板價(Floor Price)的參考入口。
\end{itemize}
\end{KeyBox}
\begin{WarnBox}{注意(風險點)}
文中強調 OpenSea 對發行方審核不嚴、誰都能上架;即使有藍勾驗證,也不等於作品權威或價值保證,因此買家需自行判斷真偽與價值。
\end{WarnBox}

\subsection{Nifty Gateway(審核制 + 可用美元)}
\begin{DefBox}{是什麼}
\textbf{Nifty Gateway} 是 Gemini 交易所旗下的 NFT 平台,具備較嚴格的上架審核機制,並支援以美元方式購買。
\end{DefBox}
\begin{KeyBox}{用途 / 用在哪}
\begin{itemize}
  \item \textbf{用途}:更偏「精品/策展」型 NFT 發售與交易;降低新手換幣門檻。
  \item \textbf{用在哪}:希望用 \textbf{美元/銀行卡} 直接購買 NFT 的情境。
\end{itemize}
\end{KeyBox}

\subsection{Axie Marketplace(遊戲專用市場)}
\begin{DefBox}{是什麼}
\textbf{Axie Marketplace} 是專門交易 \textbf{Axie Infinity} 遊戲資產(角色/道具/土地等 NFT)的官方/核心市場。
\end{DefBox}
\begin{KeyBox}{用途 / 用在哪}
\begin{itemize}
  \item \textbf{用途}:遊戲內資產交易與定價,服務單一遊戲生態。
  \item \textbf{用在哪}:Web3 遊戲中「資產即 NFT」的典型案例(用市場承接玩家流動性)。
\end{itemize}
\end{KeyBox}

\subsection{Magic Eden(Solana 生態主力市場)}
\begin{DefBox}{是什麼}
\textbf{Magic Eden} 是 Solana 生態的重要 NFT 交易平台,常被稱為「Solana 上的 OpenSea」。
\end{DefBox}
\begin{KeyBox}{用途 / 用在哪}
\begin{itemize}
  \item \textbf{用途}:Solana 鏈上 NFT 的主要交易入口與流動性匯聚地。
  \item \textbf{用在哪}:Solana NFT 生態中高頻交易與熱門系列聚集處。
\end{itemize}
\end{KeyBox}

\subsection{X2Y2(更偏 Web3 精神的交易平台:社群/DAO 取向)}
\begin{DefBox}{是什麼}
\textbf{X2Y2} 被文中描述為更偏去中心化的 NFT 交易平台:所有權屬於 Web3 社群,以 DAO 自治取向運作。
\end{DefBox}
\begin{KeyBox}{用途 / 用在哪}
\begin{itemize}
  \item \textbf{用途}:提供 NFT 市集交易功能,同時以「社群所有/治理」吸引偏好 Web3 理念的使用者。
  \item \textbf{用在哪}:對比公司化平台(如 OpenSea)時常被拿來談「平台是否去中心化」。
\end{itemize}
\end{KeyBox}

\subsection{Gem(聚合器:一次看多個市場)}
\begin{DefBox}{是什麼}
\textbf{Gem} 不是單一交易平台,而是 NFT 市集\textbf{聚合器}:把多個平台的掛單匯總,支援批量購買/批量掛單以節省成本。
\end{DefBox}
\begin{KeyBox}{用途 / 用在哪}
\begin{itemize}
  \item \textbf{用途}:一站式比價與掃貨;批量操作降低手續費與時間成本。
  \item \textbf{用在哪}:需要在多個 NFT 市場間快速搜尋、集中下單的情境。
\end{itemize}
\end{KeyBox}


% =========================================================
\section{知名 NFT Project(例子:三個代表性系列)}

\begin{KeyBox}{本節只背三件事}
\begin{itemize}
  \item 這些 Project 為什麼紅:\textbf{共識度 + 稀缺性 + 社群運營 + 使用/身份屬性}
  \item 它們不只是圖片:常結合\textbf{社群准入、商用授權、活動/空投、擴展故事}
  \item 評估時要看:\textbf{存儲方式}與\textbf{著作權授權條款}(能不能二創/商用)
\end{itemize}
\end{KeyBox}

\subsection{Azuki(PFP 系列 + 強社群 + 合約創新)}
\begin{DefBox}{是什麼}
\textbf{Azuki} 是頭像類(PFP)NFT 系列,以角色風格、配飾稀缺度與社群運營建立共識,並探索 IP 延展與現實活動結合。
\end{DefBox}
\begin{KeyBox}{用途 / 用在哪}
\begin{itemize}
  \item \textbf{用在哪}:社交頭像身份、社群歸屬、IP 擴展與二創空間(利於傳播)。
  \item \textbf{用途}:把 NFT 從「收藏」推向「品牌/IP」:讓持有者更願意公開展示、參與社群。
\end{itemize}
\end{KeyBox}
\begin{HowBox}{怎麼 work(你要能講出它做了什麼)}
\begin{itemize}
  \item \textbf{共識與社群建設}:透過白名單/社群機制篩選高共識成員,形成長期品牌效應。
  \item \textbf{合約創新(文中重點)}:提到 ERC-721A 降低鑄造成本(一次 Gas 可鑄造多枚)。
  \item \textbf{回饋機制}:空投(如 BEANZ)擴大系列與市場關注,提升持有者體感。
\end{itemize}
\end{HowBox}
\begin{WarnBox}{注意}
文中提到 Azuki 曾出現信任危機事件;此類 Project 的後續走勢高度依賴\textbf{團隊運營}與\textbf{社群信任}是否能維持。
\end{WarnBox}

\subsection{BAYC(无聊猿:身份社群 + 商用權)}
\begin{DefBox}{是什麼}
\textbf{Bored Ape Yacht Club(BAYC)} 是由 Yuga Labs 打造的現象級 PFP NFT 系列,總量 10,000,採 ERC-721 標準,並強調「收藏 + 身份 + 社群」三位一體。
\end{DefBox}
\begin{KeyBox}{用途 / 用在哪}
\begin{itemize}
  \item \textbf{用在哪}:作為社交身份象徵、社群准入與活動參與的門票(身份/圈層屬性很強)。
  \item \textbf{用途}:將 NFT 從「圖片」升級為「可商用的 IP 資產 + 社群會員資格」。
\end{itemize}
\end{KeyBox}
\begin{HowBox}{怎麼 work(成功機制:抓 3 點)}
\begin{itemize}
  \item \textbf{稀缺性 + 屬性稀有度}:系列內個體價格差異常來自特徵稀缺度與市場共識。
  \item \textbf{社群運營與玩法}:不斷釋放新玩法/活動,維持注意力與共識。
  \item \textbf{商用授權}:文中強調交易後著作權/商用權授權給持有者,提升衍生與二創價值。
\end{itemize}
\end{HowBox}

\subsection{Cheers UP(CUP:官方授權 + 平台身份標識)}
\begin{DefBox}{是什麼}
\textbf{Cheers UP(CUP)} 是文中提到的海外 NFT 系列,具備官方授權背景與固定總量機制(盲盒態/開圖態兩種形態,總量封頂)。
\end{DefBox}
\begin{KeyBox}{用途 / 用在哪}
\begin{itemize}
  \item \textbf{用在哪}:作為社群文化符號與身份展示(文中提到可設為平台海外站頭像並獲得特殊標識)。
  \item \textbf{用途}:用「授權 + 社群文化」提高共識起點,並透過活動/空投/積分等方式維持互動。
\end{itemize}
\end{KeyBox}
\begin{WarnBox}{注意}
這類依賴授權與平台生態的系列,後續表現很吃\textbf{社群熱度}與\textbf{項目方運營節奏};高起點不等於能長期維持。
\end{WarnBox}


% =========================================================
\section{快速複習題(自測用)}

\begin{KeyBox}{Q1:NFT 是资产吗?}
A:NFT 可以被市場當作資產交易,但本質更像鏈上\textbf{憑證/權益記錄}:它可代表所有權、访问权、兑付权或某種对照关系;是否「真擁有」底層內容/實物的完整權利,要看授權與法律安排。
\end{KeyBox}

\begin{KeyBox}{Q2:FT 和 NFT 的核心差別是什麼?}
A:\textbf{可互換性}。FT(如 BTC/ETH)同質可互換且可拆分;NFT 具有唯一識別,\textbf{不可互換}且多數\textbf{不可分割}。
\end{KeyBox}

\begin{KeyBox}{Q3:買 NFT 為什麼不一定買到著作權?}
A:因為 NFT 交易的是\textbf{通證所有權/權益},但底層素材著作權取決於項目的授權條款(完全開放/完全保留/有限開放)。
\end{KeyBox}

\begin{WarnBox}{最後一句總結(考前記這句)}
NFT 用智能合約把「\textbf{唯一性 + 可追溯}」帶進數位世界,讓數位內容與現實權益能以\textbf{憑證/權益}形式交易;但要特別注意\textbf{存儲方式}與\textbf{著作權授權}兩個坑。
\end{WarnBox}

\end{document}
