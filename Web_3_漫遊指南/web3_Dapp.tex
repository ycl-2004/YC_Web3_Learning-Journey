% !TEX program = xelatex
\documentclass[11pt,a4paper]{article}

% =========================
% style.tex  (ALL STYLES)
% =========================

% ---------- Page & Fonts ----------
\usepackage[a4paper,margin=1in]{geometry}
\usepackage{fontspec}
\usepackage{xeCJK}

% CJK font fallbacks (Overleaf 常見可用 Noto)
\IfFontExistsTF{Noto Serif CJK TC}{
  \setCJKmainfont{Noto Serif CJK TC}
}{
  \IfFontExistsTF{Noto Sans CJK TC}{
    \setCJKmainfont{Noto Sans CJK TC}
  }{
    \setCJKmainfont{SimSun}
  }
}
\setmainfont{TeX Gyre Termes}

% ---------- Packages ----------
\usepackage{hyperref}
\usepackage{microtype}
\usepackage{booktabs}
\usepackage{tabularx}
\usepackage{enumitem}
\usepackage{tikz}
\usetikzlibrary{positioning}
\usepackage{xcolor}
\usepackage{tcolorbox}
\tcbuselibrary{skins,breakable}
\usepackage{titlesec}
\usepackage{fancyhdr}

% ---------- Color Palette ----------
\definecolor{ink}{HTML}{1F2937}
\definecolor{muted}{HTML}{6B7280}
\definecolor{brand}{HTML}{2563EB}
\definecolor{brand2}{HTML}{0EA5E9}
\definecolor{soft}{HTML}{EFF6FF}
\definecolor{warn}{HTML}{F59E0B}
\definecolor{good}{HTML}{10B981}
\definecolor{rose}{HTML}{F43F5E}
\definecolor{paper}{HTML}{FFFFFF}

% ---------- Hyperref ----------
\hypersetup{
  colorlinks=true,
  linkcolor=brand,
  urlcolor=brand2,
  citecolor=brand
}

% ---------- Global typography ----------
\setlength{\parindent}{0pt}
\setlength{\parskip}{6pt}

% Clean lists inside boxes
\setlist[itemize]{itemsep=4pt, topsep=4pt, leftmargin=1.4em}
\setlist[enumerate]{itemsep=4pt, topsep=4pt, leftmargin=1.6em}

% ---------- Header ----------
\pagestyle{fancy}
\fancyhf{}
\lhead{\textcolor{muted}{Web3 / DApps 筆記}}
\rhead{\textcolor{muted}{\thepage}}
\renewcommand{\headrulewidth}{0pt}

% ---------- Section style ----------
\titleformat{\section}{\Large\bfseries\color{ink}}{\thesection}{0.8em}{}
\titleformat{\subsection}{\large\bfseries\color{ink}}{\thesubsection}{0.8em}{}
\titleformat{\subsubsection}{\normalsize\bfseries\color{ink}}{\thesubsubsection}{0.8em}{}
\titlespacing*{\section}{0pt}{1.0em}{0.4em}
\titlespacing*{\subsection}{0pt}{0.8em}{0.3em}
\titlespacing*{\subsubsection}{0pt}{0.5em}{0.2em}

% ---------- tcolorbox base ----------
\tcbset{
  enhanced,
  breakable,
  colback=paper,
  colframe=brand!18,
  boxrule=0.5pt,
  arc=10pt,
  left=10pt,right=10pt,top=10pt,bottom=10pt,
  before skip=10pt, after skip=10pt
}

% ---------- Box components ----------
\newtcolorbox{KeyBox}[1]{
  colback=paper,
  colframe=brand!18,
  borderline west={4pt}{0pt}{brand},
  title=\textbf{#1},
  coltitle=ink,
  fonttitle=\bfseries,
  colbacktitle=paper,
  boxed title style={boxrule=0pt, colframe=paper}
}

\newtcolorbox{DefBox}[1]{
  colback=paper,
  colframe=brand2!18,
  borderline west={4pt}{0pt}{brand2},
  title=\textbf{#1},
  coltitle=ink,
  fonttitle=\bfseries,
  colbacktitle=paper,
  boxed title style={boxrule=0pt, colframe=paper}
}

\newtcolorbox{HowBox}[1]{
  colback=paper,
  colframe=good!18,
  borderline west={4pt}{0pt}{good},
  title=\textbf{#1},
  coltitle=ink,
  fonttitle=\bfseries,
  colbacktitle=paper,
  boxed title style={boxrule=0pt, colframe=paper}
}

\newtcolorbox{WarnBox}[1]{
  colback=paper,
  colframe=warn!22,
  borderline west={4pt}{0pt}{warn},
  title=\textbf{#1},
  coltitle=ink,
  fonttitle=\bfseries,
  colbacktitle=paper,
  boxed title style={boxrule=0pt, colframe=paper}
}

% ---------- Tag pill ----------
\newcommand{\tagpill}[1]{%
  \tikz[baseline=(X.base)]\node[
    fill=soft, draw=brand!25, rounded corners=6pt,
    inner xsep=7pt, inner ysep=2.5pt
  ](X){\small\textcolor{brand}{\textbf{#1}}};%
}


% ---------------- Document ----------------
\begin{document}

\begin{center}
  {\LARGE \textbf{Web3 / DApps 重點筆記}}\\[6pt]
  {\large \textcolor{muted}{只保留:是什麼|用途|用在哪|怎麼 work|例子}}\\[10pt]
  \tagpill{DApps}\quad \tagpill{Infra}\quad \tagpill{IPFS}\quad \tagpill{Storj}\quad \tagpill{DeFi}\quad
  \tagpill{SocialFi}\quad \tagpill{GameFi}
\end{center}

\begin{KeyBox}{快速總結(考前 30 秒)}
\begin{itemize}
  \item \textbf{DApps}:建在區塊鏈上、屬於\textbf{應用層}的去中心化應用;像 Web3 世界「看得見的磚瓦」。
  \item \textbf{去中心化對使用者的價值}:核心是\textbf{資料所有權}(Web2 多由平台控制)。
  \item \textbf{五層架構}:硬體層/資料層/網路層/共識層/應用層(有時拆出協議層)。
  \item \textbf{Web3:胖協議、瘦應用}:能力(如安全)被埋在協議裡;\textbf{智能合約}是典型例子。
  \item \textbf{基礎設施類 DApps}:常以「協議」存在,連接 DApps 與底層各層(例:\textbf{IPFS / Storj})。
\end{itemize}
\end{KeyBox}

\tableofcontents

% ---------------- 1) DApps ----------------
\section{DApps 是什麼?(定位與比喻)}

\begin{DefBox}{一句話定義}
\textbf{DApps(Decentralized Applications)}是建立在\textbf{區塊鏈}上的\textbf{去中心化應用},多位於最上層的\textbf{應用層(Application Layer)}。
\end{DefBox}

% ---------------- DApp 判定標準(加在「DApps 是什麼」章節中) ----------------
\subsection{DApps 判定標準(需要滿足什麼)}

\begin{DefBox}{一句話}
要成為去中心化應用(DApp),不只是在區塊鏈上運行;它還需要在\textbf{程式控制、通證分配、治理決策}三方面符合去中心化標準。
\end{DefBox}

\begin{KeyBox}{三大標準(整理版)}
\begin{enumerate}
  \item \textbf{開源 + 自動運行 + 無單一實體控制大多數通證}\\
  應用必須完全開源並能自動運行;不應由單一實體掌控其大部分通證。資料與記錄需加密後存放於\textbf{公鏈}。
  
  \item \textbf{通證需由標準化規則產生,並用於運行與激勵}\\
  通證必須依照標準算法/規則發行,且應在開始運營時就將部分或全部通證發放給使用者。應用運行需\textbf{依賴該通證},使用者的貢獻應以通證作為獎勵。
  
  \item \textbf{規則可更改,但需多數用戶認可(治理去中心化)}\\
  系統規則可因時制宜調整,但變更需被\textbf{大多數用戶認可},而非最終解釋權掌握在某個實體手中。
\end{enumerate}
\end{KeyBox}


\begin{KeyBox}{用途/用在哪}
\begin{itemize}
  \item \textbf{用途}:在 Web3 的數位世界中,提供使用者能直接互動的功能(交易、社交、遊戲、金融等)。
  \item \textbf{用在哪}:Web3.0 由區塊鏈構成的世界;DApps 是其中「看得見摸得著」的部分。
\end{itemize}
\end{KeyBox}

\begin{HowBox}{直覺比喻}
區塊鏈的底層(硬體/資料/網路/共識)像冰山在水面下的基礎設施;\textbf{DApps}像是水面上的「建築材料」,把底層能力變成可用的產品。
\end{HowBox}

% ---------------- 2) Decentralization value ----------------
% ---------- Replace / extend your Section 2 with the following ----------

\section{去中心化對使用者的重要性}

\begin{KeyBox}{核心問題:資料所有權(Data Ownership)}
\begin{itemize}
  \item Web2 常見狀況:使用者以為資料「屬於自己」,但平台在註冊條款中已掌握大量控制權。
  \item 去中心化要解決:\textbf{資料究竟歸平台,還是歸個人?}
\end{itemize}
\end{KeyBox}

\subsection{平台信任問題:規則黑箱 vs 鏈上可驗證}

\begin{DefBox}{問題本質}
在 Web2 平台上,\textbf{帳號與規則的最終解釋權}往往掌握在平台手中;使用者面對封禁/限制時,容易陷入「規則看得見,但判定過程看不見」的黑箱。
\end{DefBox}

\begin{HowBox}{例子(短影音創作者帳號無故封禁)}
\begin{itemize}
  \item 你花大量時間經營一個\textbf{百萬粉絲}帳號(這其實是你的數位資產)。
  \item 你想把帳號轉交他人繼續經營,僅把手機號換成接收方的手機號。
  \item 你再次登入時,平台卻把帳號\textbf{永久封禁},理由是「違反社區規則」。
  \item 你讀完規則也找不到對應條款;找客服只得到幾句制式回覆,無法知道真正原因。
\end{itemize}
\end{HowBox}

\begin{KeyBox}{DApps 的差異(透明度如何提升信任)}
\begin{itemize}
  \item \textbf{行為記錄公開透明}:帳號在 App 內的行為記錄、與 App/其他帳號互動,多數可在鏈上查驗。
  \item \textbf{規則可上鏈、可追溯}:社群規則/核心機制可在上線時就記錄在鏈上,避免平台事後「黑箱解釋」。
  \item \textbf{可驗證}:使用者能以鏈上資料驗證「發生了什麼」,減少單方裁量造成的不確定性。
\end{itemize}
\end{KeyBox}

\subsection{收益分配問題:平台/公司壟斷分配權 vs 協議化分配}

\begin{DefBox}{問題本質}
在 Web2 公司或平台體系中,\textbf{價值分配規則}通常由中心化機構決定,貢獻者(創作者/員工/開發者)未必能按貢獻即時、可預期地獲得回報。
\end{DefBox}

\begin{HowBox}{例子(大廠員工期權難以變現)}
\begin{itemize}
  \item 你與團隊做出一個\textbf{月活百萬}的 App,拿到老闆給的期權。
  \item 你想離職回老家,老闆說期權必須等「上市解禁」才能變現,但上市時間不確定。
  \item 若你現在走,公司只願意以\textbf{1 元/股}回購期權,你被迫繼續留下。
  \item 後來公司為上市選擇\textbf{合股}(例如 10 股合 1),你手上的期權價值被稀釋成原本的\textbf{1/10}。
\end{itemize}
\end{HowBox}

\begin{KeyBox}{DApps 的方向(協議化分配 + 貢獻者激勵)}
\begin{itemize}
  \item \textbf{貢獻可被記錄}:貢獻(內容/開發/治理/流動性等)可透過鏈上互動或規則化指標被追蹤。
  \item \textbf{分配規則可寫入協議/智能合約}:收益分配可由智能合約自動執行,降低「事後變更/單方稀釋」的不確定性。
  \item \textbf{通證激勵}:協議可用通證把收益/治理權分配給貢獻者,讓「使用者/貢獻者」更可能成為價值共同體的一部分。
\end{itemize}
\end{KeyBox}

\begin{WarnBox}{提醒:去中心化不等於零風險}
\begin{itemize}
  \item 規則上鏈提高透明度,但仍需要良好的治理與機制設計,否則也可能出現激勵失衡或濫用。
  \item 「公開透明」通常意味著可追蹤性更強;隱私保護需要額外機制(例如加密/權限/零知識證明等)。
\end{itemize}
\end{WarnBox}


% ---------------- 3) Layers ----------------
\section{區塊鏈五層架構(每層做什麼)}

\subsection{五層功能對照表}
\renewcommand{\arraystretch}{1.2}
\begin{tabularx}{\textwidth}{@{}lX@{}}
\toprule
\textbf{層級} & \textbf{主要功能(你要記得的動詞)} \\
\midrule
硬體層 (Hardware) & 節點計算機組成的 P2P 網路,\textbf{共同運算/驗證/記錄交易} \\
資料層 (Data) & 像資料庫:\textbf{存資料} + \textbf{保障帳戶/交易安全} \\
網路層 (Network) & \textbf{組網}、\textbf{傳播資料}、\textbf{驗證資料} \\
共識層 (Consensus) & 用共識算法/機制讓節點對\textbf{區塊有效性達成一致} \\
應用層 (Application) & DApps 多在此;有時拆出協議層(Protocol Layer) \\
\bottomrule
\end{tabularx}

\subsection{視覺化:五層冰山(由下到上)}
\begin{center}
\begin{tikzpicture}[font=\small]
  \node[draw=brand!18, fill=soft, rounded corners=12pt, inner sep=12pt] (box) {
    \begin{minipage}{0.92\textwidth}
    \centering
    \begin{tikzpicture}[node distance=6mm]
      \tikzstyle{layer}=[rounded corners=10pt, minimum width=0.86\textwidth, minimum height=10mm, align=center]
      \node[layer, fill=brand!10,  draw=brand!30]  (app)  {\textbf{應用層}:DApps(使用者看得到)};
      \node[layer, fill=good!10,   draw=good!30,   below=of app]  (con)  {\textbf{共識層}:共識算法 / 一致性};
      \node[layer, fill=brand2!10, draw=brand2!30, below=of con]  (net)  {\textbf{網路層}:組網 / 傳播 / 驗證};
      \node[layer, fill=warn!10,   draw=warn!30,   below=of net]  (data) {\textbf{資料層}:存儲 / 安全};
      \node[layer, fill=rose!10,   draw=rose!30,   below=of data] (hard) {\textbf{硬體層}:節點計算機(P2P)};
    \end{tikzpicture}

    \vspace{6pt}
    \textcolor{muted}{\small 記法:越底層越「看不見」,但越是上層能否運作的基礎。}
    \end{minipage}
  };
\end{tikzpicture}
\end{center}

% ---------------- 4) Fat protocol vs fat app ----------------
\section{Web2 vs Web3:瘦協議/胖應用 vs 胖協議/瘦應用}

\begin{DefBox}{概念對照}
\begin{itemize}
  \item \textbf{Web2:瘦協議、胖應用} —— 許多能力(如安全)靠外部應用(防火牆/殺毒/安全公司)完成。
  \item \textbf{Web3:胖協議、瘦應用} —— 能力被內建進傳輸/互動協議;\textbf{智能合約}是典型「協議」形態。
\end{itemize}
\end{DefBox}

\begin{HowBox}{一句話抓重點}
Web3 把「規則與保障」寫進協議(Protocol)與智能合約,所以很多 Web2 需要外掛 App 才做得到的事,在 Web3 可能直接由協議提供。
\end{HowBox}

\section{例子精讀(只保留用途/用在哪/怎麼 work)}

\subsection{Web3Graph:資料圖譜協議}

\begin{DefBox}{是什麼}
Web3Graph 是 Web3 時代的\textbf{資料圖譜協議}:為 Web3 原住民/應用提供圖譜資訊,讓資訊交流與呈現更快速、準確。
\end{DefBox}

\begin{KeyBox}{用途/用在哪}
\begin{itemize}
  \item \textbf{用在哪}:從\textbf{資料層到應用層};支援 GameFi / DeFi / SocialFi / DAO / NFT 等提供資料基礎。
  \item \textbf{用途}:建立統一、開放、安全的資料規範與圖譜事件/關係資料,幫助應用快速啟動。
\end{itemize}
\end{KeyBox}

\begin{HowBox}{怎麼 work(核心機制)}
\begin{itemize}
  \item 借鑑 Open Graph:引入 meta 等概念。
  \item 底層採用\textbf{二分圖(Bipartite Graph)}資料結構,可表達 Open Graph 的行為/關係(被描述為其「超集」)。
  \item 產品工具:API / SDK / Studio / Extension,讓開發者可直接整合協議並復用既有事件/關係圖譜資料做冷啟動。
\end{itemize}
\end{HowBox}

\begin{KeyBox}{經濟模型(為什麼能長期運作)}
\begin{itemize}
  \item 貢獻者與使用者不割裂:開發者可使用協議開發並獲得激勵,也能用資料做分析。
  \item 開放式生態:允許第三方基於基礎資料提供二次服務(例:信用評分),形成多層次資料使用市場。
  \item 協議價值:在\textbf{資料使用與流通}中被實現。
\end{itemize}
\end{KeyBox}

\subsection{ENS:以太坊域名系統}

\begin{DefBox}{是什麼}
ENS(Ethereum Name Service)是基於以太坊的\textbf{分散式命名系統},把可讀域名(如 \texttt{vitalik.eth})解析為地址/內容哈希/元資料等標識符,或反向解析。
\end{DefBox}

\begin{KeyBox}{用途/用在哪}
\begin{itemize}
  \item \textbf{用途}:用「好記的域名」替代長地址,降低轉帳與識別門檻。
  \item \textbf{用在哪}:轉帳、查地址動向、身份識別(域名可能成為象徵)。
\end{itemize}
\end{KeyBox}

\begin{HowBox}{怎麼 work(分層與規則)}
\begin{itemize}
  \item 類似 DNS:\textbf{分層域名},用點號分隔;域名擁有者可控制子域名。
  \item 頂級域名(如 \texttt{.eth})由\textbf{Registrar 智能合約}管理,合約規定子域名分配規則。
  \item 取得方式:遵循合約規則、\textbf{先到先得}。
\end{itemize}
\end{HowBox}

\subsection{Etherscan:以太坊區塊鏈瀏覽器}

\begin{DefBox}{是什麼}
Etherscan 是常用的以太坊區塊鏈瀏覽器,用於\textbf{查交易}與\textbf{查智能合約},可直接分析鏈上底層資料。
\end{DefBox}

\begin{HowBox}{怎麼用(操作流程)}
\begin{enumerate}
  \item 在首頁搜尋框輸入地址或 ENS(例如 \texttt{vitalik.eth})。
  \item 查看地址概覽:餘額、市值,以及交易列表欄位(交易編號/類型/區塊/時間/發起方/接收方/金額/交易費)。
  \item 點進任一筆交易 → 看交易詳細狀態與執行結果。
\end{enumerate}
\end{HowBox}

\begin{WarnBox}{常見失敗交易:你要會看}
\begin{itemize}
  \item \textbf{Reverted}:交易被退回(通常已支付交易費,但因技術/執行問題失敗)。
  \item \textbf{Out of Gas}:Gas 上限不足;交易費不是固定,發起方設定上限與速度,網路依擁堵扣費;上限太低會導致執行中止。
\end{itemize}
\end{WarnBox}


% ---------------- 5) Infra DApps ----------------
\section{基礎設施類 DApps:存儲與傳輸(IPFS / Storj)}

\subsection{IPFS(InterPlanetary File System)}
\begin{DefBox}{是什麼}
\textbf{IPFS} 是一種\textbf{內容尋址}、\textbf{版本化}、\textbf{點對點(P2P)}的超媒體傳輸協議,整合 P2P、BitTorrent、Git 版本控制等概念,被視為對標 HTTP 的新一代通信協議。
\end{DefBox}

\begin{KeyBox}{用途/用在哪}
\begin{itemize}
  \item \textbf{用途}:以「內容」而不是「伺服器位置」來定位檔案,適合去中心化存取。
  \item \textbf{常見用法}:存放 \textbf{NFT 源文件}(例:無聊猿系列 NFT 的檔案常用 IPFS 存儲)。
  \item \textbf{入口}:\href{https://ipfs.io/}{https://ipfs.io/}
\end{itemize}
\end{KeyBox}

\begin{HowBox}{怎麼 work(HTTP vs IPFS)}
\begin{itemize}
  \item \textbf{HTTP}:先找到「伺服器位置」→ 再用「路徑」找檔案。
  \item \textbf{IPFS}:直接用\textbf{內容哈希(hash)}找檔案;檔案會被分配唯一哈希(由檔案內容計算)。
  \item 當你查詢某個哈希時:IPFS 用\textbf{分散式哈希表(DHT)}快速定位擁有該資料的節點 → 取回資料。
  \item \textbf{大檔案}:會自動切成多個小塊,可從多個節點\textbf{並行同步下載},只要節點在線且網路正常,速度可很快。
\end{itemize}
\end{HowBox}

\subsection{Storj(分布式雲存儲)}
\begin{DefBox}{是什麼}
\textbf{Storj} 是 Web3 常用的分布式雲存儲工具:用戶可購買存儲服務,也可提供閒置硬碟空間當節點賺取回報。
\end{DefBox}

\begin{KeyBox}{用途/用在哪(通證模型)}
\begin{itemize}
  \item \textbf{購買者}:用平台通證 \textbf{STORJ} 購買存儲服務。
  \item \textbf{供給者(節點)}:提供閒置存儲空間,獲得 \textbf{STORJ} 回報。
  \item \textbf{入口}:\href{https://www.storj.io/}{https://www.storj.io/}
\end{itemize}
\end{KeyBox}

\begin{HowBox}{怎麼 work(加密分片 \texorpdfstring{$\rightarrow$}{->} 分發 \texorpdfstring{$\rightarrow$}{->} 重組)}
\begin{itemize}
  \item 上傳時:客戶端先\textbf{加密}資料,並\textbf{分解成多個碎片}。
  \item 分發:碎片透過網路分發給多個節點;同時客戶端生成「如何找到碎片位置」的資訊。
  \item 下載時:客戶端讀取定位資訊 → 找回碎片 → 在本機\textbf{重新組裝}成原始資料。
  \item 安全性:資料全程加密,\textbf{只有所有者有密鑰},其他節點無法解密。
\end{itemize}
\end{HowBox}

\begin{KeyBox}{Storj 三類角色}
\begin{itemize}
  \item \textbf{用戶端(Client)}:上傳/下載、加密/解密、重組資料。
  \item \textbf{節點(Node)}:提供存儲與帶寬,存放碎片。
  \item \textbf{衛星(Satellite)}:伺服器集群,連接用戶端與節點;協助找最快節點、並記錄支出/收益。
\end{itemize}
\end{KeyBox}

% ---------------- 6) DeFi ----------------
\section{金融借贷(DeFi)與典型案例:Aave}

\begin{DefBox}{為什麼 DeFi 借贷熱門?}
Web3 早期大量參與者帶有投機動機;\textbf{金融借贷類 DApps} 因「高收益」成為熱門領域。2020 年的 \textbf{DeFi Summer} 指的就是借贷類 DeFi 在 2020 夏季大量湧現。
\end{DefBox}

\begin{WarnBox}{風險觀點(文本要點)}
\begin{itemize}
  \item 高收益常來自\textbf{高槓桿}與「樂觀的一致性預期」。
  \item 熊市預期轉向可能引發\textbf{踩踏},加速泡沫破裂。
  \item 極端市場下,去中心化\textbf{無法自動解決流動性問題}(文本觀點)。
\end{itemize}
\end{WarnBox}

\subsection{Aave(去中心化借贷協議)}
\begin{DefBox}{是什麼}
\textbf{Aave} 是去中心化借贷系統:用戶可把加密貨幣存入資產池賺利息(存款人),也可抵押後借出其他資產(借款人)。存/借由協議自動完成,不需傳統中介。
\end{DefBox}

\begin{HowBox}{怎麼 work(存款 \texorpdfstring{$\rightarrow$}{->} aToken \texorpdfstring{$\rightarrow$}{->} 贖回)}
\begin{itemize}
  \item 存款:把資產存入 Aave 流動池,協議按存款額發放 \textbf{aToken} 作為存款憑證。
  \item 贖回:取回資產時,協議\textbf{收回並銷毀}對應 aToken。
  \item 例:存入 ETH \texorpdfstring{$\rightarrow$}{->} 得到 aETH;交還 aETH \texorpdfstring{$\rightarrow$}{->} 取回 ETH。
  \item aToken:與存入資產的市價\textbf{1:1 掛鈎},且可轉讓/交易,使存款人仍保有一定流動性(文本要點)。
\end{itemize}
\end{HowBox}

\begin{HowBox}{借款端(利率、抵押、LTV、清算)}
\begin{itemize}
  \item 借款需要先存入\textbf{抵押品},再從資產池借出其他資產。
  \item 利率類型:\textbf{浮動利率(Variable APY)}隨市場波動(適合短期);
        \textbf{固定利率(Stable APY)}借贷期間不變(文本描述)。
  \item \textbf{Max LTV}:不同抵押品有不同最高借贷比率。
        例:抵押價值 100 美元,若 Max LTV = 82.50\%,最多可借 82.5 美元等值資產。
  \item \textbf{Liquidity Threshold(清算臨界點)}:若貸款率超過臨界點,協議會\textbf{自動清算抵押品}並收取清算罰款。
  \item 例:用 ETH 抵押借款,若 ETH 價格下跌導致貸款率上升,可能觸發清算。
\end{itemize}
\end{HowBox}

\begin{KeyBox}{補充(文本提及)}
Aave 文本中提到其目前獲得英國的電子貨幣機構牌照;但許多其他 DeFi 仍以較原始方式運營。
\end{KeyBox}

% ---------------- 7) SocialFi ----------------
\section{社交類 DApps(SocialFi)與案例:Mirror}

\begin{DefBox}{SocialFi 是什麼?}
\textbf{SocialFi} 讓用戶把社交影響力「金融化」:創作內容、參與社群、NFT 鑄造、互動、觀看影片等都可能帶來收入。
\end{DefBox}

\begin{KeyBox}{相對 Web2 社交的優勢(文本要點)}
\begin{itemize}
  \item 更公平分配廣告/收益,提供更好的用戶體驗(文本主張)。
  \item 透過 \textbf{DAO} 維護:把平台管理權/話語權交還給用戶,終結「算法霸權」,形成創作者經濟與更公平的收益模型(文本表述)。
\end{itemize}
\end{KeyBox}

\subsection{Mirror(去中心化內容發布平台)}
\begin{DefBox}{是什麼}
\textbf{Mirror} 是建立在 \textbf{Arweave(去中心化存儲協議)}上的內容發布平台。文章可被存儲到 Arweave,並可鑄造成 NFT 轉贈/出售,甚至發起 NFT 眾籌。
\end{DefBox}

\begin{KeyBox}{Mirror 的 6 個基礎功能(文本列舉)}
\begin{itemize}
  \item 發布作品(Entries)
  \item 發起眾籌(Crowdfunds)
  \item 發布 NFT(Editions)
  \item 發起競拍(Auctions)
  \item 分享收益(Splits)
  \item 發起投票(Token Race)
\end{itemize}
\end{KeyBox}

\begin{HowBox}{發布文章 / 同時鑄成 NFT(流程 + 例子)}
\begin{itemize}
  \item 發文時可選擇「同步鑄成 NFT」:需上傳一張\textbf{2:1} 寬高比圖片並支付交易費。
  \item 若不勾選 NFT:無需交易費,但也沒有直接變現通道(文本描述)。
  \item 文章 NFT 常見三檔定價(文本例):\textbf{1 ETH / 0.1 ETH / 0.01 ETH},類似打賞的預設選項。
  \item Editions:也可手動建立圖片/影片 NFT,並嵌入文章中售賣,形成內容+藏品閉環。
\end{itemize}
\end{HowBox}

\begin{HowBox}{眾籌 / 競拍 / 分潤 / 投票(文本流程要點)}
\begin{itemize}
  \item \textbf{眾籌(Crowdfunds)}:支持者存入 ETH 資助項目,換取項目通證;眾籌可嵌入任意條目。
  \item \textbf{競拍(Auctions)}:需先在 \textbf{Zora / Foundation / Rarible / SuperRare} 或自定合約鑄 NFT;
        設最低價,首次出價高於最低價後競拍才開始;每 15 分鐘更新一次,無更高出價則結束(文本描述)。
  \item \textbf{分享收益(Splits)}:可把 NFT 銷售或眾籌收益分配給不同貢獻者。
  \item \textbf{投票(Token Race)}:有門檻;需成為正式會員並持有官方通證 \textbf{WRITE}(發放稀少,文本提及)。
\end{itemize}
\end{HowBox}

% ---------------- 8) GameFi ----------------
\section{遊戲類 DApps(GameFi)與案例:Alien Worlds}

\begin{DefBox}{GameFi 是什麼?}
\textbf{GameFi} 把「遊戲 + NFT + DeFi」結合:鏈上開發遊戲,角色/道具做成 NFT,遊戲內資產可交易,玩家可 \textbf{Play-to-Earn}(邊玩邊賺)。
\end{DefBox}

\begin{WarnBox}{現況限制(文本要點)}
\begin{itemize}
  \item 底層設施仍在發展:運算速度/可擴展性有限,\textbf{娛樂性與體驗多不如 Web2 遊戲}。
  \item 許多玩家主要為 \textbf{to-Earn} 而來,部分項目偏重賣道具而忽視可玩性(文本觀點)。
  \item 若要長期運營:需要提供情緒價值,讓用戶願意為娛樂付費,而不只為炒作道具。
\end{itemize}
\end{WarnBox}

\subsection{Alien Worlds(異形世界)}
\begin{DefBox}{是什麼}
\textbf{Alien Worlds} 是基於 \textbf{WAX 公鏈}的太空探索遊戲:共有 \textbf{6 個星球},每個星球都是一個 DAO。玩家可透過挖礦、任務與戰鬥等方式參與生態。
\end{DefBox}

\begin{KeyBox}{核心機制(文本例子)}
\begin{itemize}
  \item 玩家透過\textbf{挖礦}賺取通證 \textbf{Trilium(TLM)};每次挖礦有機會挖到 NFT。
  \item 進入遊戲先獲得免費挖礦工具;也可參選星球理事候選人,當選後參與治理(文本描述)。
  \item NFT 卡牌可推出更多玩法(如「閃耀」升級、格鬥作戰等)。
\end{itemize}
\end{KeyBox}

\begin{HowBox}{兩種資產形態:NFT + 同質化通證 TLM}
\begin{itemize}
  \item \textbf{NFT}:遊戲道具/資產(非同質化)。
  \item \textbf{TLM(ERC-20 同質化通證)}:扮演遊戲「金幣」角色。
  \item TLM 用途:挖礦獲得;可\textbf{質押到星球 DAO} 參與管理決策;也用於購買/升級道具、參與任務與活動(文本描述)。
\end{itemize}
\end{HowBox}

% ---------------- 9) Other sectors ----------------
\section{其他領域:能源/碳中和、交易所}

\subsection{能源 / 碳中和:Dovu}
\begin{DefBox}{是什麼}
\textbf{Dovu} 將\textbf{碳信用}通證化,並提供基於\textbf{哈希圖(Hashgraph)}的碳交易市場,使人們能更即時地抵消自身碳排放(文本描述)。
\end{DefBox}

\begin{KeyBox}{怎麼用(需求側/供給側)}
\begin{itemize}
  \item \textbf{需求側}:提供交易通證 \textbf{DOV} 與碳計算器。
  \item \textbf{供給側}:與各地農場合作,透過植樹造林等方式降低碳排放(文本描述)。
\end{itemize}
\end{KeyBox}

\subsection{交易所:中心化 vs 去中心化}
\begin{DefBox}{基本分類}
加密資產交易所包含\textbf{中心化交易所(CEX)}與\textbf{去中心化交易所(DEX)}。文本指出中心化交易所仍佔主導,原因之一是發展較早、對 Web2 使用者更友善。
\end{DefBox}

\begin{KeyBox}{交易標的(文本提及)}
\begin{itemize}
  \item 主要交易:\textbf{加密貨幣}與 \textbf{NFT}。
\end{itemize}
\end{KeyBox}

% ---------------- Review ----------------
\section{快速複習題(自測用)}

\begin{KeyBox}{Q1:IPFS 與 HTTP 在「找檔案」方式上的差別?}
A:HTTP 先找伺服器位置再用路徑找檔案;IPFS 用\textbf{內容哈希}直接在分布式網路中定位資料(常用 DHT)。
\end{KeyBox}

\begin{KeyBox}{Q2:Storj 上傳/下載的關鍵流程是什麼?}
A:上傳時\textbf{加密 + 分片 + 分發到節點};下載時依定位資訊取回碎片並在本機\textbf{重組}。
\end{KeyBox}

\begin{KeyBox}{Q3:Aave 的 aToken 用來做什麼?}
A:作為存款憑證;存入資產得到對應 aToken(如 aETH),贖回時交還 aToken 取回資產。
\end{KeyBox}

\begin{KeyBox}{Q4:Mirror 內容變現的兩條常見路徑?}
A:文章可直接鑄成 NFT 售賣;或用 Crowdfunds 眾籌(支持者存 ETH 換項目通證),也可 Auctions 競拍與 Splits 分潤。
\end{KeyBox}

\vspace{8pt}
\begin{center}
\textcolor{muted}{\small 提醒:若 PDF 出現像 ``borderline west'' 這種字,代表它被你打進正文;LaTeX 註解要用 \%(不是 //)。}
\end{center}

\end{document}
