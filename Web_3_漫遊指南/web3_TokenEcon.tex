% !TEX program = xelatex
\documentclass[11pt,a4paper]{article}

% =========================
% style.tex  (ALL STYLES)
% =========================

% ---------- Page & Fonts ----------
\usepackage[a4paper,margin=1in]{geometry}
\usepackage{fontspec}
\usepackage{xeCJK}

% CJK font fallbacks (Overleaf 常見可用 Noto)
\IfFontExistsTF{Noto Serif CJK TC}{
  \setCJKmainfont{Noto Serif CJK TC}
}{
  \IfFontExistsTF{Noto Sans CJK TC}{
    \setCJKmainfont{Noto Sans CJK TC}
  }{
    \setCJKmainfont{SimSun}
  }
}
\setmainfont{TeX Gyre Termes}

% ---------- Packages ----------
\usepackage{hyperref}
\usepackage{microtype}
\usepackage{booktabs}
\usepackage{tabularx}
\usepackage{enumitem}
\usepackage{tikz}
\usetikzlibrary{positioning}
\usepackage{xcolor}
\usepackage{tcolorbox}
\tcbuselibrary{skins,breakable}
\usepackage{titlesec}
\usepackage{fancyhdr}

% ---------- Color Palette ----------
\definecolor{ink}{HTML}{1F2937}
\definecolor{muted}{HTML}{6B7280}
\definecolor{brand}{HTML}{2563EB}
\definecolor{brand2}{HTML}{0EA5E9}
\definecolor{soft}{HTML}{EFF6FF}
\definecolor{warn}{HTML}{F59E0B}
\definecolor{good}{HTML}{10B981}
\definecolor{rose}{HTML}{F43F5E}
\definecolor{paper}{HTML}{FFFFFF}

% ---------- Hyperref ----------
\hypersetup{
  colorlinks=true,
  linkcolor=brand,
  urlcolor=brand2,
  citecolor=brand
}

% ---------- Global typography ----------
\setlength{\parindent}{0pt}
\setlength{\parskip}{6pt}

% Clean lists inside boxes
\setlist[itemize]{itemsep=4pt, topsep=4pt, leftmargin=1.4em}
\setlist[enumerate]{itemsep=4pt, topsep=4pt, leftmargin=1.6em}

% ---------- Header ----------
\pagestyle{fancy}
\fancyhf{}
\lhead{\textcolor{muted}{Web3 / DApps 筆記}}
\rhead{\textcolor{muted}{\thepage}}
\renewcommand{\headrulewidth}{0pt}

% ---------- Section style ----------
\titleformat{\section}{\Large\bfseries\color{ink}}{\thesection}{0.8em}{}
\titleformat{\subsection}{\large\bfseries\color{ink}}{\thesubsection}{0.8em}{}
\titleformat{\subsubsection}{\normalsize\bfseries\color{ink}}{\thesubsubsection}{0.8em}{}
\titlespacing*{\section}{0pt}{1.0em}{0.4em}
\titlespacing*{\subsection}{0pt}{0.8em}{0.3em}
\titlespacing*{\subsubsection}{0pt}{0.5em}{0.2em}

% ---------- tcolorbox base ----------
\tcbset{
  enhanced,
  breakable,
  colback=paper,
  colframe=brand!18,
  boxrule=0.5pt,
  arc=10pt,
  left=10pt,right=10pt,top=10pt,bottom=10pt,
  before skip=10pt, after skip=10pt
}

% ---------- Box components ----------
\newtcolorbox{KeyBox}[1]{
  colback=paper,
  colframe=brand!18,
  borderline west={4pt}{0pt}{brand},
  title=\textbf{#1},
  coltitle=ink,
  fonttitle=\bfseries,
  colbacktitle=paper,
  boxed title style={boxrule=0pt, colframe=paper}
}

\newtcolorbox{DefBox}[1]{
  colback=paper,
  colframe=brand2!18,
  borderline west={4pt}{0pt}{brand2},
  title=\textbf{#1},
  coltitle=ink,
  fonttitle=\bfseries,
  colbacktitle=paper,
  boxed title style={boxrule=0pt, colframe=paper}
}

\newtcolorbox{HowBox}[1]{
  colback=paper,
  colframe=good!18,
  borderline west={4pt}{0pt}{good},
  title=\textbf{#1},
  coltitle=ink,
  fonttitle=\bfseries,
  colbacktitle=paper,
  boxed title style={boxrule=0pt, colframe=paper}
}

\newtcolorbox{WarnBox}[1]{
  colback=paper,
  colframe=warn!22,
  borderline west={4pt}{0pt}{warn},
  title=\textbf{#1},
  coltitle=ink,
  fonttitle=\bfseries,
  colbacktitle=paper,
  boxed title style={boxrule=0pt, colframe=paper}
}

% ---------- Tag pill ----------
\newcommand{\tagpill}[1]{%
  \tikz[baseline=(X.base)]\node[
    fill=soft, draw=brand!25, rounded corners=6pt,
    inner xsep=7pt, inner ysep=2.5pt
  ](X){\small\textcolor{brand}{\textbf{#1}}};%
}

\renewcommand{\arraystretch}{1.18}

\begin{document}

% ---------- Title ----------
\begin{center}
  {\LARGE \textbf{通证经济(Token Economics / Tokenomics)重點筆記}}\\[6pt]
  {\large \textcolor{muted}{只保留:是什麼|用途|用在哪|怎麼 work|文中例子}}\\[10pt]
  \tagpill{Web3}\quad \tagpill{Tokenomics}\quad \tagpill{Supply}\quad \tagpill{Demand}\quad
  \tagpill{Distribution}\quad \tagpill{DAO}\quad \tagpill{GameFi}\quad \tagpill{X-to-Earn}
\end{center}

\begin{KeyBox}{快速總結(考前 30 秒)}
\begin{itemize}
  \item \textbf{一句話}:通证经济(Tokenomics)就是 Web3 项目里「通证如何\textbf{供给}、如何被\textbf{需求}、如何\textbf{分配激励}」的整套规则;设计好坏直接决定项目能否形成\textbf{正循环}。
  \item \textbf{三根支柱}:\textbf{供给}(上限/已生成/净增速/销毁)+\textbf{需求}(实用/收益/投机/收藏等)+\textbf{分配}(激励从哪来、何时发、激励什么行为)。
  \item \textbf{核心规律}:供不应求价格涨、供过于求价格跌;但 Web3 特别强调\textbf{分配机制},因为它决定早期冷启动与长期留存。
  \item \textbf{高频风险}:欺诈跑路与失败项目很多;供给失控导致贬值、巨鲸撤资导致死亡螺旋;锚定币/中心化承诺存在信任与监管风险;安全事件会触发崩盘式外流。
\end{itemize}
\end{KeyBox}

\tableofcontents

% =========================================================
\section{通证经济是什麼?(定位与定义)}

\begin{DefBox}{一句话定义}
\textbf{通证经济(Token Economics / Tokenomics)}是 Web3 项目的「经济体系设计」:围绕通证的\textbf{发放、分配、流通与使用}机制,系统性安排\textbf{供给、需求、分配}三大要素,使项目能够\textbf{吸引参与者}并形成可持续的\textbf{正反馈}。
\end{DefBox}

\begin{KeyBox}{为什么说通证是 Web3 的核心?(文中要点)}
\begin{itemize}
  \item 通证是 Web3 的\textbf{激励工具}:奖励用户参与,是驱动系统运行的「燃油」。
  \item 通证是 Web3 的\textbf{连接器}:连接生态中不同项目/场景(身份、NFT、交易媒介都可以是通证)。
  \item 成败关键在\textbf{机制}:通证的\textbf{发放、分配、流通使用}机制,决定项目能否跑起来、跑得久。
\end{itemize}
\end{KeyBox}

\begin{WarnBox}{行业现实(文中提醒)}
截至 2022 年初市场上同质化通证已超过 6000 种,NFT 数量更庞大;仅 2021 年 Web3 参与者在欺诈跑路项目中损失高达 120 亿美元(不含大量非恶意失败项目)。学习通证经济学的价值在于:\textbf{更好判断项目价值与风险},参与时「擦亮眼睛」。
\end{WarnBox}

% =========================================================
\section{用途(通证经济能解决什么问题?)}

\begin{KeyBox}{用途(回答:Tokenomics 用来做什么)}
\begin{itemize}
  \item \textbf{冷启动拉新}:在产品体验尚不完善、网络效应未形成时,用财务激励吸引用户使用与贡献。
  \item \textbf{协调多方参与者}:在投资人、团队、社区成员、生态供应商之间做激励与约束,避免失衡。
  \item \textbf{维持系统运转}:通过供给/消耗/收入/金库机制,让经济体系长期可持续。
  \item \textbf{价格与门槛管理}:通过市场供需与金库操作,影响通证价格,从而影响进入门槛与参与结构(文中在 DAO 里强调)。
\end{itemize}
\end{KeyBox}

% =========================================================
\section{通证经济三大支柱:供给|需求|分配}

% -------------------------
\subsection{供给(Supply):上限|已生成|净增速|销毁}

\begin{DefBox}{供给的 3 个维度(文中框架)}
任意时刻,一个通证经济体系的供给关键维度有 3 个:\textbf{通证数量上限}、\textbf{已生成通证数量}、\textbf{通证净增长速度}。\\
若设计\textbf{销毁机制},则:\textbf{净增长 = 生成通证 - 销毁通证}。
\end{DefBox}

\begin{KeyBox}{供给模型 4 类(背分类 + 例子就够)}
\begin{itemize}
  \item \textbf{紧缩型模型}:总量有上限或净增长为负(常依赖「需求上升」的隐含假设)。
  \item \textbf{膨胀型模型}:总量无上限且净增长为正(适合流通型/实用型,但会稀释贬值)。
  \item \textbf{双通证模型}:同一生态中两种通证,一紧缩(价值储藏/治理),一膨胀(流通/消耗),并由转换机制关联。
  \item \textbf{锚定模型}:供给锚定现实资产/商品价值,用于特定场景的稳定计价与交易便利。
\end{itemize}
\end{KeyBox}

\begin{HowBox}{供给模型怎么 work(抓住「优点/代价」)}
\begin{itemize}
  \item \textbf{紧缩型}:通过稀缺性维护价值,但可能让人更愿意\textbf{囤积}而非流通,降低生态活跃度。
  \item \textbf{膨胀型}:更利于流通与使用,但增发会导致\textbf{持续贬值压力}。
  \item \textbf{双通证}:让「价值储藏」与「流通使用」分工,分别匹配不同供给逻辑,降低单一通证承载过多矛盾。
  \item \textbf{锚定型}:通过 1:1 锚定实现稳定价格,但引入\textbf{中心化信任}与审计/监管风险。
\end{itemize}
\end{HowBox}

\begin{KeyBox}{文中例子:供给模型与参数}
\begin{itemize}
  \item \textbf{BTC(紧缩型)}:上限 2100 万枚;无销毁;产出速度周期性减半;已生成约 1900 万枚,剩余约 200 万枚将在 2140 年前后完成生成。文中观点:BTC 更像 Web3 的「黄金」,偏价值储藏,不强调生态内循环使用。
  \item \textbf{DOGE(膨胀型)}:总量无上限;已生成约 1300 亿枚;每年增发 50 亿枚;无销毁。
  \item \textbf{双通证(Axie Infinity:AXS + SLP)}:
    \begin{itemize}
      \item \textbf{AXS(治理/价值储藏倾向)}:总量上限 2.7 亿枚;用于参与治理投票、购买关键道具(Axies)。
      \item \textbf{SLP(流通/消耗倾向)}:总量无上限;通过对战/任务产出;繁殖等消耗带来销毁,但净增速总体为正。
    \end{itemize}
  \item \textbf{USDT(锚定型)}:发行方声称以美元现金等价物为基石,发行等价值 USDT,使 1 USDT 约等于 1 美元;供给随储备变化(文中提到约 800 亿美元量级)。
\end{itemize}
\end{KeyBox}

\begin{WarnBox}{供给侧高频风险点(文中)}
\begin{itemize}
  \item \textbf{紧缩型的副作用}:预期升值会强化囤积,抑制流通与生态活跃。
  \item \textbf{膨胀型的代价}:增发导致贬值,若缺乏真实需求承接,会伤害长期发展。
  \item \textbf{锚定型的信任风险}:缺乏监管时,发行方声明与审计报告未必真实准确,价格可能因质疑而波动。
\end{itemize}
\end{WarnBox}

% -------------------------
\subsection{需求(Demand):通证被「想要」的原因}

\begin{DefBox}{通证像商品也像资产(文中表述)}
通证是加密数字物品:有时更像\textbf{商品}(为了使用),有时更像\textbf{资产}(为了投资)。一种通证可同时兼具多种功能,从而扩大需求面。
\end{DefBox}

\begin{KeyBox}{需求类型 5 类(背定义 + 例子)}
\begin{itemize}
  \item \textbf{持有型实用}:持有即可获得持续权利/非物化效用(会员、投票治理等)。
  \item \textbf{消耗型实用}:使用功能必须\textbf{消耗}通证(常伴随销毁/费用机制)。
  \item \textbf{收益型增值}:持有/质押带来定期收益或分成(像理财/现金流)。
  \item \textbf{投机型增值}:通证本身用途弱,需求来自「未来有人更高价接盘」的信念与情绪。
  \item \textbf{收藏}:不升值也不可交易/转赠仍有人买,原因是艺术性与偏好。
\end{itemize}
\end{KeyBox}

\begin{KeyBox}{文中例子:需求类型对应项目}
\begin{itemize}
  \item \textbf{持有型实用:BAYC} —— 持有可成为私密俱乐部会员,并可能获得团队发放的 NFT 空投。
  \item \textbf{消耗型实用:ChainLink(LINK)} —— LINK 的用途是支付使用去中心化预言机服务的费用。
  \item \textbf{收益型增值:Curve Finance(CRV)} —— 质押 CRV 得 veCRV,可分享交易所交易佣金收入。
  \item \textbf{多重角色:ETH} —— 作为最大生态的通用通证,在不同项目中扮演不同功能角色。
  \item \textbf{收藏(不可二级交易仍有需求)} —— 文中提到部分国内数字藏品平台:购买后不可二次交易/转赠,但因艺术性强仍可能被秒空。
\end{itemize}
\end{KeyBox}

\begin{WarnBox}{投机型需求的关键点(文中提醒)}
投机型需求的驱动力是对未来的信念与圈层情绪:强但难量化。判断强弱往往需要进入圈子感受持有者热情;同时也意味着\textbf{波动与崩塌风险更高}。
\end{WarnBox}

% -------------------------
\subsection{分配(Distribution):激励从哪来|何时发|激励什么}

\begin{DefBox}{分配的核心目的(文中)}
通证分配的核心目的是\textbf{合理激励用户与早期团队},推动项目形成正循环。在早期产品未成熟时,通证的财务激励是吸引用户的重要因素。
\end{DefBox}

\begin{KeyBox}{一个 Web3 项目常见参与者(文中)}
\textbf{投资人}、\textbf{发起团队}、\textbf{社区成员}、\textbf{生态供应商}。\\
分配给谁、按什么节奏分配,会直接影响项目成败。
\end{KeyBox}

\begin{HowBox}{理解分配逻辑:先回答 3 个问题(文中)}
\begin{itemize}
  \item \textbf{激励通证从何而来?}
  \item \textbf{分配时间表(Vesting/Cliff)怎样?}
  \item \textbf{激励哪些行为?}
\end{itemize}
\end{HowBox}

\begin{KeyBox}{从何而来(文中两条路)}
\begin{itemize}
  \item \textbf{通证增发}:活动奖励、贡献者奖励、流动性质押奖励、空投等(总量增加)。
  \item \textbf{项目金库(Treasury)再分配}:通过智能合约对日常交易收小额费用进入金库,再从金库发放激励(不增加总量,只是再分配)。
\end{itemize}
\end{KeyBox}

\begin{HowBox}{金库机制怎么 work(文中关键判断)}
好的通证收集与发放机制应使:\textbf{发放激励通证}、\textbf{交易活跃度}、\textbf{收集到的通证}三者呈正比例波动。\\
交易活跃时应发更多激励;若需要发放的通证少于「可/应发放」量,可能导致社区信心崩塌、用户退出。\\
增发发放不易“发不出”,但\textbf{无节制增发}会透支增长空间、造成贬值;金库发放需要\textbf{量入为出}。
\end{HowBox}

\begin{KeyBox}{分配时间表:对用户更友好的 3 个特征(文中)}
\begin{itemize}
  \item \textbf{团队 Vesting 比用户更长}:团队更久才能拿完通证,利益与项目深度绑定。
  \item \textbf{团队 Cliff 晚于用户}:先让用户拿到应得部分,再让团队开始拿,降低“造势后消失”风险。
  \item \textbf{至少 50\% 通证分给社区}:激励社区在项目成熟后最终接手运营。
\end{itemize}
\end{KeyBox}

\begin{WarnBox}{被激励的行为会把项目推向何方(文中提醒)}
激励规则决定参与者行动方向。常见激励是\textbf{活跃度}(如半年内使用/交易次数、首尾间隔等)。\\
但当行业周期与情绪变化时,原本善意的激励设计也可能产生负面作用;文中以 \textbf{UST-LUNA 机制}作为“设计会失败”的典型提醒(细节在其他章节展开)。
\end{WarnBox}

% =========================================================
\section{DAO 的经济模型(文中框架)}

\begin{DefBox}{DAO 经济模型:一句话}
DAO 的通证经济模型围绕「\textbf{创世通证}+\textbf{金库}+\textbf{成员治理投票}+\textbf{产品/服务收入}」运转:成员用原生通证治理与贡献,DAO 通过产品/服务获得收入进入金库,再用于开支、激励或调控通证供需。
\end{DefBox}

\begin{HowBox}{DAO 基本流程怎么 work(按文中叙述)}
\begin{itemize}
  \item \textbf{建立时}创造\textbf{创世通证}:部分分发给创始成员,部分留在金库。
  \item \textbf{金库资产}可能同时包含:DAO 原生通证与 ETH 等通用通证(有时创始成员需用 ETH 购买创世通证)。
  \item \textbf{治理}:成员持原生通证投票,影响 DAO 决策与金库运营。
  \item \textbf{价值创造}:成员贡献力量做出产品;用户可来自 DAO 内或外。
  \item \textbf{收入}:外部用户付费(可用原生通证、ETH,甚至法币约定);收入进金库,用于 DAO 开支或在市场调控原生通证供需。
  \item \textbf{进入门槛}:通过管理原生通证价格影响加入门槛与成员结构。
\end{itemize}
\end{HowBox}

\begin{KeyBox}{DAO 正反馈的关键(文中结论)}
让模型形成正反馈的关键在于\textbf{为成员带来价值},包括:
\begin{itemize}
  \item \textbf{外部价值}:原生通证在交易市场的增值(可出售变现)。
  \item \textbf{内部价值}:成为 DAO 一员获得的归属感、知识等内在收益。
\end{itemize}
两者相互补充、也相互影响。
\end{KeyBox}

\begin{WarnBox}{DAO 的死亡螺旋(文中机制)}
若原生通证\textbf{流动性差},个别成员退出可能引发价格大幅波动,动摇信心;当价格剧烈波动且内部价值又低时,成员可能跟随抛售,贡献者减少、产品质量下降,最终进入\textbf{死亡螺旋}。\\
文中对 DAO 运营的提醒:不要只依赖新成员资金流入,要组织资本/资产/通证\textbf{持续流入生态},控制外流并创新激励机制以降低流失率。
\end{WarnBox}

% =========================================================
\section{GameFi:Game or Fi?(文中要点)}

\begin{DefBox}{GameFi 是什么?(文中定义)}
\textbf{GameFi} 是把 Web2 游戏本体与 Web3 经济模型结合的游戏:用户资产 DeFi 化,\textbf{游戏道具 NFT 化};用户参与游戏可获得奖励;代码可更公开透明,降低项目方作弊空间。
\end{DefBox}

\begin{KeyBox}{用在哪(GameFi 常见结构)}
\begin{itemize}
  \item \textbf{单通证模型}:产出场景获得通证、消耗场景失去通证(例:CryptoKitties;Axie 初期也曾是单通证后迭代)。
  \item \textbf{双通证模型}:母通证(治理)+ 子通证(经济/流通),母通证通常更难获得、价值更高。
  \item \textbf{NFT 定位两类}:
    \begin{itemize}
      \item 入场门票:必须先买 NFT 才能参与(现多见)。
      \item 工具/能力:NFT 越多/越稀有,胜率/技能越强(文中提到 Era7)。
    \end{itemize}
  \item \textbf{质押}:质押通证或 NFT 换取奖励以稳定经济系统(例:Fancy Birds)。
\end{itemize}
\end{KeyBox}

\begin{HowBox}{GameFi 的金融本质(文中一句话)}
GameFi 的经济模型,本质是\textbf{调控游戏内部通证的价格与数量变化}:重点在于平衡\textbf{产出场景}与\textbf{消耗场景},否则通胀/出逃会破坏系统。
\end{HowBox}

\subsection{文中案例:Axie Infinity(成长与衰落)}

\begin{KeyBox}{成长时间线(文中)}
\begin{itemize}
  \item 2017/12 开发开始;2018/02 预售筹集 900 ETH;2020/05 土地售罄筹集超 4600 ETH。
  \item 2021/02 Ronin 主网上线;2021/07 日活超 35 万、日交易量超 2500 万美元;2021/08 收入 3.6 亿美元达巅峰。
\end{itemize}
\end{KeyBox}

\begin{HowBox}{爆发原因(文中归纳)}
\begin{itemize}
  \item \textbf{技术}:迁移至 Ronin 侧链(为高性能与用户爆发铺路)。
  \item \textbf{市场}:头像热潮带来 NFT 新用户。
  \item \textbf{资本}:暴跌后资本寻找新增长点;GameFi 回本周期清晰、易被接受。
\end{itemize}
\end{HowBox}

\begin{HowBox}{经济模型(文中)}
\begin{itemize}
  \item \textbf{双通证}:SLP(子通证,用于繁殖等费用)+ AXS(母通证/治理 ERC-20,可玩游戏、质押、参与投票)。
  \item \textbf{准入}:需购买 Axie NFT 才能参与。
  \item \textbf{产出}:战斗胜利等获得通证奖励。
  \item \textbf{消耗}:哺育/繁衍、获取治理资格等;也可在二级市场出售 NFT 获取通证;质押也可得奖励。
\end{itemize}
\end{HowBox}

\begin{WarnBox}{死亡螺旋触发点(文中)}
当负面情绪出现、通证价值下降,玩家抛售出逃会破坏系统平衡。文中指出 2022/03 Ronin Network 遭大规模黑客攻击后,各项指标暴跌,Axie 难回巅峰规模。
\end{WarnBox}

\begin{KeyBox}{GameFi 改变了什么(文中)}
P2E 的关键差异不只是复杂度,而是\textbf{资产处置权}:运营商破产/关服时,玩家不必失去已获得资产。更投入的玩家也可能更忠诚、并参与治理。
\end{KeyBox}

% =========================================================
\section{X-to-Earn(做 X 就赚钱):增长模型与生命周期}

\begin{DefBox}{X-to-Earn 是什么?(文中定义)}
\textbf{X-to-Earn} 本质是:通过执行「X」这一动作获得通证,从而为用户带来收益。X 可以是任何动词(Move/Learn/Sing/Sleep 等)。它常用通证做冷启动,用市值膨胀吸引参与者投入劳动或资本。
\end{DefBox}

\begin{HowBox}{增长模型怎么 work(文中)}
\begin{itemize}
  \item \textbf{早期}:用通证激励吸引参与者(劳动/资本)。
  \item \textbf{规模形成后}:通过游戏化、收费服务或外部性经济活动实现盈利与留存。
  \item \textbf{对参与者}:形成新的要素市场,让资本/劳动力更高效地变现并获得回报。
\end{itemize}
\end{HowBox}

\begin{KeyBox}{项目生命周期 5 阶段(文中)}
\begin{itemize}
  \item \textbf{1 市值膨胀期}:吸引参与者;月活决定上限(文中提到:月活 100 万可称现象级)。
  \item \textbf{2 市值收缩期}:早期获利者离场;看是否有长期参与者留下、贡献值多大。
  \item \textbf{3 不确定期}:寻找真实需求;看生态产品如何留人。
  \item \textbf{4 第二增长曲线}:出现外部付费或外部性规模化带来爆发。
  \item \textbf{5 死亡螺旋}:另一种结局:未找到商业模式与正外部性,参与者持续流失。
\end{itemize}
\end{KeyBox}

\begin{WarnBox}{X-to-Earn 的结构性风险(文中)}
核心风险是:若用户获得的通证大量立刻离开体系进入市场,容易造成\textbf{通胀与价值走低};若无法找到真实商业模式承接需求,就可能在收缩期走向\textbf{死亡螺旋}。
\end{WarnBox}
% =========================================================
\section{X-to-Earn 的 6 种经济模型(按:频次 × 本金 × 劳动)}

\begin{DefBox}{核心分类方法(文中)}
理性的 X-to-Earn 参与者会考虑三件事:\textbf{本金投入}、\textbf{劳动投入}(体力/知识/决策/创作等)、\textbf{获得奖励的频次}。\\
把「频次(高/低)」「本金(高/低)」「劳动(高/低)」组合起来,可演化出多种经济模型;文中重点展开其中\textbf{6 种}(另外两种“低频次、低劳动”的模型文中表示暂不具现实意义,略)。
\end{DefBox}

% -------------------------
\subsection{1) 高频次|高本金|高劳动:资本 \& 机器人密集型}

\begin{KeyBox}{典型案例(文中)}
Bitcoin 与各类 Mining Networks;新生代包括 Chainlink、The Graph、Render Network 等。
\end{KeyBox}

\begin{HowBox}{怎么 work(结构特征)}
\begin{itemize}
  \item 高频奖励:产出与结算频繁(持续竞争、持续产出)。
  \item 高本金:参与需要大量资本投入(设备/算力/基础设施或等价资源)。
  \item 高劳动:劳动更多体现为\textbf{标准化计算/验证工作}与持续运营投入。
\end{itemize}
\end{HowBox}

\begin{KeyBox}{优点(文中)}
\begin{itemize}
  \item 网络沉淀的经济价值大,有机会形成 \textbf{Fat Protocol}。
\end{itemize}
\end{KeyBox}

\begin{WarnBox}{缺点与风险(文中)}
\begin{itemize}
  \item 网络经济体内卷:计算标准化、军备竞赛激烈。
  \item 资本竞争强:组织化机构林立,\textbf{散户参与度低}。
  \item 更适合 \textbf{B 端}产品而非 C 端大众产品。
\end{itemize}
\end{WarnBox}

\begin{HowBox}{设计建议(文中)}
\begin{itemize}
  \item 在验证工作的算法中引入更高的数学难度与随机性,避免大参与者的非对称优势无序增长,降低基尼系数。
  \item 不适合一开始就采用:门槛太高会挡住大量参与者,容易变成少数人控制的游戏;早期应\textbf{降低门槛},后续再逐步提高。
\end{itemize}
\end{HowBox}

% -------------------------
\subsection{2) 高频次|高本金|低劳动:资本密集型}

\begin{KeyBox}{典型案例(文中)}
POS 网络、Staking 平台、Liquidity Mining 网络、以资产证明为准入门槛的网络等。
\end{KeyBox}

\begin{KeyBox}{优点(文中)}
\begin{itemize}
  \item 参与者劳动成本低,资本效率被放大,是有效筹集资本与流动性的一类网络。
\end{itemize}
\end{KeyBox}

\begin{WarnBox}{缺点与风险(文中)}
\begin{itemize}
  \item 存在被巨鲸挟持的风险:收益不达预期时巨鲸撤资,网络价值下跌,进入\textbf{死亡螺旋}。
\end{itemize}
\end{WarnBox}

\begin{HowBox}{设计建议(文中)}
\begin{itemize}
  \item 引入更多参与者相互博弈,提高忠诚度(如用 POAP 做治理/参与奖励)。
  \item 设计长期价值锁定机制(例如引入投票权机制)。
  \item 长期方向:引入更多工作量与劳动是必然趋势。
\end{itemize}
\end{HowBox}

% -------------------------
\subsection{3) 高频次|低本金|高劳动:劳动密集型}

\begin{KeyBox}{典型案例(文中)}
Play-to-Earn、Move-to-Earn、Learn-to-Earn 等。
\end{KeyBox}

\begin{KeyBox}{优点(文中)}
\begin{itemize}
  \item 门槛低:不需要很有钱,有劳动付出即可换取奖励。
\end{itemize}
\end{KeyBox}

\begin{WarnBox}{缺点与风险(文中)}
\begin{itemize}
  \item 劳动不容易量化,需要找到背后的商业模式承接价值。
\end{itemize}
\end{WarnBox}

\begin{HowBox}{设计建议(文中)}
\begin{itemize}
  \item 引入智能硬件与预言机等\textbf{防作弊}技术。
  \item 价值创造两条路:
    \begin{itemize}
      \item 向内:设计更复杂、随机的游戏化商业生态。
      \item 向外:寻找外部性的经济价值。
    \end{itemize}
\end{itemize}
\end{HowBox}

% -------------------------
\subsection{4) 高频次|低本金|低劳动:流量巨大但价值较低}

\begin{KeyBox}{典型案例(文中)}
Sleep-to-Earn、Read-to-Earn 等。
\end{KeyBox}

\begin{KeyBox}{优点(文中)}
\begin{itemize}
  \item 门槛极低,受众巨大,容易获得大量参与者。
\end{itemize}
\end{KeyBox}

\begin{WarnBox}{缺点与风险(文中)}
\begin{itemize}
  \item 参与者画像不精准(谁都能完成简单动作),劳动价值与资本贡献都低,容易变成低价值网络。
\end{itemize}
\end{WarnBox}

\begin{HowBox}{设计建议(文中)}
\begin{itemize}
  \item 提高本金或劳动门槛,尽量找到明确的垂直场景,让简单劳动更有意义。
  \item 适当增加本金投入:让作恶者有成本,否则可罚没本金。
\end{itemize}
\end{HowBox}

% -------------------------
\subsection{5) 低频次|低本金|高劳动:技能密集型}

\begin{KeyBox}{典型案例(文中)}
Research-to-Earn、Code-to-Earn、Sing-to-Earn 等。
\end{KeyBox}

\begin{KeyBox}{优点(文中)}
\begin{itemize}
  \item 用户精准、劳动技术含量高,并且更可能存在商业模式。
\end{itemize}
\end{KeyBox}

\begin{WarnBox}{缺点与风险(文中)}
\begin{itemize}
  \item 参与者精英化、规模难做大;任务难量化且奖励结算困难。
\end{itemize}
\end{WarnBox}

\begin{HowBox}{设计建议(文中)}
\begin{itemize}
  \item 将复杂技能任务\textbf{拆分为更大众化的小任务}再组合,提高结算频次、简化动作。
  \item 文中举例:与其把股票分析师整体做成 Research-to-Earn,不如拆成负责收集信息的 Read-to-Earn + 给出意见的 Comment-to-Earn。
\end{itemize}
\end{HowBox}

% -------------------------
\subsection{6) 低频次|高本金|高劳动:接近 Venture DAO(较少见)}

\begin{KeyBox}{文中定位}
这种模型不常见,比较接近 \textbf{Venture DAO}:参与者共同出钱、共同做投资策略、一起分红。
\end{KeyBox}

\begin{WarnBox}{难点(文中)}
\begin{itemize}
  \item 确认与激励频次低;工作量难衡量;任务极度不标准化,难以达成共识,因此难形成规模。
\end{itemize}
\end{WarnBox}

\begin{WarnBox}{文中说明:为何只讲 6 种?}
文中提到另外两种\textbf{低频次、低劳动}模型:参与者既不额外付出劳动、奖励频次也低,暂不具备符合 Web3 精神的现实意义,且几乎没有现实案例,因此不展开。
\end{WarnBox}

% -------------------------
\subsection{文中例子:Move-to-Earn(STEPN)}

\begin{DefBox}{STEPN(Move-to-Earn)}
Move-to-Earn:通过运动获取收益。文中以 STEPN 为代表项目,强调其用「运动 + 奖励」吸引大量 Web2 用户进入。
\end{DefBox}

\begin{KeyBox}{通证与 NFT 结构(文中)}
\begin{itemize}
  \item \textbf{双通证}:经济通证 \textbf{GST} 与治理通证 \textbf{GMT}(鞋 1--29 级主要获得 GST,升到 30 级后可获得 GMT)。
  \item \textbf{准入}:至少购买一双\textbf{鞋子 NFT}才有 Earn 资格。
  \item \textbf{鞋子类型}:步行者/慢跑者/速跑者/训练者(对应最佳配速范围,影响收益)。
  \item \textbf{属性}:效率(GST 产出)、运气(掉落概率)、舒适度(GMT 效率)、耐久度(磨损)。
  \item \textbf{供给起点}:市场上鞋子 NFT 由最初 10,000 双创世鞋“铸造/创造”而来;质量分级:普通/罕见/稀有/史诗/传奇。
  \item \textbf{能量机制}:运动消耗能量;能量耗尽无法获得奖励;能量按固定节奏恢复;鞋子数量增加会提高每日能量上限。
\end{itemize}
\end{KeyBox}

\begin{HowBox}{产出与消耗场景(文中)}
\begin{itemize}
  \item \textbf{产出}:运动是主要产出场景(规划个人/马拉松/后台模式等),还有租赁系统(出租/承租 NFT 分成收益)。
  \item \textbf{消耗}:修复(用 GST 保持高效产出)、升级(GST+GMT 提升属性)、创造/铸造(GST+GMT 产出鞋盒)。
  \item \textbf{设计目的}:相比产出更强调消耗场景,让收益尽量留在体系中,维持正循环,避免通胀。
\end{itemize}
\end{HowBox}

\begin{KeyBox}{文中总结的“吸引力”}
用户既能锻炼身体又能获得收益;若用户量增长,还可能带来碳中和等外部性效应(文中表述为“一举多得”)。
\end{KeyBox}

% -------------------------
\subsection{文中例子:Learn-to-Earn(Let me speak)}

\begin{DefBox}{Let me speak(Learn-to-Earn)}
Learn-to-Earn:通过学习获得通证收益。Let me speak 的核心是:购买角色 NFT(类似“英语老师”)获得 Earn 资格,通过学习获得奖励。
\end{DefBox}

\begin{KeyBox}{通证与 NFT(文中)}
\begin{itemize}
  \item \textbf{双通证}:治理通证 \textbf{LMS} 与经济通证 \textbf{LSTARS}(文中提到截至 2022/05 LMS 暂未发行)。
  \item \textbf{角色 NFT 分级}:普通/罕见/稀有/史诗/传奇;质量不同带来属性差异。
  \item \textbf{关键参数(文中解释)}:天赋(学习赚钱效率)、回报率(不同等级收益水平)、签证长度(有效期;到期需重新购买或铸造;可延长次数与天数在文中给出)。
\end{itemize}
\end{KeyBox}

\begin{HowBox}{产出与消耗场景(文中)}
\begin{itemize}
  \item \textbf{产出}:学习获得 LSTAR;NFT 可租赁(出租/承租分成)。
  \item \textbf{消耗}:消耗 LSTAR 铸造新角色 NFT、延长签证;未来 LMS 作为升级与治理投票等消耗/用途。
\end{itemize}
\end{HowBox}

\begin{WarnBox}{前提条件(文中强调)}
该模式的正向作用(促进学习动力与乐趣)建立在通证经济体系\textbf{健康}的前提下;否则用户信心不足,参与意愿下降。
\end{WarnBox}

% =========================================================
\section{更多 Web3 领域的例子(文中提到的用法)}

\subsection{SocialFi(社交化金融)}

\begin{DefBox}{SocialFi 是什么?(文中定义)}
\textbf{SocialFi} = Social + Finance:在区块链上结合社交与金融,针对 Web2 平台控制用户数据、用户收益少的问题,强调把\textbf{数据所有权归还用户}。
\end{DefBox}

\begin{KeyBox}{文中例子:CyberConnect(去中心化社交图谱)}
\begin{itemize}
  \item 目标:让用户在不同 Web3 平台间无缝迁移社交数据(关注关系/粉丝)。
  \item 形态:网页 App,可用 MetaMask 连接;功能包括关注、检索关注者/追随者名单、推荐关注者。
  \item 文中备注:暂未发行通证,但可观察到行业里 Personal Tokens / Community Tokens / Social Platform Tokens 等分类。
\end{itemize}
\end{KeyBox}

\subsection{NFTFi(NFT + Finance)}

\begin{DefBox}{NFTFi 在解决什么?(文中逻辑)}
当 NFT 被大量囤积却缺乏应用场景时,NFTFi 尝试把 NFT 与金融结合,提供\textbf{额外使用场景与增值空间}(如抵押借贷、租赁等)。
\end{DefBox}

\begin{HowBox}{文中例子:BendDAO(NFT 抵押借贷)}
\begin{itemize}
  \item 机制:把蓝筹 NFT 抵押进 NFT 池,转换为 boundNFT,并按抵押率即时借出 ETH(随借随还,无需撮合期限)。
  \item 风控:NFT 地板价波动大;文中提到触及清算线后有 48 小时保护期,期间还款可避免被清算。
  \item 供需:出借人提供 ETH 流动性赚利息;借款人用 NFT 抵押即时借 ETH,形成供需关系。
\end{itemize}
\end{HowBox}

\begin{KeyBox}{NFT 租赁(文中提到)}
文中提到可租赁 NFT 的提案 \textbf{ERC-4907} 已通过,但是否大规模应用仍未知。
\end{KeyBox}

\subsection{更多案例(文中“破圈/营销/新通证形态”)}

\begin{KeyBox}{Netflix《爱、死亡与机器人》NFT 寻宝(文中)}
\begin{itemize}
  \item 玩法:在剧集画面/广告牌/社交媒体中随机出现线索(如二维码),扫码可在 OpenSea 购买对应主题 NFT。
  \item 特征:无限增发,更多作为收藏品;对 Netflix 来说既有铸造收入,也能提升话题热度(营销破圈)。
\end{itemize}
\end{KeyBox}

\begin{KeyBox}{Web2 的 X-to-Earn 类比:蚂蚁森林(文中)}
\begin{itemize}
  \item 用户低碳行为积攒“绿色能量”(文中把它类比为一种通证)。
  \item 将“绿色能量”兑换为种树等公益行动;公园门票/露营收入可返还给贡献者(文中描述其形成一个小型经济闭环)。
\end{itemize}
\end{KeyBox}

\begin{DefBox}{SBT(灵魂绑定通证,文中提到的概念)}
文中引用维塔利克提出的 \textbf{SBT(Soulbound Token)}:个体可发行主题 SBT 发送给朋友;接收者\textbf{无法转发},从而形成链上关系网络,可用于\textbf{去中心化社会}与\textbf{身份自证},并被认为在金融应用上有广阔想象空间。
\end{DefBox}

% =========================================================
\section{最后的考点清单(自测用)}

\begin{KeyBox}{Q1:通证经济(Tokenomics)三大支柱是什么?}
A:\textbf{供给}(上限/已生成/净增速/销毁)+\textbf{需求}(实用/收益/投机/收藏等)+\textbf{分配}(激励从哪来、何时发、激励什么行为)。
\end{KeyBox}

\begin{KeyBox}{Q2:紧缩型 vs 膨胀型供给模型的核心优劣?}
A:紧缩型通过稀缺性维护价值,但易囤积、抑制流通;膨胀型促进流通使用,但增发带来贬值压力。
\end{KeyBox}

\begin{KeyBox}{Q3:为什么分配时间表很重要?文中给了哪些“对用户友好”的特征?}
A:时间表体现团队信心与绑定程度。对用户友好的特征:\textbf{团队 vesting 更长}、\textbf{团队 cliff 更晚}、\textbf{至少 50\% 通证分配给社区}。
\end{KeyBox}

\begin{WarnBox}{最后一句总结(照背)}
通证经济是 Web3 的发动机:用供给/需求/分配把激励、使用与价值创造连成闭环;但设计一旦失衡(增发失控、需求虚弱、分配不公或安全事件),就可能引发信心崩塌与\textbf{死亡螺旋}。
\end{WarnBox}

\end{document}
