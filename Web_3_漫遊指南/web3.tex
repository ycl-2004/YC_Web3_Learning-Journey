\documentclass[11pt,a4paper]{article}

% =========================
% style.tex  (ALL STYLES)
% =========================

% ---------- Page & Fonts ----------
\usepackage[a4paper,margin=1in]{geometry}
\usepackage{fontspec}
\usepackage{xeCJK}

% CJK font fallbacks (Overleaf 常見可用 Noto)
\IfFontExistsTF{Noto Serif CJK TC}{
  \setCJKmainfont{Noto Serif CJK TC}
}{
  \IfFontExistsTF{Noto Sans CJK TC}{
    \setCJKmainfont{Noto Sans CJK TC}
  }{
    \setCJKmainfont{SimSun}
  }
}
\setmainfont{TeX Gyre Termes}

% ---------- Packages ----------
\usepackage{hyperref}
\usepackage{microtype}
\usepackage{booktabs}
\usepackage{tabularx}
\usepackage{enumitem}
\usepackage{tikz}
\usetikzlibrary{positioning}
\usepackage{xcolor}
\usepackage{tcolorbox}
\tcbuselibrary{skins,breakable}
\usepackage{titlesec}
\usepackage{fancyhdr}

% ---------- Color Palette ----------
\definecolor{ink}{HTML}{1F2937}
\definecolor{muted}{HTML}{6B7280}
\definecolor{brand}{HTML}{2563EB}
\definecolor{brand2}{HTML}{0EA5E9}
\definecolor{soft}{HTML}{EFF6FF}
\definecolor{warn}{HTML}{F59E0B}
\definecolor{good}{HTML}{10B981}
\definecolor{rose}{HTML}{F43F5E}
\definecolor{paper}{HTML}{FFFFFF}

% ---------- Hyperref ----------
\hypersetup{
  colorlinks=true,
  linkcolor=brand,
  urlcolor=brand2,
  citecolor=brand
}

% ---------- Global typography ----------
\setlength{\parindent}{0pt}
\setlength{\parskip}{6pt}

% Clean lists inside boxes
\setlist[itemize]{itemsep=4pt, topsep=4pt, leftmargin=1.4em}
\setlist[enumerate]{itemsep=4pt, topsep=4pt, leftmargin=1.6em}

% ---------- Header ----------
\pagestyle{fancy}
\fancyhf{}
\lhead{\textcolor{muted}{Web3 / DApps 筆記}}
\rhead{\textcolor{muted}{\thepage}}
\renewcommand{\headrulewidth}{0pt}

% ---------- Section style ----------
\titleformat{\section}{\Large\bfseries\color{ink}}{\thesection}{0.8em}{}
\titleformat{\subsection}{\large\bfseries\color{ink}}{\thesubsection}{0.8em}{}
\titleformat{\subsubsection}{\normalsize\bfseries\color{ink}}{\thesubsubsection}{0.8em}{}
\titlespacing*{\section}{0pt}{1.0em}{0.4em}
\titlespacing*{\subsection}{0pt}{0.8em}{0.3em}
\titlespacing*{\subsubsection}{0pt}{0.5em}{0.2em}

% ---------- tcolorbox base ----------
\tcbset{
  enhanced,
  breakable,
  colback=paper,
  colframe=brand!18,
  boxrule=0.5pt,
  arc=10pt,
  left=10pt,right=10pt,top=10pt,bottom=10pt,
  before skip=10pt, after skip=10pt
}

% ---------- Box components ----------
\newtcolorbox{KeyBox}[1]{
  colback=paper,
  colframe=brand!18,
  borderline west={4pt}{0pt}{brand},
  title=\textbf{#1},
  coltitle=ink,
  fonttitle=\bfseries,
  colbacktitle=paper,
  boxed title style={boxrule=0pt, colframe=paper}
}

\newtcolorbox{DefBox}[1]{
  colback=paper,
  colframe=brand2!18,
  borderline west={4pt}{0pt}{brand2},
  title=\textbf{#1},
  coltitle=ink,
  fonttitle=\bfseries,
  colbacktitle=paper,
  boxed title style={boxrule=0pt, colframe=paper}
}

\newtcolorbox{HowBox}[1]{
  colback=paper,
  colframe=good!18,
  borderline west={4pt}{0pt}{good},
  title=\textbf{#1},
  coltitle=ink,
  fonttitle=\bfseries,
  colbacktitle=paper,
  boxed title style={boxrule=0pt, colframe=paper}
}

\newtcolorbox{WarnBox}[1]{
  colback=paper,
  colframe=warn!22,
  borderline west={4pt}{0pt}{warn},
  title=\textbf{#1},
  coltitle=ink,
  fonttitle=\bfseries,
  colbacktitle=paper,
  boxed title style={boxrule=0pt, colframe=paper}
}

% ---------- Tag pill ----------
\newcommand{\tagpill}[1]{%
  \tikz[baseline=(X.base)]\node[
    fill=soft, draw=brand!25, rounded corners=6pt,
    inner xsep=7pt, inner ysep=2.5pt
  ](X){\small\textcolor{brand}{\textbf{#1}}};%
}


% ---------------- Document ----------------
\begin{document}

% ========== 建议:把这一段放在 main.tex 对应章节位置 ==========
\section{Web 3.0 与元宇宙:概念区分、机制与基础设施}

\begin{KeyBox}
\textbf{本节阅读导图:}
\begin{itemize}
  \item 先分清三个概念:\textbf{区块链(底层技术)}、\textbf{Web 3.0(关系与确权)}、\textbf{元宇宙(宏大数字世界系统)}
  \item 再看演进逻辑:Web 2.0 的矛盾 $\rightarrow$ 区块链带来的确权与协作 $\rightarrow$ Web 3.0 的组织与激励(DAO/通证)
  \item 最后解释互补:\textbf{元宇宙偏前端体验},\textbf{Web 3.0 偏后端制度与基础设施}
\end{itemize}
\end{KeyBox}

\subsection{一、三个概念:是什么?怎么区分?}

\subsubsection{1) 定义(What it is)}
\begin{DefBox}
\textbf{区块链(Blockchain)是什么:}
一种为了解决\textbf{中心化服务器}带来的信任与控制问题而出现的底层技术(文中提到 2008 年正式诞生)。通过\textbf{分布式账本/数据库}与\textbf{加密协议}实现点对点记录与验证。

\medskip
\textbf{Web 3.0 是什么:}
以\textbf{“用户能拥有互联网”}为核心的互联网演进方向:用户不只是使用者,还能通过\textbf{通证(token)/智能合约}对数字资产与价值分配拥有更强控制;组织形态上以\textbf{DAO}作为“企业替代物/社区所有”的载体。

\medskip
\textbf{元宇宙(Metaverse)是什么:}
一个更宏大的\textbf{社会经济系统级}数字世界设想。大众常把它理解为体验升级(VR/AR、数字人等),但文中强调它不只是“大型游戏”,还需要能支撑“宇宙”的\textbf{真实性}与基础设施。
\end{DefBox}

\subsubsection{2) 用途与主线矛盾(Why it matters)}
\begin{KeyBox}
\textbf{用途/解决什么问题(文中主线):}
\begin{itemize}
  \item \textbf{Web 2.0 的矛盾:}平台巨头掌握大量用户数据与规则解释权;用户资产不受控,可能遭遇“杀熟”、封号等(用户被当作“流量”,平台沿用 AARRR 与 S-R 模式)。
  \item \textbf{区块链为 Web 2.0 提供“升级方向”:}用去中心化与确权机制,让用户能在平台中\textbf{拥有资产/分享价值},从而把 Web 2.0 推向 Web 3.0。
  \item \textbf{Web 3.0 对组织与激励的改变:}从“管理层做、用户用”,走向\textbf{社区所有、成员共同受益}的 DAO。
  \item \textbf{元宇宙与 Web 3.0 的互补:}元宇宙更偏\textbf{前端体验},Web 3.0 更偏\textbf{后端制度与基础设施};开放、互操作的元宇宙需要 Web 3.0 的底座。
\end{itemize}
\end{KeyBox}

\subsection{二、Web 2.0 $\rightarrow$ Web 3.0:机制与激励怎么变?}

\subsubsection{1) 机制对比(How it works)}
\begin{HowBox}
\textbf{Web 2.0(文中描述的典型机制):}
\begin{itemize}
  \item 平台围绕 AARRR:获客 $\rightarrow$ 激活 $\rightarrow$ 留存 $\rightarrow$ 推荐 $\rightarrow$ 变现;
  \item 用户被当作“资产/流量”,数据与规则由平台集中控制与变现;
  \item 终局叙事偏 IPO/并购退出:收益与用户关联弱,用户难以分享增长红利。
\end{itemize}

\medskip
\textbf{Web 3.0(文中描述的典型机制):}
\begin{itemize}
  \item \textbf{通证 + 智能合约:}持有通证使用服务;通证也是\textbf{收益分配凭证},使“使用者”进入“所有者社群”。
  \item \textbf{DAO(社区所有):}所有者是社区成员;成员共享利益,用户与产品利益绑定更强。
  \item \textbf{IEO vs. IPO:}IEO 门槛更低、无需中心化审核;通证在上所前也可能已公开、去中心化并可链上交易,从而获得更大流动性。
  \item \textbf{用户为何能成为所有者:}使用产品完成\textbf{测试}与\textbf{推广}两项工作;推广成本本是 Web 2.0 的大头之一,因此收益分配给使用者具备经济直觉。
\end{itemize}
\end{HowBox}

\subsubsection{2) 风险与现实约束(Risks \& Limits)}
\begin{WarnBox}
\textbf{风险/注意(文中明确或隐含的点):}
\begin{itemize}
  \item \textbf{风险与收益并存:}若产品失败,通证/社群财富可能随之清零。
  \item \textbf{竞争更激烈:}利益绑定带来黏性与稳固,也可能导致更急迫的“跑马圈地”。
  \item \textbf{元宇宙落地难:}要成为“宇宙”,需要极复杂的基础设施与生态系统,短时间难以实现;目前更接近的是 Web 3.0。
  \item \textbf{推进不确定:}在巨头生态中重分配利益,Web 3.0 推动是否顺利仍待观察。
\end{itemize}
\end{WarnBox}

\subsection{三、元宇宙 $\leftrightarrow$ Web 3.0:为何说相辅相成?}

\subsubsection{1) 前端 vs. 后端:角色分工(Front-end vs. Back-end)}
\begin{HowBox}
\textbf{文中的分工框架:}
\begin{itemize}
  \item \textbf{元宇宙:}偏前端/表现层/应用层(交互体验、VR/AR、AI、IoT 等)。
  \item \textbf{Web 3.0:}偏中后端/技术层与制度层(去中心化数据库、加密技术、协议、确权与协作机制)。
\end{itemize}
\end{HowBox}

\subsubsection{2) 开放互操作所需模块(Infrastructure modules)}
\begin{HowBox}
\textbf{为了“开放且互操作”的元宇宙,文中点名的 Web 3.0 模块:}
\begin{itemize}
  \item \textbf{去中心化身份:}跨世界保持“核心身份”一致;资产与身份绑定并强调唯一性。
  \item \textbf{去中心化存储:}支撑开放性与抗停机;例:\textbf{Filecoin / IPFS}。
  \item \textbf{NFT:}现实物品与元宇宙物品的一一对应。
  \item \textbf{DAO:}作为社区“操作系统”,提供全球性、去中心化协作层,减少第三方依赖。
\end{itemize}
\end{HowBox}

\subsubsection{3) 元宇宙真实性:物理 vs. 社会(Authenticity)}
\begin{KeyBox}
\textbf{元宇宙的“真实性”问题(文中强调的深层点):}
\begin{itemize}
  \item \textbf{物理真实性:}更多由 VR/AR 等技术去逼近与模拟。
  \item \textbf{社会关系真实性:}Web 3.0 通过确权与规则透明(智能合约/通证)提升秩序稳定性,减少“中心化服务器当上帝”的不确定性。
  \item \textbf{稀缺性机制:}Web 2.0 游戏由服务器决定产出与分配;Web 3.0 通过智能合约限制通证生成条件与上限,形成更稳定的稀缺性与规则边界。
\end{itemize}
\end{KeyBox}
% ========== 到这里结束 ==========

\end{document}
