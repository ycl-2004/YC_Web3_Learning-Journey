% !TEX program = xelatex
\documentclass[11pt,a4paper]{article}

% =========================
% style.tex  (ALL STYLES)
% =========================

% ---------- Page & Fonts ----------
\usepackage[a4paper,margin=1in]{geometry}
\usepackage{fontspec}
\usepackage{xeCJK}

% CJK font fallbacks (Overleaf 常見可用 Noto)
\IfFontExistsTF{Noto Serif CJK TC}{
  \setCJKmainfont{Noto Serif CJK TC}
}{
  \IfFontExistsTF{Noto Sans CJK TC}{
    \setCJKmainfont{Noto Sans CJK TC}
  }{
    \setCJKmainfont{SimSun}
  }
}
\setmainfont{TeX Gyre Termes}

% ---------- Packages ----------
\usepackage{hyperref}
\usepackage{microtype}
\usepackage{booktabs}
\usepackage{tabularx}
\usepackage{enumitem}
\usepackage{tikz}
\usetikzlibrary{positioning}
\usepackage{xcolor}
\usepackage{tcolorbox}
\tcbuselibrary{skins,breakable}
\usepackage{titlesec}
\usepackage{fancyhdr}

% ---------- Color Palette ----------
\definecolor{ink}{HTML}{1F2937}
\definecolor{muted}{HTML}{6B7280}
\definecolor{brand}{HTML}{2563EB}
\definecolor{brand2}{HTML}{0EA5E9}
\definecolor{soft}{HTML}{EFF6FF}
\definecolor{warn}{HTML}{F59E0B}
\definecolor{good}{HTML}{10B981}
\definecolor{rose}{HTML}{F43F5E}
\definecolor{paper}{HTML}{FFFFFF}

% ---------- Hyperref ----------
\hypersetup{
  colorlinks=true,
  linkcolor=brand,
  urlcolor=brand2,
  citecolor=brand
}

% ---------- Global typography ----------
\setlength{\parindent}{0pt}
\setlength{\parskip}{6pt}

% Clean lists inside boxes
\setlist[itemize]{itemsep=4pt, topsep=4pt, leftmargin=1.4em}
\setlist[enumerate]{itemsep=4pt, topsep=4pt, leftmargin=1.6em}

% ---------- Header ----------
\pagestyle{fancy}
\fancyhf{}
\lhead{\textcolor{muted}{Web3 / DApps 筆記}}
\rhead{\textcolor{muted}{\thepage}}
\renewcommand{\headrulewidth}{0pt}

% ---------- Section style ----------
\titleformat{\section}{\Large\bfseries\color{ink}}{\thesection}{0.8em}{}
\titleformat{\subsection}{\large\bfseries\color{ink}}{\thesubsection}{0.8em}{}
\titleformat{\subsubsection}{\normalsize\bfseries\color{ink}}{\thesubsubsection}{0.8em}{}
\titlespacing*{\section}{0pt}{1.0em}{0.4em}
\titlespacing*{\subsection}{0pt}{0.8em}{0.3em}
\titlespacing*{\subsubsection}{0pt}{0.5em}{0.2em}

% ---------- tcolorbox base ----------
\tcbset{
  enhanced,
  breakable,
  colback=paper,
  colframe=brand!18,
  boxrule=0.5pt,
  arc=10pt,
  left=10pt,right=10pt,top=10pt,bottom=10pt,
  before skip=10pt, after skip=10pt
}

% ---------- Box components ----------
\newtcolorbox{KeyBox}[1]{
  colback=paper,
  colframe=brand!18,
  borderline west={4pt}{0pt}{brand},
  title=\textbf{#1},
  coltitle=ink,
  fonttitle=\bfseries,
  colbacktitle=paper,
  boxed title style={boxrule=0pt, colframe=paper}
}

\newtcolorbox{DefBox}[1]{
  colback=paper,
  colframe=brand2!18,
  borderline west={4pt}{0pt}{brand2},
  title=\textbf{#1},
  coltitle=ink,
  fonttitle=\bfseries,
  colbacktitle=paper,
  boxed title style={boxrule=0pt, colframe=paper}
}

\newtcolorbox{HowBox}[1]{
  colback=paper,
  colframe=good!18,
  borderline west={4pt}{0pt}{good},
  title=\textbf{#1},
  coltitle=ink,
  fonttitle=\bfseries,
  colbacktitle=paper,
  boxed title style={boxrule=0pt, colframe=paper}
}

\newtcolorbox{WarnBox}[1]{
  colback=paper,
  colframe=warn!22,
  borderline west={4pt}{0pt}{warn},
  title=\textbf{#1},
  coltitle=ink,
  fonttitle=\bfseries,
  colbacktitle=paper,
  boxed title style={boxrule=0pt, colframe=paper}
}

% ---------- Tag pill ----------
\newcommand{\tagpill}[1]{%
  \tikz[baseline=(X.base)]\node[
    fill=soft, draw=brand!25, rounded corners=6pt,
    inner xsep=7pt, inner ysep=2.5pt
  ](X){\small\textcolor{brand}{\textbf{#1}}};%
}

\renewcommand{\arraystretch}{1.18}

% =========================
% DeFi Notes (NEW FILE)
% =========================
\begin{document}

% ---------- Title ----------
\begin{center}
  {\LARGE \textbf{DeFi(去中心化金融)重點筆記}}\\[6pt]
  {\large \textcolor{muted}{只保留:是什麼|用途|用在哪|怎麼 work|例子}}\\[10pt]
  \tagpill{DeFi}\quad \tagpill{Wallet}\quad \tagpill{DEX}\quad \tagpill{AMM}\quad \tagpill{Uniswap}\quad
  \tagpill{Liquidity}\quad \tagpill{TVL}
\end{center}

\begin{KeyBox}{快速總結(考前 30 秒)}
\begin{itemize}
  \item \textbf{一句話}:DeFi 用\textbf{智能合約}把「借貸、交易、資產管理」搬到鏈上,讓\textbf{個人即帳戶}、資產不再依賴平台托管。
  \item \textbf{完整金融版圖要含}:不只借貸/收益協議,還要把\textbf{錢包 + 交易所(DEX/CEX)}納入理解,才看得見 Web3 的金融基礎設施。
  \item \textbf{錢包的本質}:\textbf{私鑰的容器}(丟助記詞=丟錢包);地址像卡號,私鑰像最終控制權。
  \item \textbf{DEX 的本質}:不托管資金,交易用\textbf{私鑰簽名 + 合約執行};資產直接到錢包,無須「充值/提現」。
  \item \textbf{AMM 的本質}:用演算法報價做市(常見 CPMM:\(xy=k\)),\textbf{流動性}越大,滑點越低;套利讓價格回到市場。
\end{itemize}
\end{KeyBox}

\tableofcontents

% =========================================================
\section{DeFi 是什麼?(定位與範圍)}

\begin{DefBox}{一句話定義}
\textbf{DeFi(Decentralized Finance)}是基於區塊鏈的去中心化金融系統:透過\textbf{智能合約}在鏈上完成交易、借貸、資產管理等金融功能,減少對銀行/券商等中心化機構的依賴。
\end{DefBox}

\begin{KeyBox}{用途/用在哪(建立 Web3 金融地圖)}
\begin{itemize}
  \item \textbf{用途}:讓資產控制權回到個人;規則可驗證、流程可自動化;降低平台托管風險。
  \item \textbf{用在哪}:面向消費者的 DApps 中,DeFi 是最早且數量最多的一類。
  \item \textbf{範圍提醒(文章主張)}:不要只把 DeFi 狹義理解為「借貸/收益」;\textbf{錢包與交易所}也屬於 Web3 金融基礎設施的一部分。
\end{itemize}

\textbf{文中例子(Web2 vs Web3)}
\begin{itemize}
  \item Web2:你可能同時有\textbf{支付寶、微信、網商銀行}等帳戶;資產與權限由不同服務商平台掌控與解釋。
  \item Web3:錢包可脫離平台存在;\textbf{個人即帳戶本身},資產統一歸屬個人。
\end{itemize}
\end{KeyBox}

\begin{HowBox}{怎麼 work(直覺對比:Web2 平台錢包 vs Web3 私鑰錢包)}
\begin{itemize}
  \item \textbf{Web2}:帳戶是「平台給你的」;資產(餘額/券/資料/商品與服務權益)都依附平台;平台決定認證、權限、交易型態。
  \item \textbf{Web3}:區塊鏈作為去中心化底座;錢包不是平台帳戶,而是\textbf{私鑰系統};你用私鑰簽名完成資產操作,平台無須成為托管者。
\end{itemize}
\end{HowBox}

% =========================================================
\section{Web 3.0 錢包(Wallet)}

\begin{DefBox}{一句話定義:錢包本質是什麼?}
\textbf{加密錢包}的本質是\textbf{保存私鑰的容器}:它讓你能簽名交易、證明你對鏈上資產的控制權。  
(廣義可理解為「存放加密資產的容器」;但核心仍是私鑰。)
\end{DefBox}

\begin{KeyBox}{用途/用在哪}
\begin{itemize}
  \item \textbf{用途}:接收/保管/轉移加密資產;在 Web3 中對應「存款/資產保管」功能。
  \item \textbf{用在哪}:比特幣等加密資產的\textbf{轉帳、歸屬、存儲}需要靠錢包完成(點對點支付 + 公開可驗證的交易記錄)。
\end{itemize}
\end{KeyBox}

\begin{HowBox}{錢包怎麼 work(地址 / 私鑰 / 助記詞)}
\textbf{組成}
\begin{itemize}
  \item \textbf{地址}:由公鑰哈希生成,用於收款(類比「銀行卡號」)。
  \item \textbf{私鑰}:驗證身份並操作資產的鑰匙(類比「最終控制權」)。
\end{itemize}

\textbf{非托管錢包(例:MetaMask)的助記詞機制(文章版流程)}
\begin{enumerate}
  \item 建立錢包時生成 \textbf{12 個英文單詞}的助記詞(BIP39,來源於 2048 詞庫)。
  \item 助記詞對應數字序列(Seed Integer)。
  \item Seed Integer 經 SHA256 生成私鑰;再經 ECDSA 等生成公鑰/地址。
\end{enumerate}

\textbf{文中例子(跨錢包恢復)}
\begin{quote}
MetaMask 生成的助記詞,輸入到 imToken 仍可進入並控制同一資產。\newline
因此錢包 App 只是「殼」,\textbf{助記詞才是錢包本體}。
\end{quote}
\end{HowBox}

\begin{KeyBox}{分類(一定要會背的兩軸)}
\textbf{按控制權:托管 vs 非托管}
\begin{itemize}
  \item \textbf{托管錢包}:私鑰由平台掌握;登入多是「手機/信箱/帳密」形式(例:各種中心化交易所的錢包)。
  \item \textbf{非托管錢包}:私鑰完全由自己掌握(例:MetaMask)。
\end{itemize}

\textbf{按存儲介質:}
\begin{itemize}
  \item 網頁錢包、桌面錢包、手機錢包、\textbf{硬體錢包}(最推薦之一)、紙錢包
\end{itemize}

\textbf{熱/冷錢包:}
\begin{itemize}
  \item 私鑰實際保存地在網上:\textbf{熱錢包}
  \item 私鑰實際保存地不聯網:\textbf{冷錢包}
\end{itemize}
\end{KeyBox}

\begin{WarnBox}{風險/注意(文章明確點名)}
\begin{itemize}
  \item \textbf{資產兼容性}:不同錢包能存的資產種類不同;強行存入不支持資產可能導致\textbf{資產丟失}。
  \item \textbf{助記詞/私鑰安全}:助記詞就是錢包全部;丟失=丟錢包(沒有找回機制)。
  \item \textbf{托管錢包不等於去中心化}:公鑰/私鑰在托管方手中,依然存在平台風險。
\end{itemize}
\end{WarnBox}

% =========================================================
\section{DEX vs CEX(去中心化交易所 vs 中心化交易所)}

\begin{DefBox}{一句話定義}
\textbf{DEX(去中心化交易所)}是鏈上交易基礎設施:不要求用戶把資金與個資轉入交易所,只用智能合約完成撮合與結算,交易發生在參與者之間。  
\textbf{CEX(中心化交易所)}則由平台負責充提、撮合、結算與托管(像把交易所+券商+投行功能集合在一個中心化系統)。
\end{DefBox}

\begin{KeyBox}{用在哪(文章提到的例子)}
\begin{itemize}
  \item \textbf{CEX 代表}:Binance、Coinbase、FTX(文章列舉)
  \item \textbf{DEX 代表}:Uniswap(交易量最大的 DEX;本質是以太坊上的協議)
\end{itemize}
\end{KeyBox}

\begin{HowBox}{怎麼 work(核心流程差異)}
\textbf{交易所核心環節(文章)}:充提、下單、訂單撮合、資金結算、提現
\begin{itemize}
  \item \textbf{CEX}:撮合與結算在平台內完成(不上鏈),速度快、深度好,但資產集中托管。
  \item \textbf{DEX}:上述環節\textbf{全部上鏈}由智能合約執行;下單需用\textbf{私鑰簽名};撮合成功後資產\textbf{直接到錢包},無須提現。
\end{itemize}

\textbf{文中例子(資產是否觸碰)}
\begin{itemize}
  \item CEX:用戶資金集中在交易所帳戶,資金量大→更容易成為黑客目標。
  \item DEX:平台不托管資金;只在交易瞬間由合約撮合與驗證;資產直接回到錢包。
\end{itemize}
\end{HowBox}

\begin{WarnBox}{取捨(文章點出的現實問題)}
\begin{itemize}
  \item \textbf{DEX}:完全鏈上→可能流動性較差、成本高、速度慢。
  \item \textbf{CEX}:集中托管→一旦出事可能影響幾乎所有用戶。
\end{itemize}
\end{WarnBox}

% =========================================================
\section{Uniswap(DEX 協議)}

\begin{DefBox}{是什麼}
\textbf{Uniswap}並非「公司式交易所」,而是部署在以太坊上的一套\textbf{協議(智能合約集合)}。  
V1(2018)到 V3 的核心升級主線:\textbf{提高資本使用效率}。
\end{DefBox}

\begin{KeyBox}{用途/用在哪(文章列出的設計目標)}
\begin{itemize}
  \item \textbf{易用性}:Token A 換 Token B,可\textbf{一筆交易}完成
  \item \textbf{Gas 高利用率}:相對主流 DEX,交易 Gas 較低
  \item \textbf{零抽租}:協議不抽走交易費;費用回到\textbf{流動性提供者(LP)}
  \item \textbf{抗審查}:上架新通證門檻低(任何人可上架任意通證)
\end{itemize}

\textbf{文中例子(易用性對比)}
\begin{quote}
在某些交易所你可能要兩筆:A 先換成 ETH/DAI 等媒介,再換成 B;\newline
而 Uniswap 一筆交易即可完成 A $\rightarrow$ B。
\end{quote}
\end{KeyBox}

% =========================================================
\section{AMM(自動做市商)與 CPMM:\(xy=k\)}

\begin{DefBox}{一句話定義}
\textbf{AMM(Automated Market Maker)}是 DEX 的核心引擎:用演算法自動報價並與交易者成交;  
\textbf{流動性質押(Liquidity Staking)}則是 AMM 的能量來源(提供資產池彈藥)。
\end{DefBox}

\begin{KeyBox}{AMM 至少要滿足的 4 件事(文章列舉)}
\begin{itemize}
  \item 持有兩種資產(雙向報價)
  \item 資產池可充值/提現
  \item 能隨市場自動調價
  \item 能靠交易費等獲利/分配收益
\end{itemize}
\end{KeyBox}

\begin{HowBox}{怎麼 work(CPMM:恒定乘積做市)}
最常見 AMM 模型:\textbf{恒定乘積}(CPMM)
\[
x \cdot y = k
\]
其中 \(x\)、\(y\) 是池內兩種資產數量,\(k\) 維持不變;交易會改變 \(x,y\),價格隨之自動調整。

\textbf{文中例子(“無人超市:雞蛋 × 牛奶 = 5000”)}
\begin{itemize}
  \item 初始:100 顆雞蛋、50 瓶牛奶,\(100 \times 50 = 5000\)。
  \item 買 2 顆雞蛋後:雞蛋剩 98,牛奶需變為 \(5000/98 \approx 51.02\),所以要付 \(51.02-50=1.02\) 瓶牛奶。
  \item 若一次買走 50 顆雞蛋:雞蛋剩 50,牛奶需 \(5000/50=100\),要付 50 瓶牛奶(價格急升)。
  \item \textbf{套利者}會在外部市場換到雞蛋再來池子換牛奶,靠套利修正價格,使池內價格回到合理區間。
\end{itemize}
\end{HowBox}

\begin{WarnBox}{注意:流動性不足會放大滑點(文章給的直覺算例)}
\begin{itemize}
  \item 若池子很小(100/50),買 2 顆雞蛋就出現約 2\% 的價差。
  \item 若池子很大(100 萬/50 萬),同樣買 2 顆雞蛋,報價接近 \(1.000002\)(約 0.002\% 的價差)。
  \item \textbf{結論}:流動性越大,價格越穩、滑點越低。
\end{itemize}
\end{WarnBox}

% =========================================================
\section{流動性質押(Liquidity Staking / Mining)與 TVL}

\begin{DefBox}{一句話定義}
\textbf{流動性質押}(又稱流動性挖礦)是 LP 把資產質押進 AMM 資產池,為交易/借貸提供流動性以獲取回報。  
\textbf{TVL(Total Value Locked)}是協議中被鎖定的資產總額。
\end{DefBox}

\begin{HowBox}{怎麼 work(收益從哪來)}
\begin{enumerate}
  \item LP 把資產存入池子(提供流動性)
  \item 交易者在 DEX 交易支付\textbf{手續費}
  \item 手續費按 LP 佔池子總量的比例分配
\end{enumerate}

\textbf{文中例子(按占比分成)}
\begin{quote}
若你提供的質押價值佔資產池 1\%,那你可獲得該池交易費總收益的 1\%。
\end{quote}

\textbf{文中歷史例子}
\begin{itemize}
  \item IDEX(2017)提出流動性質押概念
  \item Compound 在 2020 年 DeFi Summer 引入流動性質押
\end{itemize}
\end{HowBox}

% =========================================================
\section{Uniswap V1 / V2 / V3:交易機制進化史}

\begin{HowBox}{V1 \texorpdfstring{$\rightarrow$}{->} V2 \texorpdfstring{$\rightarrow$}{->} V3(只記升級主線)}
\textbf{V1(基礎 CPMM)}
\begin{itemize}
  \item 典型恒定乘積池,主要靠套利修正價格偏離
\end{itemize}

\textbf{V2(TWAP + Flash Swap)}
\begin{itemize}
  \item \textbf{TWAP(時間加權平均價)}:用一段時間的累積價格/持續時間得到平均價;\textbf{平均時間越長、流動性越高},操縱成本越高。
  \item \textbf{Flash Swap}:可先取走想要的 ERC-20 通證,\textbf{交易結束時歸還}即可(降低套利前置資本門檻)。
\end{itemize}

\textbf{文中例子(Flash Swap 的意義)}
\begin{quote}
如果套利者沒有本錢也借不到資產,價格異常可能無法及時被修正;\newline
Flash Swap 讓套利可在無前置成本下完成,促進價格回歸。
\end{quote}

\textbf{V3(集中流動性:Concentrated Liquidity)}
\begin{itemize}
  \item V2:LP 資金沿 \(xy=k\) 的\textbf{全價格區間}均勻分布(理論 0 到無窮)。
  \item V3:LP 可自選\textbf{特定價格區間}提供流動性,把資金集中在交易最活躍區間,提高資本效率並\textbf{降低滑點}。
\end{itemize}
\end{HowBox}

% =========================================================
\section{DeFi 的未來:對傳統金融功能的映射與缺口}

\begin{KeyBox}{金融「三駕馬車」映射(文章框架)}
\begin{itemize}
  \item \textbf{銀行}:借貸 $\rightarrow$ 借貸池(Lending Pool);存款 $\rightarrow$ 錢包
  \item \textbf{券商}:做市/經紀 $\rightarrow$ AMM + LP(已相對成熟)
  \item \textbf{保險}:相對容易遷移,但難在\textbf{定價與核保}
\end{itemize}
\end{KeyBox}

\begin{HowBox}{文章提到的對應案例(只留點名 + 直覺)}
\textbf{借貸池(銀行借貸對應)}
\begin{itemize}
  \item Aave(規模最大)、MakerDAO、Compound、Anchor
  \item 無抵押借貸:TrueFi、Wing(規模小,偏向 Web3 資管公司,難以評估實體經濟信用)
\end{itemize}

\textbf{投行/資管探索}
\begin{itemize}
  \item Ondo:把鏈上資產彙集後,按風險收益拆分重包裝成多種產品
  \item Cult DAO:去中心化風投;資金來自 CULT 交易費;持幣者投票決定投資
\end{itemize}

\textbf{保險(產品點名)}
\begin{itemize}
  \item Nexus Mutual、Unslashed、inSure、Solace、Bridge Mutual
  \item 定價多仍靠中心化方式(如精算);核保常用投票(或委託 Kleros 這類陪審團機制)
\end{itemize}
\end{HowBox}

\begin{WarnBox}{最大瓶頸:KYC / 身份與信用(文章收束點)}
\begin{itemize}
  \item 金融業關鍵要素之一是 \textbf{KYC};DeFi 若要擴到更複雜場景,需要更成熟的鏈上身份與信用機制。
  \item 文中引用觀點:需要一種與人深度綁定的\textbf{身份通證},能在加密世界還原人際關係並綁定個人信用(提及維塔利克 2022/5 的相關論述)。
\end{itemize}
\end{WarnBox}

% =========================================================
\section{快速複習題(自測用)}

\begin{KeyBox}{Q1:為什麼文章說理解 DeFi 要把錢包與交易所也算進來?}
A:因為 Web3 的金融活動(存儲、轉移、交易、借貸)都離不開\textbf{錢包(私鑰/資產控制)}與\textbf{交易所(資產交換)}這兩個基礎設施;只看借貸會漏掉「資產如何被你真正擁有與交換」。
\end{KeyBox}

\begin{KeyBox}{Q2:錢包的兩個核心組件是什麼?}
A:\textbf{地址}(收款,類比卡號)+\textbf{私鑰}(簽名與控制權,丟了找不回)。
\end{KeyBox}

\begin{KeyBox}{Q3:托管錢包 vs 非托管錢包差在哪?}
A:\textbf{私鑰誰掌握}。托管錢包私鑰在平台;非托管錢包私鑰在自己(例:MetaMask)。
\end{KeyBox}

\begin{KeyBox}{Q4:DEX 為什麼說「平台不碰資產」?}
A:因為用戶資產不需要充值到平台;交易靠\textbf{私鑰簽名}與\textbf{智能合約}撮合驗證,資產\textbf{直接回到錢包},無須提現。
\end{KeyBox}

\begin{KeyBox}{Q5:AMM 的 CPMM 為什麼需要套利者?}
A:因為 \(xy=k\) 會在大額交易後造成價格偏離;套利者利用外部市場價格差來交易,從而把池內價格推回合理區間。
\end{KeyBox}

\begin{WarnBox}{最後一句(考前記這句)}
Web2 的錢包是「平台帳戶」;Web3 的錢包是「\textbf{私鑰系統}」。\newline
DEX 用\textbf{合約}替代平台撮合,AMM 用\textbf{演算法}替代訂單簿;流動性與套利共同維持市場可用性。
\end{WarnBox}

\end{document}
