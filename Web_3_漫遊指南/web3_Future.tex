% !TEX program = xelatex
\documentclass[11pt,a4paper]{article}

% =========================
% style.tex  (ALL STYLES)
% =========================

% ---------- Page & Fonts ----------
\usepackage[a4paper,margin=1in]{geometry}
\usepackage{fontspec}
\usepackage{xeCJK}

% CJK font fallbacks (Overleaf 常見可用 Noto)
\IfFontExistsTF{Noto Serif CJK TC}{
  \setCJKmainfont{Noto Serif CJK TC}
}{
  \IfFontExistsTF{Noto Sans CJK TC}{
    \setCJKmainfont{Noto Sans CJK TC}
  }{
    \setCJKmainfont{SimSun}
  }
}
\setmainfont{TeX Gyre Termes}

% ---------- Packages ----------
\usepackage{hyperref}
\usepackage{microtype}
\usepackage{booktabs}
\usepackage{tabularx}
\usepackage{enumitem}
\usepackage{tikz}
\usetikzlibrary{positioning}
\usepackage{xcolor}
\usepackage{tcolorbox}
\tcbuselibrary{skins,breakable}
\usepackage{titlesec}
\usepackage{fancyhdr}

% ---------- Color Palette ----------
\definecolor{ink}{HTML}{1F2937}
\definecolor{muted}{HTML}{6B7280}
\definecolor{brand}{HTML}{2563EB}
\definecolor{brand2}{HTML}{0EA5E9}
\definecolor{soft}{HTML}{EFF6FF}
\definecolor{warn}{HTML}{F59E0B}
\definecolor{good}{HTML}{10B981}
\definecolor{rose}{HTML}{F43F5E}
\definecolor{paper}{HTML}{FFFFFF}

% ---------- Hyperref ----------
\hypersetup{
  colorlinks=true,
  linkcolor=brand,
  urlcolor=brand2,
  citecolor=brand
}

% ---------- Global typography ----------
\setlength{\parindent}{0pt}
\setlength{\parskip}{6pt}

% Clean lists inside boxes
\setlist[itemize]{itemsep=4pt, topsep=4pt, leftmargin=1.4em}
\setlist[enumerate]{itemsep=4pt, topsep=4pt, leftmargin=1.6em}

% ---------- Header ----------
\pagestyle{fancy}
\fancyhf{}
\lhead{\textcolor{muted}{Web3 / DApps 筆記}}
\rhead{\textcolor{muted}{\thepage}}
\renewcommand{\headrulewidth}{0pt}

% ---------- Section style ----------
\titleformat{\section}{\Large\bfseries\color{ink}}{\thesection}{0.8em}{}
\titleformat{\subsection}{\large\bfseries\color{ink}}{\thesubsection}{0.8em}{}
\titleformat{\subsubsection}{\normalsize\bfseries\color{ink}}{\thesubsubsection}{0.8em}{}
\titlespacing*{\section}{0pt}{1.0em}{0.4em}
\titlespacing*{\subsection}{0pt}{0.8em}{0.3em}
\titlespacing*{\subsubsection}{0pt}{0.5em}{0.2em}

% ---------- tcolorbox base ----------
\tcbset{
  enhanced,
  breakable,
  colback=paper,
  colframe=brand!18,
  boxrule=0.5pt,
  arc=10pt,
  left=10pt,right=10pt,top=10pt,bottom=10pt,
  before skip=10pt, after skip=10pt
}

% ---------- Box components ----------
\newtcolorbox{KeyBox}[1]{
  colback=paper,
  colframe=brand!18,
  borderline west={4pt}{0pt}{brand},
  title=\textbf{#1},
  coltitle=ink,
  fonttitle=\bfseries,
  colbacktitle=paper,
  boxed title style={boxrule=0pt, colframe=paper}
}

\newtcolorbox{DefBox}[1]{
  colback=paper,
  colframe=brand2!18,
  borderline west={4pt}{0pt}{brand2},
  title=\textbf{#1},
  coltitle=ink,
  fonttitle=\bfseries,
  colbacktitle=paper,
  boxed title style={boxrule=0pt, colframe=paper}
}

\newtcolorbox{HowBox}[1]{
  colback=paper,
  colframe=good!18,
  borderline west={4pt}{0pt}{good},
  title=\textbf{#1},
  coltitle=ink,
  fonttitle=\bfseries,
  colbacktitle=paper,
  boxed title style={boxrule=0pt, colframe=paper}
}

\newtcolorbox{WarnBox}[1]{
  colback=paper,
  colframe=warn!22,
  borderline west={4pt}{0pt}{warn},
  title=\textbf{#1},
  coltitle=ink,
  fonttitle=\bfseries,
  colbacktitle=paper,
  boxed title style={boxrule=0pt, colframe=paper}
}

% ---------- Tag pill ----------
\newcommand{\tagpill}[1]{%
  \tikz[baseline=(X.base)]\node[
    fill=soft, draw=brand!25, rounded corners=6pt,
    inner xsep=7pt, inner ysep=2.5pt
  ](X){\small\textcolor{brand}{\textbf{#1}}};%
}

\renewcommand{\arraystretch}{1.18}

\begin{document}

% ---------- Title ----------
\begin{center}
  {\LARGE \textbf{Web 3.0 未來展望重點筆記}}\\[6pt]
  {\large \textcolor{muted}{只保留:是什麼|用途|用在哪|怎麼 work|文中觀點/例子}}\\[10pt]
  \tagpill{Web 3.0}\quad
  \tagpill{去中心化}\quad
  \tagpill{监管}\quad
  \tagpill{中国路径}\quad
  \tagpill{不可能三角}
\end{center}

\begin{KeyBox}{快速總結(考前 30 秒)}
\begin{itemize}
  \item \textbf{一句話}:Web 3.0 的未來不是「完全去中心化、無監管」,而是\textbf{在現實法律與社會結構下,用去中心化技術保障數字產權與契約關係}。
  \item \textbf{核心張力}:效率 vs 公平 vs 安全(不可能三角),任何實際落地都必須取捨。
  \item \textbf{中國路徑}:強調\textbf{穩健、合規、產業落地},可能成為全球最成熟、最快發展的 Web 3.0 生態。
  \item \textbf{關鍵判斷標準}:是否真的提升生產效率、形成可持續的經濟閉環,而非僅靠敘事與炒作。
\end{itemize}
\end{KeyBox}

\tableofcontents

% =========================================================
\section{是什麼(Web 3.0 的未來定位)}

\begin{DefBox}{Web 3.0 的現實主義定義}
Web 3.0 並非一個\textbf{完全去中心化、去監管}的理想烏托邦,而更可能是:\\
在\textbf{由核心節點主導的區塊鏈網路}上,利用\textbf{去中心化帳本與加密技術},保障不同經濟主體之間的\textbf{數字產權、商業價值與契約關係}。
\end{DefBox}

\begin{KeyBox}{與「原教旨去中心化」的差異}
\begin{itemize}
  \item Web 3.0 的目標是\textbf{奪回數據所有權},而不是否定所有中心與規則。
  \item 完全去中心化在效率與擴展性上不可持續。
  \item 現實世界的法律、文化與經濟結構,仍然是 Web 3.0 必須嵌套的基礎。
\end{itemize}
\end{KeyBox}

% =========================================================
\section{用途(Web 3.0 能解決什麼)}

\begin{KeyBox}{Web 3.0 的實際用途}
\begin{itemize}
  \item \textbf{數字產權保護}:防止數據、資產與價值被單一平台攫取。
  \item \textbf{契約關係強化}:用智能合約降低交易摩擦與信任成本。
  \item \textbf{生產關係優化}:讓更多個體直接參與價值創造並獲得回報。
  \item \textbf{產業數位化基礎設施}:為金融、物流、製造、政務等提供可信底層。
\end{itemize}
\end{KeyBox}

\begin{WarnBox}{用途判斷紅線}
如果一個 Web 3.0 應用:
\begin{itemize}
  \item 無法提升實際生產效率;
  \item 無法形成經濟閉環;
  \item 僅依賴敘事與價格上漲;
\end{itemize}
那它更可能是\textbf{投機工具,而非生產模式革新}。
\end{WarnBox}

% =========================================================
\section{用在哪(落地場景與地域差異)}

\subsection{中國 Web 3.0 的適用場景}

\begin{KeyBox}{中國路徑的核心特徵}
\begin{itemize}
  \item \textbf{本土化定制能力}:深度貼合複雜、多元的用戶需求。
  \item \textbf{快速響應能力}:在高度競爭的垂直市場中快速迭代。
  \item \textbf{線上線下結合}:Web 技術深度嵌入實體經濟(O2O)。
  \item \textbf{生態構建能力}:內容 + 社交 + 產業的整體協同。
\end{itemize}
\end{KeyBox}

\begin{HowBox}{中國 Web 3.0 的主要落地方向}
\begin{itemize}
  \item 聯盟鏈:內容版權、供應鏈金融、保險、稅務、司法存證。
  \item 政務與產業:物流溯源、商品防偽、製造業數位化。
  \item X-to-Earn 類型的「共享經濟」延伸(如出行、技能服務)。
\end{itemize}
\end{HowBox}

% =========================================================
\section{怎麼 work(效率、公平與監管)}

\subsection{不可能三角}

\begin{DefBox}{區塊鏈「不可能三角」}
去中心化、可擴展性、安全性三者無法同時最大化;\\
兼顧兩者,必然犧牲第三者。
\end{DefBox}

\begin{KeyBox}{現實取捨}
\begin{itemize}
  \item 提升效率 $\Rightarrow$ 犧牲部分去中心化。
  \item 強化安全 $\Rightarrow$ 增加運算與能源成本。
  \item 絕對去中心化 $\Rightarrow$ 成為效率與擴展性的天敵。
\end{itemize}
\end{KeyBox}

\subsection{監管如何介入 Web 3.0}

\begin{HowBox}{監管的現實路徑}
\begin{itemize}
  \item Web 3.0 \textbf{不是去監管},而是改變監管介入方式。
  \item 難以直接監管 DApp,應轉向\textbf{使用者與行為層面}。
  \item 合約設計中主動嵌入合規規則,是可行方向。
\end{itemize}
\end{HowBox}

\begin{WarnBox}{監管的雙刃劍}
\begin{itemize}
  \item 過度監管:抑制創新。
  \item 無監管:放大風險、破壞信任。
  \item 合規是 Web 3.0 進入主流的必要條件。
\end{itemize}
\end{WarnBox}

% =========================================================
\section{文中例子(觀點型案例)}

\subsection{中國企業與聯盟鏈}

\begin{DefBox}{聯盟鏈的角色}
由企業、政府與機構共同維護的區塊鏈,用於商業與公共服務場景,兼顧效率、合規與可信度。
\end{DefBox}

\begin{KeyBox}{文中提及的方向}
\begin{itemize}
  \item 內容版權與數字存證
  \item 金融(保險、債券、供應鏈金融)
  \item 政務與司法
  \item 商品防偽與物流
\end{itemize}
\end{KeyBox}

\subsection{X-to-Earn 的中國想像}

\begin{HowBox}{DAO 化的出行示例(文中假想)}
\begin{itemize}
  \item 車與司機符合監管即可直接撮合需求。
  \item 定價與派單透明,避免平台算法黑箱。
  \item DAO 僅收取極低比例維護費用。
\end{itemize}
\end{HowBox}

% =========================================================
\section{最後總結}

\begin{WarnBox}{最後一句話(考前必背)}
Web 3.0 的真正未來,不在於極端去中心化的理想主義,\\
而在於\textbf{在現實法律與社會結構中,理性運用去中心化技術,提升產權保障、契約效率與生產關係}。
\end{WarnBox}

\end{document}
